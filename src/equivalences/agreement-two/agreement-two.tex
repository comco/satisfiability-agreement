\begin{definition}
Let $\seq{\relD, \relE} \subseteq \domAA$ be a sequence of two
equivalence relations on $\domA$.
The relation $\relD$ is \emph{finer} than the relation $\relE$ if every
equivalence class of $\relD$ is a subset of some equivalence class of $\relE$.
Equivalently, $\relD \subseteq \relE$. Equivalently,
\[
  (\forall \ea \in \domA)(\forall \eb \in \domA) \relD(\ea, \eb) \limp
  \relE(\ea, \eb).
\]
If $\relD$ is finer than $\relE$, then $\relE$ is \emph{coarser} than $\relD$.
The sequence $\seq{\relD, \relE}$ is a sequence of equivalence relations on
$\domA$ in \emph{refinement} if $\relD$ is finer $\relE$.

The sequence $\seq{\relD, \relE}$ is a sequence of equivalence relations in
\emph{global agreement} if either $\relD$ is finer than $\relE$ or $\relE$ is
finer than $\relD$.

The sequence $\seq{\relD, \relE}$ is a sequence of equivalence relations
\emph{in local agreement} if for every $\ea \in \domA$, 
either $\relD[\ea] \subseteq \relE[\ea]$ or $\relE[\ea] \subseteq \relD[\ea]$.
Equivalently, no two equivalence classes $\relE[\ea]$ and $\relD[\eb]$ properly
intersect.
Equivalently
\[
  (\forall \ea \in \domA)
  \left[(\forall \eb \in \domA) \relD(\ea, \eb) \limp \relE(\ea, \eb)\right]
  \lor \left[(\forall \eb \in \domA) \relE(\ea, \eb) \limp
  \relD(\ea, \eb)\right].
\]
\end{definition}

Let $\ES = \seq{\sd, \se}$ be a predicate signature consisting of the two binary
predicate symbols $\sd$ and $\se$.
Let $\StrA$ is an $\ES$-structure and suppose that $\sd$ and $\se$ are
interpreted in $\StrA$ as equivalence relations on $\domA$.
Let $\relD = \at\StrA\sd$ and $\relE = \at\StrA\se$ be the interpretations of
the two symbols.

\begin{definition}
Define the $\vFoF2\ES$-sentence $\gls{frefine-d-e}$ by:
\[
  \frefine{\sd,\se} = \forall\xx \forall\yy (\sd(\xx, \yy) \limp \se(\xx, \yy)).
\]
\end{definition}
Then $\seq{\relD, \relE}$ is in refinement iff 
$\StrA \vDash \frefine{\sd,\se}$.

\begin{definition}
Define the $\vFoF2\ES$-sentence $\gls{fglobal-d-e}$ by:
\[
  \fglobal{\sd,\se} = \frefine{\sd,\se} \lor \frefine{\se,\sd}.
\]
\end{definition}
Then $\seq{\relD, \relE}$ is in global agreement iff
$\StrA \vDash \fglobal{\sd,\se}$.

\begin{definition}
Define the $\vFoF2\ES$-sentence $\gls{flocal-d-e}$ by:
\[
\flocal{\sd,\se} =
    \forall \xx 
    ((\forall \yy \sd(\xx,\yy) \limp \se(\xx,\yy))
    \lor 
    (\forall \yy \se(\xx,\yy) \limp \sd(\xx,\yy)).
\]
\end{definition}
Then $\seq{\relD, \relE}$ is in global agreement iff
$\StrA \vDash \flocal{\sd,\se}$.

\begin{lemma}\label{lem:local-2}
If $\seq{\relD, \relE} \subseteq \domAA$ is a sequence two
equivalence relations on $\domA$, then
it is in local agreement iff $\relL = \relD \cup \relE$ is an equivalence
relation on $\domA$.
\end{lemma}
\begin{proof}
The union of two equivalence relations on $\domA$ is a reflexive and symmetric
relation.

First suppose that $\relD$ and $\relE$ are in local agreement.
We claim that $\relL$ is transitive.
Let $\ea, \eb, \ec \in \domA$ be such that $(\ea, \eb) \in \relL$ and 
$(\eb, \ec) \in \relL$.
Since $\relD$ and $\relE$ are in local agreement,
without loss of generality $\relD[\eb] \subseteq \relE[\eb]$.
Since $(\ea, \eb) \in \relL$, either $\ea \in \relD[\eb] \subseteq \relE[\eb]$
or $\ea \in \relE[\eb]$.
Similarly $\ec \in \relE[\eb]$.
Therefore $(\ea, \ec) \in \relE \subseteq \relL$.
    
Next suppose that $\relL$ is an equivalence relation, let $\eb \in \domA$ and
assume towards a contradiction
that $\relD[\eb] \not\subseteq \relE[\eb]$ and $\relE[\eb] \not\subseteq
\relD[\eb]$.
There is some $\ea \in \relD[\eb] \setminus \relE[\eb]$ and $\ec \in \relE[\eb]
\setminus \relD[\eb]$.
Then $(\ea, \eb) \in \relD \subseteq \relL$ and $(\eb, \ec) \in \relE \subseteq
\relL$, hence $(\ea, \ec) \in \relL$.
Without loss of generality $(\ea, \ec) \in \relE$.
Since $\ec \in \relE[\eb]$, we have $\ea \in \relE[\eb]$ --- a contradiction.
\end{proof}