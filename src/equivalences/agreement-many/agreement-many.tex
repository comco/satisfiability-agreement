% Agreement between many equivalence relations
Let $\sze$ be a positive natural number.
\begin{definition}
Let $\seq{\relEE1, \relEE2, \dots, \relEE\sze} \subseteq \domAA$ be
a sequence of equivalence relations on $\domA$.

The sequence is in \emph{refinement} if $\relEE1 \subseteq \relEE2 \subseteq
\dots \subseteq \relEE\sze$.

The sequence is in \emph{global agreement} if the equivalence relations form a
chain under inclusion, that is for all $\ii, \jj \in [1, \sze]$,
either $\relEE\ii \subseteq \relEE\jj$ or $\relEE\jj \subseteq \relEE\ii$.
Equivalently, there is a (not necessarily unique) permutation
$\permnu \in \Perms\sze$ such that
$\relEE{\permnu(1)} \subseteq \relEE{\permnu(2)} \subseteq \dots \subseteq
\relEE{\permnu(\sze)}$.

The sequence is in \emph{local agreement} if for every element $\ea \in \domA$,
the equivalence classes $\relEE1[\ea], \relEE2[\ea], \dots, \relEE\sze[\ea]$ form
a chain under inclusion.
Equivalently, no two equivalence classes $\relEE\ii[\ea]$ and $\relEE\jj[\eb]$
properly intersect.
\end{definition}

Let $\ES = \seq{\see1, \see2, \dots, \see\sze}$ be a predicate signature
consisting of $\sze$ binary predicate symbols.
Let $\StrA$ be an $\ES$-structure and suppose that the symbols $\see\ii$ are
interpreted as equivalence relations on $\domA$. Let 
$\relEE\ii = \at\StrA{\see\ii}$.
\begin{definition}
Define the $\vFoF2\ES$-sentence $\gls{frefine-many}$ by:
\[
  \frefine{\see1,\see2,\dots,\see\sze} =
  \forall\xx\forall\yy \bigwedge_{1 \leq \ii < \sze}
  \see\ii(\xx,\yy) \limp \see{\ii+1}(\xx,\yy).
\]
\end{definition}
Then $\seq{\relEE1,\relEE2,\dots,\relEE\sze}$ is in refinement iff
$\StrA \vDash \frefine{\see1,\see2,\dots,\see\sze}$.

\begin{definition}
Define the $\vFoF2\ES$-sentence $\gls{fglobal-many}$ by:
\[
  \fglobal{\see1,\see2,\dots,\see\sze} =
  \bigvee_{\permnu \in \Perms\sze}
  \frefine{\see{\permnu(1)},\see{\permnu(2)},\dots,\see{\permnu(\sze)}}.
\]
\end{definition}
Then $\seq{\relEE1,\relEE2,\dots,\relEE\sze}$ is in global agreement iff
$\StrA \vDash \fglobal{\see1,\see2,\dots,\see\sze}$.
Note that the length of the formula $\fglobal{\see1,\see2,\dots,\see\sze}$ grows
exponentially as $\sze$ grows.

\begin{definition}
Define the $\vFoF2\ES$-sentence $\gls{flocal-many}$ by:
\[
  \flocal{\see1,\see2,\dots,\see\sze} = \forall\xx
  \bigvee_{\permnu \in \Perms\sze}
  \forall\yy \bigwedge_{1 \leq \ii < \sze}
  \see{\permnu(\ii)}(\xx,\yy) \limp \see{\permnu(\ii+1)}(\xx,\yy).
\]
\end{definition}
Then $\seq{\relEE1,\relEE2,\dots,\relEE\sze}$ is in local agreement iff
$\StrA \vDash \flocal{\see1,\see2,\dots,\see\sze}$.
Note that the length of the formula $\flocal{\see1,\see2,\dots,\see\sze}$ grows
exponentially as $\sze$ grows.

Let $\seqE = \seq{\relEE1,\relEE2,\dots,\relEE\sze} \subseteq \domAA$
be a sequence of equivalence relations on $\domA$.
\begin{theorem}\label{thm:local}
The sequence $\seqE$ is in local agreement iff
the union $\cup \setS$ of any nonempty subset $\setS \subseteq \seqE$
is an equivalence relation on $\domA$.
\end{theorem}
\begin{proof}
First suppose that the equivalence relations $\relEE\ii$ are in local agreement.
We show that the union $\cup \setS$ of arbitrary nonempty subset 
$\setS = \setsm{\relEE{\ii(1)}, \relEE{\ii(2)}, \dots, \relEE{\ii(\sizes)}}$ is
an equivalence relation by induction on $\sizes$, the cardinality of $\setS$. If
$\sizes = 1$, this statement is trivial.
Suppose $\sizes > 1$.
By the induction hypothesis, $\relD = \cup \setsm{\relEE{\ii(1)},
\relEE{\ii(2)}, \dots, \relEE{\ii(\sizes-1)}}$ is an equivalence relation on
$\domA$.
We claim that $\relD$ and $\relEE{\ii(\sizes)}$ are in local agreement.
Indeed, let $\ea \in \domA$ be arbitrary and consider
$\relD[\ea] = \relEE{\ii(1)}[\ea] \cup \relEE{\ii(2)}[\ea] \cup \dots \cup
\relEE{\ii(\sizes - 1)}[\ea]$ and $\relEE{\ii(\sizes)}[\ea]$.
Since all equivalences $\relEE\kk$
are in local agreement, either
$\relEE{\ii(\sizes)}[\ea] \subseteq \relEE{\ii(\jj)}[\ea]$ 
for some $\jj \in [1, \sizes-1]$,
or $\relEE{\ii(\jj)}[\ea] \subseteq \relEE{\ii(\sizes)}[\ea]$
for all $\jj \in [1, \sizes-1]$.
In the first case $\relEE{\ii(\sizes)}[\ea] \subseteq \relD[\ea]$;
in the second case $\relD[\ea] \subseteq \relEE{\ii(\sizes)}[\ea]$.
Thus $\relD$ and $\relEE{\ii(\sizes)}$ are in local agreement.
By \cref{lem:local-2}, $\cup \setS = \relD \cup \relEE{\ii(\sizes)}$ is an
equivalence relation on $\domA$.

Next suppose that the equivalences are not in local agreement.
There is an element $\ea \in \domA$ such that
$\setbd{\relEE\ii[\ea]}{\ii \in [1, \sze]}$ is not a chain. 
There are $\ii, \jj \in [1, \sze]$ such that
$\relEE\ii[\ea] \not\subseteq \relEE\jj[\ea]$ and
$\relEE\jj[\ea] \not\subseteq \relEE\ii[\ea]$.
Thus $\relEE\ii$ and $\relEE\jj$ are not in local agreement.
By \cref{lem:local-2}, the union $\relEE\ii \cup \relEE\jj$ is not an
equivalence relation on $\domA$.
\end{proof}

Suppose that the sequence
$\seqE = \seq{\relEE1, \relEE2, \dots, \relEE\sze}$ is in local agreement.
\begin{definition}
An \emph{index set} is an element $\istI \in \nepset[1, \sze]$.
Define $\istap\seqE\cdot : \nepset[1, \sze] \to \nepset\seqE$ by:
\[
  \istap\seqE\istI = \setbd{\relEE\ii}{\ii \in \istI}.
\]

The \emph{level sequence} $\seqL = \seq{\relLL1,\relLL2,\dots,\relLL\sze}
\subseteq \domAA$ of the sequence $\seqE$ is defined as follows.
For $\kk \in [1, \sze]$:
\[
  \relLL\kk = \cap \setbd{\cup \istap\seqE\istI}{\istI \in \cpset\kk[1, \sze]}.
\]
\end{definition}
\begin{remark}
All $\relLL\kk$ are equivalence relations on $\domA$.
\end{remark}
\begin{proof}
Let $\istI \in \cpset\kk[1,\sze]$ be a $\kk$-index set, where $\kk \geq 1$.
By \cref{thm:local}, $\cup \istap\seqE\istI$ is an equivalence relation on
$\domA$.
Since intersection of equivalence relations on $\domA$ is again an equivalence 
relation on $\domA$, the level
$\relLL\kk = \cap \setbd{\cup \istap\seqE\istI}{\istI \in \cpset\kk[1,\sze]}$
is an equivalence relation on $\domA$.
\end{proof}
\begin{remark}\label{rem:local-lvl-refine}
The level sequence $\seqL$ is a sequence of equivalence relations on $\domA$ in
refinement.
\end{remark}
\begin{proof}
Let $\ii < \jj \in [1, \sze]$.
Let $\istJ \in \cpset\jj[1, \sze]$ be any $\jj$-index set.
We claim that $\relLL\ii \subseteq \cup \istap\seqE\istJ$.
Indeed, choose some $\ii$-index set $\istI \subset \istJ$.
By the definition of $\relLL\ii$ we have
$\relLL\ii \subseteq \cup \istap\seqE\istI \subseteq \cup \istap\seqE\istJ$.
Hence
$\relLL\ii \subseteq \cap \setbd{\cup \seqE[\istJ]}{\istJ \in \cpset\jj[1,\sze]}
= \relLL\jj$.
\end{proof}
Let $\ea \in \domA$. Since the sequence $\seqE$ is in local agreement, there is
a permutation $\permnu \in \Perms\sze$ such that:
\begin{equation}\label{eq:local-perm}
  \relEE{\permnu(1)}[\ea] \subseteq
  \relEE{\permnu(2)}[\ea] \subseteq
  \dots \subseteq
  \relEE{\permnu(\sze)}[\ea].
\end{equation}
\begin{lemma}\label{lem:local-lvl-perm}
If $\permnu \in \Perms\sze$ is a permutation satisfying \cref{eq:local-perm},
then $\relLL{\inv\permnu(\ii)}[\ea] = \relEE\ii[\ea]$ for all $\ii \in [1,\sze]$.
\end{lemma}
\begin{proof}
Let $\kk = \inv\permnu(\ii)$, so $\permnu(\kk) = \ii$.
We claim that $\relLL\kk[\ea] = \relEE\ii[\ea]$.
First, consider the $\kk$-index set 
$\istK = \set{\permnu(1), \permnu(2), \dots, \permnu(\kk)}$.
By the definition of $\relLL\kk$, followed by \cref{eq:local-perm},
we have $\relLL\kk[\ea] \subseteq \cup \istap\seqE\istK[\ea] = 
\relEE{\permnu(\kk)}[\ea] = \relEE\ii[\ea]$.
Next, let $\istK \subseteq \cpset\kk[1,\sze]$ be any $\kk$-index set. By the
pigeonhole principle, there is some $\kkp \geq \kk$ such that $\kkp \in \istK$.
By \cref{eq:local-perm} we have:
\[
  \relEE\ii[\ea] = \relEE{\permnu(\kk)}[\ea] \subseteq 
  \relEE{\permnu(\kkp)}[\ea] \subseteq \cup \istap\seqE\istK[\ea].
\]
Hence
$\relEE\ii[\ea] \subseteq \cap
\setbd{\cup \istap\seqE\istK[\ea]}{\istK \in \cpset\kk[1, \sze]} =
\relLL\kk[\ea]$.
\end{proof}