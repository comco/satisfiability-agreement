% Type realizibility og the two-variable first-order logic with equivalences in
% refinement
In this section we consider the logic $\Lvp\Fo2\nopow\Eea\sze\agrefine$
featureing $\sze \geq 2$ equivalence symbols $\see1,\see2,\dots,\see\sze$ in
refinement. Abbreviate the coarsest equivalence symbol $\se = \see\sze$.
Recall from the previous section that every $\vFo2$-sentence $\fphi$ can be
reduced in polynomial time to an equisatisfiable sentence in the form:
\begin{equation*}
  \forall\xx\forall\yy (\falp(\xx,\yy) \lor \xx = \yy) \land
  \bigwedge_{1 \leq \ii \leq \nm} \forall\xx\exists\yy
  (\smm\ii(\xx,\yy) \land \xx \neq \yy),
\end{equation*}
that may use additional unary predicate symbols and the new message symbols
$\smm\ii$, where the $\forall\forall$-part $\falp$ is quantifier-free.
\begin{definition}
A \emph{classified signature} for $\Lvp\Fo2\nopow\Eea\sze\agrefine$ is a
predicate signature $\SigS$ together with a sequence $\sms =
\smm1\smm2\dots\smm\nm$ of distinct binary predicate symbols \emph{distinct
from the equivalence symbols} from $\SigS$ having the intended interpretation
\begin{equation*}
  \bigwedge_{1 \leq \ii \leq \nm} \forall\xx\exists\yy 
  (\smm\ii(\xx,\yy) \land \xx \neq \yy).
\end{equation*}
\end{definition}
Again, note that a classified signature \emph{automatically includes} the
$\forall\exists$-part of formulas and $\seq{\SigS,\sms}$-structures
\emph{automatically satisfy} the $\forall\exists$-part.

The (finite) classified satisfiability problem for
$\Lvp\Fo2\nopow\Eea\sze\agrefine$ is defined as in~\Cref{def:clsig-twovar}.
Again we have the reduction:
\[
  \FinASat{\vFo2} \red\cP \FinAClSat{\vFo2}.
\]
Let $\ClSig\SigS\sms$ be a classified signature for
$\Lvp\Fo2\nopow\Eea\sze\agrefine$.
A type instance $\Tpi\TpiP\TpiT$ over $\ClSig\SigS\sms$ is defined as
in~\Cref{def:tpinst-twovar}; also the notions of a model for a type instance,
full realization, characteristic type instance of a model and the (finite)
(full) type realizibility problem for $\Lvp\Fo2\nopow\Eea\sze\agrefine$ are
defined. Again, the reductions from~\Cref{rem:red-real-to-full-real}
and~\Cref{rem:red-sat-to-real} hold:
\begin{align*}
  \FinAReal{\Lvp\Fo2\nopow\Eea\sze\agrefine} &\red\cNP 
  \FinAFullReal{\Lvp\Fo2\nopow\Eea\sze\agrefine} \\
  \FinAClSat{\Lvp\Fo2\nopow\Eea\sze\agrefine} &\red\cExpTime
  \FinAReal{\Lvp\Fo2\nopow\Eea\sze\agrefine}.
\end{align*}

We proceed to define new terms, specific to the case of many equivalence
symbols.
The terminology is loosely based on~\cite{MALQ:MALQ201400102}.

\begin{definition}
A $2$-type $\tpTt \in \TpT\SigS$ is a \emph{galactic type} if $\se(\xx,\yy) \in
\tau$.
Otherwise, if $(\lnot\se(\xx,\yy)) \in \tpTt$, the type is a \emph{cosmic type}.
The set of galactic $2$-types is $\TpTg\SigS$.
The set of cosmic $2$-types is $\TpTc\SigS$.
\end{definition}
Informally, we think of the $\se$-classes in a structure as galaxies, the
whole structure as the cosmos, of the galactic $2$-types as characterizing the
interactions in the internals of the $\se$-classes of a structure, while cosmic $2$-types characterize the interactions between elements in different
$\se$-classes.

Let $\Tpi\TpiP\TpiT$ be a type instance over $\ClSig\SigS\sms$.
\begin{definition}
Two $1$-types $\tpIp, \tpIpp \in \TpiP$ are \emph{galactically connectable}
(written $\tpIp \gconn \tpIpp$), if $\tpIp = \xtp\tpTt$ and $\tpIpp = \ytp\tpTt$
for some galactic $\tpTt \in \TpiT$.
The types are \emph{cosmically connectable} (written $\tpIp \cconn \tpIpp$), if
$\tpIp = \xtp\tpTt$ and $\tpIpp = \ytp\tpTt$ for some cosmic $\tpTt \in \TpiT$.
\end{definition}

Note that the types may be both galactically and cosmically connectible and the
(unqualified) definition of connectible types from~\Cref{def:connectable} is
equivalent to:
\[\miff{\tpIp \conn \tpIpp}{\tpIp \gconn \tpIpp \text{ or } \tpIp \cconn
\tpIpp}.\]



\begin{definition}
Let $\StrA$ be a $\ClSig\SigS\sms$-structure and let $\relE = \at\StrA\se$.
The \emph{cosmic spectrum} $\cspIX\StrA\eclX$ of an $\relE$-class
$\eclX \in \Ecl\relE$ is the set of cosmic types realized at $\eclX$:
\[
  \cspIX\StrA\eclX = \setbd{\tpIab\StrA\ea\eb \in \TpTc\SigS}{\ea\in\eclX, \eb \in
  \domA\sub\set{\ea}} = \setbd{\tpIab\StrA\ea\eb}{\ea \in \eclX, \eb \in \domA
  \sub \eclX}.
\]
\end{definition}