% Type realizibility og the two-variable first-order logic with equivalences in
% refinement
In this section we consider the logic $\Lvp\Fo2\nopow\Eea\sze\agrefine$
featuring $\sze \geq 2$ equivalence symbols $\see1,\see2,\dots,\see\sze$ in
refinement. Abbreviate the coarsest equivalence symbol $\se = \see\sze$.
Recall from the previous section that every $\vFo2$-sentence $\fphi$ can be
reduced in polynomial time to an equisatisfiable sentence in the form:
\begin{equation*}
  \forall\xx\forall\yy (\falp(\xx,\yy) \lor \xx = \yy) \land
  \bigwedge_{1 \leq \ii \leq \nm} \forall\xx\exists\yy
  (\smm\ii(\xx,\yy) \land \xx \neq \yy),
\end{equation*}
that may use additional unary predicate symbols and the new message symbols
$\smm\ii$, where the $\forall\forall$-part $\falp$ is quantifier-free.
\begin{definition}
A \emph{classified signature} for $\Lvp\Fo2\nopow\Eea\sze\agrefine$ is a
predicate signature $\SigS$ together with a sequence $\sms =
\smm1\smm2\dots\smm\nm$ of distinct binary predicate symbols \emph{distinct
from the equivalence symbols} from $\SigS$ having the intended interpretation
\begin{equation*}
  \bigwedge_{1 \leq \ii \leq \nm} \forall\xx\exists\yy 
  (\smm\ii(\xx,\yy) \land \xx \neq \yy).
\end{equation*}
\end{definition}
Again, note that a classified signature \emph{automatically includes} the
$\forall\exists$-part of formulas and $\seq{\SigS,\sms}$-structures
\emph{automatically satisfy} the $\forall\exists$-part.

The (finite) classified satisfiability problem for
$\Lvp\Fo2\nopow\Eea\sze\agrefine$ is defined as in~\Cref{def:clsig-twovar}.
Again we have the reduction:
\[
  \FinASat{\vFo2} \red\cP \FinAClSat{\vFo2}.
\]
Let $\ClSig\SigS\sms$ be a classified signature for
$\Lvp\Fo2\nopow\Eea\sze\agrefine$.
A type instance $\Tpi\TpiP\TpiT$ over $\ClSig\SigS\sms$ is defined as
in~\Cref{def:tpinst-twovar}; also the notions of a model for a type instance,
full realization, characteristic type instance of a model and the (finite)
(full) type realizibility problem for $\Lvp\Fo2\nopow\Eea\sze\agrefine$ are
defined. Again, the reductions from~\Cref{rem:red-real-to-full-real}
and~\Cref{rem:red-sat-to-real} hold:
\begin{align*}
  \FinAReal{\Lvp\Fo2\nopow\Eea\sze\agrefine} &\red\cNP 
  \FinAFullReal{\Lvp\Fo2\nopow\Eea\sze\agrefine} \\
  \FinAClSat{\Lvp\Fo2\nopow\Eea\sze\agrefine} &\red\cExpTime
  \FinAReal{\Lvp\Fo2\nopow\Eea\sze\agrefine}.
\end{align*}

We proceed to define new terms, specific to the case of many equivalence
symbols.
The terminology is loosely based on~\cite{MALQ:MALQ201400102}.

\begin{definition}
A $2$-type $\tpTt \in \TpT\SigS$ is a \emph{galactic type} if $\se(\xx,\yy) \in
\tau$.
Otherwise, if $(\lnot\se(\xx,\yy)) \in \tpTt$, the type is a \emph{cosmic type}.
The set of galactic $2$-types is $\TpTg\SigS$.
The set of cosmic $2$-types is $\TpTc\SigS$.
\end{definition}
Informally, we think of the $\se$-classes in a structure as galaxies; of the
whole structure as the cosmos; of the galactic $2$-types as characterizing the
interactions in the internals of the galaxies, while cosmic $2$-types
characterize the interactions between different galaxies.

\begin{definition}
Let $\Tpi\TpiP\TpiT$ be a type instance over $\ClSig\SigS\sms$.
Two $1$-types $\tpIp, \tpIpp \in \TpiP$ are \emph{galactically connectable}
(written $\tpIp \gconn \tpIpp$), if $\tpIp = \xtp\tpTt$ and $\tpIpp = \ytp\tpTt$
for some galactic $\tpTt \in \TpiT$.
The types are \emph{cosmically connectable} (written $\tpIp \cconn \tpIpp$), if
$\tpIp = \xtp\tpTt$ and $\tpIpp = \ytp\tpTt$ for some cosmic $\tpTt \in \TpiT$.
\end{definition}

Note that the types may be both galactically and cosmically connectible and the
(unqualified) definition of connectible types from~\Cref{def:connectable} is
equivalent to:
\[\miff{\tpIp \conn \tpIpp}{\tpIp \gconn \tpIpp \text{ or } \tpIp \cconn
\tpIpp}.\]

Recall that a $1$-type $\tpIk$ is a \emph{king type} if $\tpIk \not\conn \tpIk$.
A $1$-type $\tpIn$ is a \emph{noble type} if $\tpIn \not\cconn \tpIn$. Note that
every king type is a noble type. A $1$-type $\tpIp \in \TpiP$ is a peasant type
if it is not a noble type.
Note that this is a different definition of a peasant type from the definition in the
no builtin equivalence symbols case.
Let $\TKg = \TKgi\TpiP\TpiT$ be the set of king types, $\TNo = \TNoi\TpiP\TpiT$
be the set of noble types and $\TPs = \TPsi\TpiP\TpiT$ be the set of peasant
types. The following is a generalization of~\Cref{rem:twovar-king-once}
including noble types:
\begin{remark}\label{rem:twovar-noble-once}
If $\StrA$ is a full model for $\Tpi\TpiP\TpiT$, then every king type
$\tpIk \in \TKgi\TpiP\TpiT$ is realized once in $\StrA$ and every noble type
$\tpIn \in \TNoi\TpiP\TpiT$ is realized (possibly many times) in only one galaxy
of $\StrA$.
\end{remark}

Our strategy to resolve the type realizability problem would be to encode the
galactic structure of enough galaxies into instances of the full type
realizability problem for the logic featuring one less equivalence symbol.

The following definition characterizes the local structure of a single galaxy in
a full model.
\begin{definition}
A \emph{cosmic spectrum} $\csps \subseteq (\TpiT \cap \TpTc\SigS)$ for the
type instance $\Tpi\TpiP\TpiT$ is a nonempty set of cosmic types satisfying
the following conditions:
\begin{itemize}
  \item[\cspcondIp]\label{cond:csp-Ip}
  Call the set of $1$-types $(\xtp\restriction\stps)$ the ``inside'' of $\csps$.
  This is not a condition and Condition~\refstpcond1 for star-types has no
  analogue in this setting.
  \item[\cspcondIIp]\label{cond:csp-IIp}
  If $\tpIn \in (\xtp\restriction\csps) \cap \TNo$ is a noble type
  ``inside'' $\stps$, then no $\tpTt \in \stps$ has $\ytp\tpTt = \tpIn$.
  This is the analogue of Condition~\refstpcond2 for star-types.
  \item[\cspcondIIIn]\label{cond:csp-IIIn}
  If $\tpIn \in \TNo \sub (\xtp\restriction\stps)$ is any noble type not
  occurring ``inside'' $\csps'$, then some (possibly many) $\tpTt \in \stps$ has
  $\ytp\tpTt = \tpIn$.
  This is an analogue of Condition~\refstpcond3 for star-types.
  \item[\cspcondIIIk]\label{cond:csp-IIIk}
  If $\tpIk \in \TKg \sub (\xtp\restriction\stps)$ is any king type not
  occurring ``inside'' $\csps$, then only one $\tpTt \in \stps$ has $\ytp\tpTt =
  \tpIk$.
  Note that the existence already follows from~\refcspcondIIIn.
  This is an analogue of Condition~\refstpcond3 for star-types.
  \item[\cspcondIVp]\label{cond:csp-IVp}
  This is not a condition. The notion of local consistency
  will be the analogue of Condition~\refstpcond4 for star-types.
\end{itemize}
The cosmic spectrum is \emph{noble} if there is a noble $1$-type $\tpIn \in
(\xtp\restriction\stps)$.
\end{definition}
\begin{definition}
Let $\StrA$ be a $\ClSig\SigS\sms$-structure and let $\relE = \at\StrA\se$.
The \emph{cosmic spectrum} $\cspIX\StrA\eclX$ of an $\relE$-class
$\eclX \in \Ecl\relE$ is the set of cosmic types realized at $\eclX$:
\[
  \cspIX\StrA\eclX = \setbd{\tpIab\StrA\ea\eb \in \TpTc\SigS}{\ea\in\eclX, \eb 
  \in \domA\sub\set{\ea}} = \setbd{\tpIab\StrA\ea\eb}{\ea \in \eclX, \eb \in
  \domA \sub \eclX}.
\]
Note that the cosmic spectrum is empty iff $\relE = \domA\cprod\domA$.
Without loss of generality, we may assume that $\relE$ is not the full relation,
because the full relation is trivially definable in $\vFo2$. So without loss of
generality, we will consider models with nonempty cosmic spectrums of their
$\se$-classes.
\end{definition}
\begin{remark}
Let $\Tpi\TpiP\TpiT$ be a type instance for $\ClSig\SigS\sms$, $\StrA$ be a full
$\Tpi\TpiP\TpiT$-model such that $\relE = \at\StrA\se$ is not full on $\domA$
and let $\eclX \in \Ecl\relE$ be any galaxy.
Then $\csps = \cspIX\StrA\eclX$ is a cosmic spectrum for $\Tpi\TpiP\TpiT$.
\end{remark}
\begin{proof}
That $\csps$ is nonempty follows from the assumption that $\relE$ is not full on
$\domA$. We verify the conditions for a cosmic spectrum:
\begin{itemize}
  \item[\refcspcondIIp] If $\tpIn \in (\xtp\restriction\stps)$ is a noble type, then all
  elements realizing $\tpIn$ in $\StrA$ are in $\eclX$, hence there is no cosmic
  $\tpTt \in \stps$ having $\ytp\tpTt = \tpIn$.
  \item If $\tpIn \not\in (\xtp\restriction\stps)$ is a noble type,
  since $\StrA$ is a full model, there is some element $\eb \in \domA \sub
  \eclX$ realizing $\tpIn$. Then $\tpTt = \tpIab\StrA\ea\eb$ is a $2$-type
  having $\tpTt \in \stps$ and $\ytp\tpTt = \tpIn$.
  \item If $\tpIk \not\in (\xtp\restriction\stps)$ is a king type, then there is
  a unique element $\eb \in \domA$ realizing $\tpIk$.
  Note that $\eb \not\in \eclX$ and the $2$-type
  $\tpTt = \tpIab\StrA\ea\eb$ is the unique $\tpTt \in \stps$ such that
  $\ytp\tpTt = \tpIk$.
\end{itemize}
\end{proof}

We like to think about a cosmic spectrum of a galaxy as the \emph{reason why}
that galaxy is possible. For this we need to define the notion of a
\emph{locally-consistent} cosmic spectrum. For this we extract a type instance
$(\TpiP_\stps,\TpiT_\stps)$ over the \emph{simpler logic}
$\Lvp\Fo2\nopow\Eea{(\sze-1)}\agrefine$ out of the cosmic spectrum $\stps$:
\begin{definition}
Let $\Tpi\TpiP\TpiT$ be a type instance over $\ClSig\SigS\sms$ and $\stps
\subseteq (\TpiT \cap \TpTc\SigS)$ be a cosmic spectrum for $\Tpi\TpiP\TpiT$.
Let $\si$ (intended to label the inside of the galaxy) and $\sbh$ (intended to
label some \emph{black hole} outside the galaxy) be two new unary predicate
symbols.
For a $1$-type $\tpIp$ or a $2$-type $\tpTt$, denote by $\noe\tpIp$ and $\noe\tpTt$ the reducts of $\tpIp$ and
$\tpTt$ to the language $\SigS - \set{\se}$.
That is, $\noe\tpIp \subset \tpIp$ and $\noe\tpTt \subset \tpTt$ consist of
those literals that do not feature the predicate symbol $\se$.
We define the \emph{spectral type instance} $(\TpiP_\stps,\TpiT_\stps)$
corresponding to the cosmic spectrum $\stps$ as a type instance over
the $\Lvp\Fo2\nopow\Eea{(\sze-1)}\agrefine$-signature 
$\ClSig{\SigS - \set{\se} + \set{\si,\sbh}}\sms$ as follows:
\begin{description}
  \item[(I)] For every $\tpIp \in (\xtp\restriction\stps)$,
  the $1$-type $\noe\tpIp \cup \set{\si(\xx),\lnot\sbh(\xx)}$ is added to $\TpiP_\stps$.
  \item[(O)] For every $\tpIp \in (\ytp\restriction\stps)$,
  the $1$-type $\noe\tpIp \cup \set{\lnot\si(\xx),\lnot\sbh(\xx)}$ is added to $\TpiP_\stps$.
  \item[(B)] Let $\beta$ be an arbitrary $1$-type extending
  $\set{\lnot\si(\xx),\sbh(\xx)}$. Call $\beta$ the \emph{black hole type}.
  The black hole type is added to $\TpiP_\stps$.
  \item[(II)] For every (galactic) $\tpTt \in (\TpiT \cap
  \TpTg\SigS)$, the $2$-type
  \[
    \noe\tpTt \cup \set{\si(\xx),\lnot\sbh(\xx),\si(\yy),\lnot\sbh(\yy)}
  \]
  is added to $\TpiT_\stps$.
  \item[(IO),(OI)] For every (cosmic) $\tpTt \in \stps$,
  the $2$-type
  \[
    \noe\tpTt \cup \set{\si(\xx),\lnot\sbh(\xx),\lnot\si(\yy),\lnot\sbh(\yy)}  
  \] and its inverse are added to $\TpiT_\stps$.
  \item[(OO)] For every unordered pair of (not necessarly
  distinct) $1$-types $\set{\tpIp_\stps, \tpIp_\stps'} \subseteq \TpiP_\stps$ that are
  ``out''-types:
  $\set{\lnot\si(\xx),\lnot\sbh(\xx)} \subseteq \tpIp_\stps$ and
  $\set{\lnot\si(\xx),\lnot\sbh(\xx)} \subseteq \tpIp_\stps'$, add an arbitrary 
  $2$-type $\tpTt$ connecting them (so $\xtp\tpTt = \tpIp_\stps$ and $\ytp\tpTt
  = \tpIp_\stps'$) and its inverse $\inv\tpTt$ to $\TpiT_\stps$.
  \item[(IB)] For every $1$-type $\tpIp_\stps \in
  \TpiP_\stps$ that is an ``in''-type, so
  $\set{\si(\xx),\lnot\sbh(\xx)} \subseteq \tpIp_\stps$, let $\tpTt$ be an
  arbitrary $2$-type connecting $\tpIp_\stps$ and the black hole type $\beta$
  that witnesses no messages for $\tpIp_\stps$ and everything for $\beta$:
  $(\lnot\sm(\xx,\yy)) \in \tpTt$ and $\sm(\yy,\xx) \in \tpTt$ for every $\sm
  \in \sms$, add $\tpTt$ and $\inv\tpTt$ to $\TpiT_\stps$.
  \item[(OB)] For every $1$-type $\tpIp_\stps \in
  \TpiP_\stps$ that is an ``out''-type, so
  $\set{\lnot\si(\xx),\lnot\sbh(\xx)} \subseteq \tpIp_\stps$,
  let $\tpTt$ be an arbitrary $2$-type connecting $\tpIp_\stps$ and the black
  hole type $\beta$ that witnesses every message for $\tpIp_\stps$ and for
  $\beta$: $\sm(\xx,\yy) \in \tpTt$ and $\sm(\yy,\xx) \in \tpTt$ for every $\sm
  \in \sms$, add $\tpTt$ and $\inv\tpTt$ to $\TpiT_\stps$.
\end{description}
\end{definition}
\begin{definition}
Let $\Tpi\TpiP\TpiT$ be a type instance over the
$\Lvp\Fo2\nopow\Eea\sze\agrefine$-classified signature $\ClSig\SigS\sms$.
The cosmic spectrum $\stps \subseteq (\TpiT \cap \TpTc\SigS)$ is
\emph{locally consistent} if its spectral type instance
$(\TpiP_\stps, \TpiT_\stps)$ is fully satisfiable (as a type instance for the
$\Lvp\Fo2\nopow\Eea{(\sze-1)}\agrefine$-classified signature
$\ClSig{\SigS - \set{\se} + \set{\si, \sbh}}\sms$.
\end{definition}
\begin{remark}\label{rem:csp-is-locally-consistent}
Let $\StrA$ be a full $\Tpi\TpiP\TpiT$-model, $\relE = \at\StrA\se$ is not the
full relation of $\domA$ and $\eclX \in \Ecl\relE$ be a galaxy. Then
$\cspIX\StrA\eclX$ is a locally consistent cosmic spectrum for $\Tpi\TpiP\TpiT$.
\end{remark}
\begin{proof}
Let $\SigSp = \SigS - \set{\se} + \set{\si,\sbh}$. We build a $\SigSp$-structure
$\StrAp$ based on $\StrA$ that fully realizes the spectral type instance
$(\TpiP_\stps, \TpiT_\stps)$. The domain of $\StrAp$ is $\domA' = \domA \cup
\set{\ec}$, where $\ec$ is a new \emph{black hole} element. The $1$-type of
every element in $\StrAp$ is defined as follows:
\begin{description}
  \item[(I)] If $\ea \in \eclX$ is inside, then $\tpIa\StrAp\ea =
  \noe{\tpIa\StrA\ea} \cup \set{\si(\xx),\lnot\sbh(\xx)}$,
  that is the elements from the galaxy are inside.
  \item[(O)] If $\ea \in \domA\sub\eclX$ is outside, then $\tpIa\StrAp\ea =
  \noe{\tpIa\StrA\ea} \cup \set{\lnot\si(\xx),\lnot\sbh(\xx)}$, that is the
  elements outside the galaxy are outside.
  \item[(B)] If $\ea = \ec$, then $\tpIa\StrAp\ea = \beta$, that is the black
  hole element has the black hole type.
\end{description}
The $2$-type between every pair of distinct elements $\ea \neq \eb \in \domA'$
is defined as follows:
\begin{description}
  \item[(II)] If $\ea,\eb \in \eclX$ are inside, then:
  \[
    \tpIab\StrAp\ea\eb = \noe{\tpIab\StrA\ea\eb} \cup
    \set{\si(\xx),\lnot\sbh(\xx),\si(\yy),\lnot\sbh(\yy)}.
  \]
  \item[(IO)] If $\ea \in \eclX$ is inside and $\eb \in \domA \sub \eclX$ is
  outside, then:
  \[
    \tpIab\StrAp\ea\eb = \noe{\tpIab\StrA\ea\eb} \cup
    \set{\si(\xx),\lnot\sbh(\xx),\lnot\si(\yy),\lnot\sbh(\yy)}.
  \]
  \item[(OI)] If $\ea \in \domA \sub \eclX$ is outside and $\eb \in \eclX$ is
  inside, then:
  \[
    \tpIab\StrAp\ea\eb = \noe{\tpIab\StrA\ea\eb} \cup
    \set{\lnot\si(\xx),\lnot\sbh(\xx),\si(\yy),\lnot\sbh(\yy)}.
  \]
  Note that this produces the $2$-type that is exactly the inverse of the
  previous point.
  \item[(OO)] If $\ea, \eb \in \domA \sub \eclX$ are outside, then let $\tpTt
  \in \TpiT_\stps$ be the $2$-type connecting $\tpIp_\stps = \tpIa\StrAp\ea$ and
  $\tpIp_\stps' = \tpIa\StrAp\eb$ and assign $\tpIab\StrAp\ea\eb = \tpTt$. Note
  that this assignment is symmetric, since the unique $2$-type connecting
  $\tpIp_\stps'$ and $\tpIp_\stps$ in $\TpiT_\stps$ is $\inv\tpTt$.
  \item[(IB)] If $\ea \in \eclX$ is inside and $\eb = \ec$ is the black hole,
  then let $\tpTt \in \TpiT_\stps$ be the unique $2$-type connecting $\tpIp_\stps =
  \tpIa\StrAp\ea$ and $\beta$ and assign $\tpIab\StrAp\ea\eb = \tpTt$.
  \item[(OB)] If $\ea \in \domA \sub \eclX$ is outside and $\eb = \ec$ is the
  black hole, then let $\tpTt \in \TpiT_\stps$ be the unique $2$-type connecting
  $\tpIp_\stps = \tpIa\StrAp\ea$ and $\beta$ and assign $\tpIab\StrAp\ea\eb =
  \tpTt$.
\end{description}
We have to verify that $\StrAp$ is indeed a $\ClSig\SigSp\sms$-structure, that
is that the star-type of every element of $\StrAp$ contains $\sm(\xx,\yy)$ for
every $\sm \in \sms$. 
\begin{description}
\item[(I)] For the in-elements $\ea \in \eclX \subset \domA'$, every
$\sm \in \sms$ that is included in $\stpIa\StrA\ea$ is also included in
$\stpIa\StrAp\ea$, since a $2$-type $\noe\tpTt$ contains $\sm(\xx,\yy)$ iff
$\tpTt$ contains $\sm(\xx,\yy)$\footnote{this is where we use the condition
that the message symbols of a classified signature are distinct from the
builtin equivalence symbols}.
\item[(O)] For the out-elements $\ea \in (\domA \sub \eclX) \subset \domA'$, we
just need to recall that in the (OB) case, we included a $2$-type $\tpTt$ that
witnesses every message symbol for both of its ends: $\sm(\xx,\yy) \in \tpTt$
and $\sm(\yy,\xx) \in \tpTt$. By construction, this is exactly the $2$-type
between $\ea$ and the black hole $\ec$.
\item[(B)] For the black hole $\ec \in \domA'$, we just have to recall that in
the (IB) and (OB) cases, all included $2$-types connecting to the black hole
type contained $\sm(\yy,\xx)$ for every $\sm \in \sms$.
\end{description}
Since $\StrA$ was a full $\Tpi\TpiP\TpiT$-model, it is clear that $\StrAp$ is a
full $(\TpiP_\stps,\TpiT_\stps)$-model.
\end{proof}

\begin{definition}
Let $\StrA$ be a full model for the type instance $\Tpi\TpiP\TpiT$ and let
$\relE = \at\StrA\se$.
Let $\Nu \subseteq \TpiP$ be the set of noble types (realized in $\StrA$).
For every noble type $\tpIn \in \Nu$, let $\eclX_\tpIn \in \Ecl\relE$ be the
(unique) $\relE$-class containing a (not necessarly unique) element realizing
$\tpIn$, that is the set $\setbd{\eclX_\tpIn}{\tpIn \in \Nu}$ is the set of
all $\relE$-classes containing a noble element. Note that the cardinality of
this set is at most $\card\TpiP$, since by definition no noble type is
cosmically connected to itself.
The model $\StrA$ is \emph{nobly distinguished} if every element in $\eclX_\tpIn$ realizes a noble type for every $\tpIn \in \Nu$. That is, a model is nobly distinguished if every class that contains a noble element cointains
only noble elements.
\end{definition}
We want to restrict our attention to the problem of nobly distinguished type
realizability. To do this we will use some new unary predicate symbols and will
``color'' the elements living along with nobles in a single $\se$-class to make
them noble as well.
For this we need to encode possible connections between noble types and
$1$-types realized in the same class, so we enrich the type instance to
accommodate such information.
\begin{definition}
Let $\Tpi\TpiP\TpiT$ be a type instance for $\ClSig\SigS\sms$. For every
$1$-type $\tpIp \in \TpiP$, let $\spp\tpIp$ be a new unary predicate symbol. Let
$\SigSp = \SigS + \seqbd{\spp\tpIp}{\tpIp \in \TpiP}$ be an enrichment of
$\SigS$ featuring the new predicate symbols.
For every $\tpIp \in \TpiP$, let 
\[
  \spe^\tpIp = \spe^\tpIp(\xx) = \set{\spp\tpIp(\xx)} \cup
  \setbd{\lnot\spp\tpIr(\xx)}{\tpIr \in \TpiP \sub \set{\tpIp}}.
\]
For every pair of (not necessarily distinct) $1$-types $\tpIp, \tpIpp \in
\TpiP$, let $\tpIp_\tpIp'$ be a $1$-type over $\SigSp$ defined as follows:
\[
  \tpIp_\tpIp' = \tpIp' \cup \spe^\tpIp(\xx).
\]
We refer to $\tpIp_\tpIp'$ as the \emph{$\tpIp$-copy of $\tpIp'$}. Let
$\TpiP_\TpiP = \setbd{\tpIp'_\tpIp}{\tpIp',\tpIp \in \TpiP}$ be the set of all
the copies. Note that this is a set of $1$-types over $\SigSp$. Let
$\TpiT_\TpiP$ be a set of $2$-types over $\SigSp$ defined as follows:
\[
  \TpiT_\TpiP = \setbd{\tpTt \cup \spe^\tpIp(\xx) \cup
  \spe^\tpIpp(\yy)}{\tpIp,\tpIpp \in \TpiP}.
\]
Then $(\TpiP_\TpiP,\TpiT_\TpiP)$ is a type instance for $\ClSig\SigSp\sms$. Note
that the size of $(\TpiP_\TpiP,\TpiT_\TpiP)$ is polynomially bounded by the size
of $\Tpi\TpiP\TpiT$.
\end{definition}
\begin{remark}
Let $\StrA$ be a full model for the type instance $\Tpi\TpiP\TpiT$ over
$\ClSig\SigS\sms$.
Then there exists a
$\ClSig\SigSp\sms$-type instance $(\TpiP',\TpiT')$ such that $\TpiP$ ``is
contained in'' $\TpiP'$, $\TpiP' \subseteq \TpiP_\TpiP$, $\TpiT' \subseteq
\TpiT_\TpiT$ and a $\SigSp$-enrichment $\StrAp$ of $\StrA$ such that $\StrAp$ is a nobly
distinguished full model of $(\TpiP',\TpiT')$.
The set of $1$-types $\TpiP$ ``is contained in'' $\TpiP'$ if for every $\tpIp'
\in \TpiP$ there is some $\tpIp \in \TpiP$ such $\tpIp_\tpIp' \in \TpiP'$.
\end{remark}
\begin{proof}TODO SKETCH:
Let $\relE = \at\StrA\se$.
Recall that for every noble $\tpIn \in \Nu \subseteq \TpiP$, $\eclX_\tpIn \in
\Ecl\relE$ is the unique $\relE$-class containing a (not necessarly unique)
element $\ec \in \eclX$ realizing $\tpIn$. For every $\relE$-class $\eclX$ that
contains some noble element, arbitrary choose one noble type $\tpIn$ realized by
it: $\eclX = \eclX_\tpIn$. For every other $\relE$-class $\eclY$ that contains
no noble element, arbitrary choose one $1$-type $\tpIp$ realized in it.
Define the enrichment $\StrAp$ as follows: for every $\ea \in \domA$, if $\ea
\in \eclX_\tpIn$ is an element of an $\relE$-class containing a noble element,
let $\tpIa\StrAp\ea = \tpIa\StrA\ea_\tpIn$. Otherwise $\ea \in \eclY$, where
$\eclY$ is an $\relE$-class containing no noble elements; let $\tpIp$ be the
chosen $1$-type realized in $\eclY$ and let $\tpIa\StrAp\ea =
\tpIa\StrA\ea_\tpIp$. The characteristic type instance $(\TpiP',\TpiT')$ for
$\StrAp$ satisfies the conditions of this remark.
\end{proof}
TODO: In nondeterministic polynomial time we may reduce the (full) (finite) type
satisfiability problem to the (full) (finite) type satisfiability problem over
the class of nobly distinguished structures. That is, the nobly distinguished
structures form a nondeterministic polynomial time reduction class for (full)
(finite) satisfiability.
\begin{definition}
A \emph{certificate} $\Cert$ for the type instance $\Tpi\TpiP\TpiT$ is a
nonempty set of locally consistent cosmic spectrums satisfying the following
conditions:
\begin{enumerate}
  \item\label{cond:cert-many-1} If $\tpIp \in \TpiP$, then some (possibly many)
  $\stps \in \Cert$ have $\tpIp \in (\xtp \restriction \stps)$.
  \item\label{cond:cert-many-2} If $\tpIn \in \TpiP$ is noble, then only one
  $\stps \in \Cert$ has $\tpIn \in (\xtp \restriction \stps)$. Note that the
  existence is already implied by Condition~\ref{cond:cert-many-1}.
  \item\label{cond:cert-many-3} If $\tpTt \in \cup\Cert$, then $\inv\tpTt \in
  \cup\Cert$.
  \item\label{cond:cert-many-4} (If $\stps\in\Cert$ is noble, then for every
  $\tpTt \in \stps$ there is another $\stps'\in\Cert, \stps' \neq \stps$ such
  that $\tpTt \in \stps'$.) TODO: we don't need this for nobly distinguished
  structures!
\end{enumerate}
\end{definition}
Again, a note that a certificate may have size that is exponential with respect
to the size of the underlying type instance. However, polynomial certificates
exist:

\begin{lemma}[Certificate extraction] Let $\StrA$ be a full model for the type
instance $\Tpi\TpiP\TpiT$ and suppose that $\relE = \at\StrA\se$ is not the full
relation on $\domA$. Recall that the cardinality of $\StrA$ is at least $2$.
Let $\TpiTp \subseteq (\TpiT \cap \TpTc\SigS)$ be the set of cosmic $2$-types
from the type instance that are realized in $\StrA$. For every cosmic $2$-type
$\tpTt \in \TpiTp$ consider two distinct elements $\ea_\tpTt \neq \eb_\tpTt \in
\domA$ realizing $\tpTt$: $\tpIab\StrA{\ea_\tpTt}{\eb_\tpTt} = \tpTt$.
Let
\[
  \Cert = \setbd{\cspIX\StrA{\relE[\ea_\tpTt]}}{\tpTt \in \TpiTp} \cup
  \setbd{\cspIX\StrA{\relE[\eb_\tpTt]}}{\tpTt \in \TpiTp}.
\]
Then $\Cert$ is a certificate for the type instance. Moreover, its size is
linearly bounded by $\card{\TpiT}$, hence also by the size of the type instance.
\end{lemma}
\begin{proof}
By~\Cref{rem:csp-is-locally-consistent}, all elements $\stps \in \Cert$ are
locally consistent cosmic spectrums. We check the conditions for a certificate:
\begin{enumerate}
  \item If $\tpIp \in \TpiP$, since $\StrA$ is a full model for
  $\Tpi\TpiP\TpiT$, there is some $\ea \in \domA$ such that $\tpIa\StrA\ea =
  \tpIp$. Since $\relE$ is not the full relation on $\domA$, there is some $\eb
  \in \domA \sub \relE[\ea]$ and so $\tpTt = \tpIab\StrA\ea\eb \in \TpiTp$, such
  that $\tpIp = \xtp\tpTt$ and $\tpTt \in \cspIX\StrA{\relE[\ea_\tpTt]}$.
  \item Let $\tpIn \in \TpiP$ be noble. Then all elements $\ea \in \domA$
  realizing $\tpIn$ are in the same $\relE$-class: for every $\tpTt \in (\TpiT
  \cup \TpTc\SigS)$ such that $\xtp\tpTt = \tpIn$, we have $\relE[\ea_\tpTt] =
  \eclX$.
  \item That $\tpTt \in \cup\Cert$ implies $\inv\tpTt \in \cup\Cert$ follows
  immediately from the definition of $\Cert$.
  \item Let $\stps \in \Cert$ be noble and let $\tpTt \in \stps$ be any cosmic
  $2$-type. Then $\stps = \cspIX\StrA\eclX$ for some unique $\eclX \in
  \Ecl\relE$ and $\tpTt = \tpIab\StrA{\ea_\tpTt}{\eb_\tpTt}$. If $\ea_\tpTt \in \eclX$, since
  $\tpTt$ is cosmic, we must have that $\eb_\tpTt \not\in \eclX$, so
  $\stps' = \cspIX\StrA{\relE[\eb_\tpTt]} \neq \stps$ is appropriate.
  \item TODO: If $\ea_\tpTt \not\in \eclX$, this breaks! Need to define
  certificate more precisely! For this we may define a weak form of
  differentiation! In nobly distinguished structures this must work!
\end{enumerate}
\end{proof}
\section{NOTES}
TODO: Certificate extraction: for every cosmic $2$-type $\tpTt$ realized in
$\StrA$, let $\ea_\tpTt, \eb_\tpTt$ be a pair of distinct elements (and in
distinct $\se$-classes, since $\tpTt$ is cosmic) realizing it. Take:
\[
  \Cert = \setbd{\cspIX\StrA{E[\ea_\tpTt]}, \cspIX\StrA{E[\eb_\tpTt]}}{\tpTt
  \dots}
\]