% Type realizibility og the two-variable first-order logic with equivalences in
% refinement
In this section we consider the logic $\Lvp\Fo2\nopow\Eea\sze\agrefine$
featuring $\sze \geq 2$ equivalence symbols $\see1,\see2,\dots,\see\sze$ in
refinement. Abbreviate the coarsest equivalence symbol $\se = \see\sze$.
Recall from the previous section that every $\vFo2$-sentence $\fphi$ can be
reduced in polynomial time to an equisatisfiable sentence in the form:
\begin{equation*}
  \forall\xx\forall\yy (\falp(\xx,\yy) \lor \xx = \yy) \land
  \bigwedge_{1 \leq \ii \leq \nm} \forall\xx\exists\yy
  (\smm\ii(\xx,\yy) \land \xx \neq \yy),
\end{equation*}
that may use additional unary predicate symbols and the new message symbols
$\smm\ii$, where the $\forall\forall$-part $\falp$ is quantifier-free.
\begin{definition}
A \emph{classified signature} for $\Lvp\Fo2\nopow\Eea\sze\agrefine$ is a
predicate signature $\SigS$ together with a sequence $\sms =
\smm1\smm2\dots\smm\nm$ of distinct binary predicate symbols \emph{distinct
from the equivalence symbols} from $\SigS$ having the intended interpretation
\begin{equation*}
  \bigwedge_{1 \leq \ii \leq \nm} \forall\xx\exists\yy 
  (\smm\ii(\xx,\yy) \land \xx \neq \yy).
\end{equation*}
\end{definition}
Again, note that a classified signature \emph{automatically includes} the
$\forall\exists$-part of formulas and $\seq{\SigS,\sms}$-structures
\emph{automatically satisfy} the $\forall\exists$-part.

The (finite) classified satisfiability problem for
$\Lvp\Fo2\nopow\Eea\sze\agrefine$ is defined as in~\Cref{def:clsig-twovar}.
Again we have the reduction:
\[
  \FinASat{\vFo2} \red\cP \FinAClSat{\vFo2}.
\]
Let $\ClSig\SigS\sms$ be a classified signature for
$\Lvp\Fo2\nopow\Eea\sze\agrefine$.
A type instance $\Tpi\TpiP\TpiT$ over $\ClSig\SigS\sms$ is defined as
in~\Cref{def:tpinst-twovar}; also the notions of a model for a type instance,
and the (finite) type realizability problem for
$\Lvp\Fo2\nopow\Eea\sze\agrefine$ are defined as before. 
Again, the reduction of~\Cref{rem:red-sat-to-real} holds:
\[
  \FinAClSat{\Lvp\Fo2\nopow\Eea\sze\agrefine} \red\cNExpTime
  \FinAReal{\Lvp\Fo2\nopow\Eea\sze\agrefine}.
\]

We proceed to define new terms, specific to the case of many equivalence
symbols.
The terminology is loosely based on~\cite{MALQ:MALQ201400102}.

\begin{definition}
A $2$-type $\tpTt \in \TpT\SigS$ is a \emph{galactic type} if $\se(\xx,\yy) \in
\tau$.
Otherwise, if $(\lnot\se(\xx,\yy)) \in \tpTt$, the type is a \emph{cosmic type}.
The set of galactic $2$-types is $\TpTg\SigS$.
The set of cosmic $2$-types is $\TpTc\SigS$.
\end{definition}
Informally, we think of the $\se$-classes in a structure as galaxies; of the
whole structure as the cosmos; of the galactic $2$-types as characterizing the
interactions in the internals of the galaxies, while cosmic $2$-types
characterize the interactions between different galaxies.

\begin{definition}
Let $\Tpi\TpiP\TpiT$ be a type instance over $\ClSig\SigS\sms$.
Two $1$-types $\tpIp, \tpIpp \in \TpiP$ are \emph{galactically connectable}
(written $\tpIp \gconn \tpIpp$), if $\tpIp = \xtp\tpTt$ and $\tpIpp = \ytp\tpTt$
for some galactic $\tpTt \in \TpiT$.
The types are \emph{cosmically connectable} (written $\tpIp \cconn \tpIpp$), if
$\tpIp = \xtp\tpTt$ and $\tpIpp = \ytp\tpTt$ for some cosmic $\tpTt \in \TpiT$.
\end{definition}

Note that the types may be both galactically and cosmically connectable and the
(unqualified) definition of connectable types from~\Cref{def:connectable} is
equivalent to:
\[\miff{\tpIp \conn \tpIpp}{\tpIp \gconn \tpIpp \text{ or } \tpIp \cconn
\tpIpp}.\]

Recall that a $1$-type $\tpIk$ is a \emph{king type} if $\tpIk \not\conn \tpIk$.
A $1$-type $\tpIn$ is a \emph{noble type} if $\tpIn \not\cconn \tpIn$. Note that
every king type is a noble type. A $1$-type $\tpIp \in \TpiP$ is a peasant type
if it is not a noble type.
Note that this is a different definition of a peasant type from the definition in the
no builtin equivalence symbols case.
Let $\TKg = \TKgi\TpiP\TpiT$ be the set of king types, $\TNo = \TNoi\TpiP\TpiT$
be the set of noble types and $\TPs = \TPsi\TpiP\TpiT$ be the set of peasant
types. The following is a generalization of~\Cref{rem:twovar-king-once}
including noble types:
\begin{remark}\label{rem:twovar-noble-once}
If $\StrA$ is a model for $\Tpi\TpiP\TpiT$, then every king type
$\tpIk \in \TKgi\TpiP\TpiT$ is realized once in $\StrA$ and every noble type
$\tpIn \in \TNoi\TpiP\TpiT$ is realized (possibly many times) in only one galaxy
of $\StrA$.
\end{remark}

Our strategy to resolve the type realizability problem would be to encode the
galactic structure of enough galaxies into instances of the type
realizability problem for the logic featuring one less equivalence symbol.

The following characterizes the local structure of a single galaxy in
a model.
\begin{definition}
A \emph{cosmic spectrum} $\csps \subseteq (\TpiT \cap \TpTc\SigS)$ for the
type instance $\Tpi\TpiP\TpiT$ is a nonempty set of cosmic types satisfying
the following conditions:
\begin{itemize}
  \item[\cspcondIp]\label{cond:csp-Ip}
  Call the set of $1$-types $(\xtp\restriction\stps)$ the inside of $\csps$.
  This is not a condition and Condition~\refstpcond1 for star-types has no
  analogue in this setting.
  \item[\cspcondIIp]\label{cond:csp-IIp}
  If $\tpIn \in (\xtp\restriction\csps) \cap \TNo$ is a noble type
  inside $\stps$, then no $\tpTt \in \stps$ has $\ytp\tpTt = \tpIn$.
  This is the analogue of Condition~\refstpcond2 for star-types.
  \item[\cspcondIIIp]\label{cond:csp-IIIp}
  If $\tpIn \in \TNo \sub (\xtp\restriction\stps)$ is any noble type outside
  $\csps$ and $\tpIp \in (\xtp\restriction\stps)$ is any $1$-type inside
  $\csps$, then some (possibly many) $\tpTt \in \stps$ have 
  $\xtp\tpTt = \tpIp$ and $\ytp\tpTt = \tpIn$.
  This is the analogue of Condition~\refstpcond3 for star-types.
  \item[\cspcondIVp]\label{cond:csp-IVp}
  This is not a condition. The notion of local consistency
  will be the analogue of Condition~\refstpcond4 for star-types.
\end{itemize}
The cosmic spectrum is \emph{noble} if there is a noble $1$-type $\tpIn \in
(\xtp\restriction\stps)$.
\end{definition}
\begin{definition}
Let $\StrA$ be a $\ClSig\SigS\sms$-structure and let $\relE = \at\StrA\se$.
The \emph{cosmic spectrum} $\cspIX\StrA\eclX$ of an $\relE$-class
$\eclX \in \Ecl\relE$ is the set of cosmic types realized at $\eclX$:
\[
  \cspIX\StrA\eclX = \setbd{\tpIab\StrA\ea\eb \in \TpTc\SigS}{\ea\in\eclX, \eb 
  \in \domA\sub\set{\ea}} = \setbd{\tpIab\StrA\ea\eb}{\ea \in \eclX, \eb \in
  \domA \sub \eclX}.
\]
Note that the cosmic spectrum is empty iff $\relE = \domA\cprod\domA$ is the
full relation on $\domA$.
We will concentrate on the case where $\relE$ is not the full relation,
since the full relation is trivially definable in $\vFo2$. This enables us to
consider models with nonempty cosmic spectrums of their galaxies without loss
of generality.
\end{definition}
\begin{remark}
Let $\Tpi\TpiP\TpiT$ be a type instance over $\ClSig\SigS\sms$, let $\StrA$ be a
model for $\Tpi\TpiP\TpiT$ such that $\relE = \at\StrA\se$ is not full on
$\domA$ and let $\eclX \in \Ecl\relE$ be any galaxy.
Then $\csps = \cspIX\StrA\eclX$ is a cosmic spectrum for $\Tpi\TpiP\TpiT$.
\end{remark}
\begin{proof}
That $\csps$ is nonempty follows from the assumption that $\relE$ is not full on
$\domA$. We verify the conditions for a cosmic spectrum:
\begin{itemize}
  \item[\refcspcondIIp]
   Let $\tpIn \in (\xtp\restriction\csps) \cap \TNo$ be a noble type inside
   $\csps$. Since $\StrA$ is a $\Tpi\TpiP\TpiT$-model, we have that $\csps
   \subseteq \TpiT$ and by noble type definition, no $\tpTt \in \TpiT$
   connects $\tpIn$ with itself.
   \item[\refcspcondIIIp]
   Let $\tpIn \in \TNo \sub (\xtp\restriction\csps)$ be a noble type outside
   $\csps$ and let $\tpIp \in (\xtp\restriction\csps)$ be a $1$-type inside
   $\csps$.
   Since $\StrA$ is a $\Tpi\TpiP\TpiT$-model, there is some
   $\ea \in \eclX$ having $\tpIa\StrA\ea = \tpIp$ and some
   $\eb \in \domA \sub \eclX$ having $\tpIa\StrA\eb = \tpIn$.
   Then $\tpTt = \tpIab\StrA\ea\eb \in \csps$.
\end{itemize}
\end{proof}

In order to simplify our construction, we are going to restrict our attention
to the class of nobly distinguished models.
\begin{definition}\label{def:nobly-distinguished-csp}
A cosmic spectrum $\csps$ for the type instance $\Tpi\TpiP\TpiT$ is \emph{nobly
distinguished} if $(\xtp\restriction\csps) \cap \TNoi\TpiP\TpiT \neq \eset$
implies $(\xtp\restriction\csps) \subseteq \TNoi\TpiP\TpiT$, that is if $\csps$
is noble, then every $1$-type inside $\csps$ is noble.
\end{definition}
\begin{definition}\label{def:nobly-distinguished-structure}
Let $\StrA$ be a $\ClSig\SigS\sms$-structure such that $\relE = \at\StrA\se$ is
not full on $\domA$ and let $\Tpi\TpiP\TpiT$ be the type instance of $\StrA$. The
structure is \emph{nobly distinguished} if the cosmic spectrum of every galaxy
is nobly distinguished, that is if $\cspIX\StrA\eclX$ is nobly distinguished
over $\Tpi\TpiP\TpiT$ for every $\eclX \in \Ecl\relE$.
\end{definition}
We use some new unary predicate symbols to color the peasants living along 
nobles to promote them to nobles.
\begin{definition}
Let $\Tpi\TpiP\TpiT$ be a type instance for $\ClSig\SigS\sms$.
For every $1$-type $\tpIp \in \TpiP$, let $\spp\tpIp$ be a new unary predicate
symbol. Let $\SigSp = \SigS + \seqbd{\spp\tpIp}{\tpIp \in \TpiP}$ be an
enrichment of $\SigS$ featuring these new symbols. For every $\tpIp \in \TpiP$,
define the following set of literals:
\[
  \sPe\tpIp(\xx) = \set{\spp\tpIp(\xx)} \cup
  \setbd{\lnot\spp\tpIpp(\xx)}{\tpIpp \in \TpiP \sub \set{\tpIp}}.
\]
Let $\bot \not\in \TpiP$ be a special element and define the special set
$\sPe\bot(\xx)$ of literals:
\[
  \sPe\bot(\xx) = 
  \setbd{\lnot\spp\tpIp(\xx)}{\tpIp \in \TpiP}.
\]
For every $\tpIp \in \TpiP$, $\tpIr \in \TpiP\cup\set{\bot}$,
let $\mr\tpIp\tpIr$ be the following $1$-type over $\SigSp$:
\[
  \mr\tpIp\tpIr = \tpIp \cup \sPe\tpIr(\xx).
\]
We refer to $\mr\tpIp\tpIr$ as the $\tpIr$-copy of $\tpIp$.
Define $\Tpi{\mr\TpiP\TpiP}{\mr\TpiT\TpiT}$ as follows:
\begin{align*}
  \mr\TpiP\TpiP &= 
  \setbd{\mr\tpIp\tpIr}{\tpIp\in\TpiP, \tpIr\in\TpiP\cup\set{\bot}} \\
  \mr\TpiT\TpiT &= \setbd{\tpTt \cup \sPe\tpIr(\xx) \cup
  \sPe\tpIrp(\xx)}{\tpIr,\tpIrp \in \TpiP\cup\set{\bot}}.
\end{align*}
Then $\Tpi{\mr\TpiP\TpiP}{\mr\TpiT\TpiT}$ is a type instance for
$\ClSig\SigSp\sms$.
Note that the size of $\Tpi{\mr\TpiP\TpiP}{\mr\TpiT\TpiT}$ is quadratic, hence
polynomially bounded, with respect to the size of the original type instance
$\Tpi\TpiP\TpiT$.
\end{definition}
\begin{definition}
Let $\Tpi\TpiP\TpiT$ be a type instance over $\ClSig\SigS\sms$ and let
$\SigSp = \SigS + \seqbd{\spp\tpIp}{\tpIp \in \TpiP}$.
Let $\TpiPb \subseteq \mr\TpiP\TpiP$ and $\TpiTb \subseteq \mr\TpiT\TpiT$ be
such that $\Tpi\TpiPb\TpiTb$ is a type instance over $\ClSig\SigSp\sms$.
Then $\Tpi\TpiPb\TpiTb$ is a \emph{promotion} of $\Tpi\TpiP\TpiT$ if
$\TpiPb$ contains a copy of every $\tpIp \in \TpiP$, 
that is for every $\tpIp \in \TpiP$ there is some $\tpIr \in
\TpiP\cup\set{\bot}$ such that $\mr\tpIp\tpIr \in \TpiPb$.
\end{definition}
\begin{lemma}[Noble distinguishability]\label{lem:noble-distinguishability}
Let $\Tpi\TpiP\TpiT$ be a type instance over $\ClSig\SigS\sms$ and let
$\SigSp = \SigS + \seqbd{\spp\tpIp}{\tpIp \in \TpiP}$.
Then $\Tpi\TpiP\TpiT$ has a (finite) model iff there is some promotion
$\Tpi\TpiPb\TpiTb$ of $\Tpi\TpiP\TpiT$ that has a (finite) nobly distinguished
model.
\end{lemma}
\begin{proof}
First, suppose that $\StrA$ is a model for $\Tpi\TpiP\TpiT$ and let $\relE
= \at\StrA\se$.
We define a promotion $\Tpi\TpiPb\TpiTb$ of $\Tpi\TpiP\TpiT$ and
a $\SigSp$-enrichment $\StrAp$ of $\StrA$ that realizes $\Tpi\TpiPb\TpiTb$.
Let $\Tpi\TpiP\TpiTp$ be the type instance of $\StrA$.
For every noble $\tpIn \in \TNoi\TpiP\TpiTp$, let $\eclX_\tpIn \in \Ecl\relE$ be
the unique galaxy realizing it. Note that there might be distinct noble types
realized in the same galaxy: 
$\eclX_\tpIn = \eclX_\tpInp$ for $\tpIn \neq \tpInp \in \TNoi\TpiP\TpiTp$.
Let $\eclX_\TNo = \setbd{\eclX_\tpIn}{\tpIn \in \TNoi\TpiP\TpiTp}$ be the set of
galaxies realizing some (possibly many) noble type. For every $\eclX \in
\eclX_\TNo$ choose an arbitrary noble type $\tpIn$ realized in it: $\eclX =
\eclX_\tpIn$.
Define the enrichment $\StrAp$ as follows: for every $\ea \in \domA$:
\begin{itemize}
  \item if $\ea \in \eclX_\tpIn$ is an element of some noble galaxy, then let
$\tpIa\StrAp\ea = \mr{\tpIa\StrA\ea}\tpIn$
  \item otherwise, let $\tpIa\StrAp\ea = \mr{\tpIa\StrA\ea}\bot$.
\end{itemize}
Let $\Tpi\TpiPb\TpiTb$ be the type instance of $\StrAp$. By construction,
$\Tpi\TpiPb\TpiTb$ is a promotion of $\Tpi\TpiP\TpiT$ and $\StrAp$ is a
model of $\Tpi\TpiPb\TpiTb$.
We need to check that $\StrAp$ is nobly distinguished.
We claim that every peasant type $\tpIpp \in \TPsi\TpiPb\TpiTb$ has the form
$\tpIpp = \mr\tpIp\bot$ for some $\tpIp\in\TpiP$.
Suppose not and let $\tpIpp\in \TPsi\TpiPb\TpiTb$ be some peasant type having
the form $\tpIpp = \mr\tpIp\tpIn$ for some $\tpIp\in\TpiP$ and
$\tpIn\in\TNoi\TpiP\TpiTp$. 
Since $\tpIpp$ is a peasant type, there is some cosmic $\tpTtp \in \TpiTb$ such
that $\xtp\tpTtp = \ytp\tpTtp$. Since $\Tpi\TpiPb\TpiTb$ is the type instance of
$\StrAp$, there are some $\ea \neq \eb \in \domA$ such that $\tpIab\StrAp\ea\eb = \tpTtp$ and
since $\tpTtp$ is cosmic, $\ea$ and $\eb$ must lie in distinct $\relE$-classes.
But $\tpIa\StrAp\ea = \tpIa\StrAp\eb = \tpIpp = \mr\tpIp\tpIn$ and by
construction we must have that $\ea, \eb \in \eclX_\tpIn$ --- a contradiction.
Therefore every peasant type $\tpIpp \in \TPsi\TpiPb\TpiTb$ has the form
$\tpIpp = \mr\tpIp\bot$ for some $\tpIp \in \TpiP$. So every noble type $\tpInp
\in \TNoi\TpiPb\TpiTb$ must have the form $\tpInp = \mr\tpIp\tpIn$ for some
$\tpIp \in \TpiP$ and $\tpIn \in \TNoi\TpiP\TpiTp$. Now by construction it is
immediate that $\StrAp$ is nobly distinguished.

Next, suppose that $\Tpi\TpiPb\TpiTb$ is a promotion of $\Tpi\TpiP\TpiT$ and
that $\StrAp$ is a model for $\Tpi\TpiPb\TpiTb$. Then the reduct of $\StrAp$
to a $\SigS$-structure is a model for $\Tpi\TpiP\TpiT$.
\end{proof}
Denote the type realizability problem for $\vFo2\Eea\sze\agrefine$ restricted to
the class of nobly distinguished structures by $\NDReal{\vFo2\Eea\sze\agrefine}$ and the finite
version of that problem by $\FinNDReal{\vFo2\Eea\sze\agrefine}$.
The noble distinguishability lemma shows that the class of nobly distinguished
models is a nondeterministic polynomial time reduction class:
\[
  \FinAReal{\vFo2\Eea\sze\agrefine} \red{\cNP} \FinANDReal{\vFo2\Eea\sze\agrefine}.
\]

We like to think about a cosmic spectrum of a galaxy as the \emph{reason why}
that galaxy is possible. For this we need to define the notion of a
\emph{locally consistent} cosmic spectrum. For this we extract a type instance
$\Tpi{\TpiPs\csps}{\TpiTs\csps}$ over the \emph{simpler logic}
$\Lvp\Fo2\nopow\Eea{(\sze-1)}\agrefine$ out of the cosmic spectrum $\csps$:
\begin{definition}
Let $\Tpi\TpiP\TpiT$ be a type instance over the
$\Lvp\Fo2\nopow\Eea\sze\agrefine$-classified signature $\ClSig\SigS\sms$.
Let $\csps \subseteq (\TpiT \cap \TpTc\SigS)$ be a cosmic spectrum over
$\Tpi\TpiP\TpiT$.
Let $\si$ (intended to label the inside of the galaxy) and $\sbh$ (intended to
label some \emph{black hole} outside the galaxy) be two new unary predicate
symbols. Define the following sets of literals: 
\begin{align*}
  \tIN(\xx) &= \set{\si(\xx),\lnot\sbh(\xx)} \\
  \tOUT(\xx) &= \set{\lnot\si(\xx), \lnot\sbh(\xx)} \\
  \tBH(\xx) &= \set{\lnot\si(\xx), \sbh(\xx)}.
\end{align*}
For a $1$-type $\tpIp\in\TpiP$ or a $2$-type $\tpTt\in\TpiT$, denote by
$\noe\tpIp$ and $\noe\tpTt$ the reducts of $\tpIp$ and $\tpTt$ to the language $\SigS - \set{\se}$.
That is, $\noe\tpIp \subset \tpIp$ and $\noe\tpTt \subset \tpTt$ consist of
those literals that do not feature the predicate symbol $\se$.

Define the \emph{spectral type instance} $\Tpi{\TpiPs\csps}{\TpiTs\csps}$ of the
cosmic spectrum $\stps$ as a type instance over the $\Lvp\Fo2\nopow\Eea{(\sze-1)}\agrefine$-signature 
$\ClSig{\SigS - \seq{\se} + \seq{\si,\sbh}}\sms$\footnote{Note that here we use
the condition that no builtin equivalence symbol $\se$ occurs as a message
symbol.} as follows:
\begin{itemize}
  \item[\sticondI]\label{sti-I}
  For every $\tpIp \in (\xtp\restriction\stps)$ inside $\csps$, add the
  $1$-type $(\noe\tpIp \cup \tIN(\xx))$ to $\TpiPs\csps$.
  \item[\sticondO]\label{sti-O}
  For every $\tpIp \in (\ytp\restriction\stps)$ outside $\csps$, add the
  $1$-type $(\noe\tpIp \cup \tOUT(\xx))$ to $\TpiPs\csps$. Note that some
  $1$-types might occur both inside and outside of $\csps$.
  \item[\sticondB]\label{sti-B}
  Let $\tpIb$ be an arbitrary $1$-type extending $\tBH(\xx)$.
  Call $\tpIb$ the \emph{black hole type} and add it to $\TpiPs\csps$.
  \item[\sticondII]\label{sti-II}
  For every galactic $\tpTt \in (\TpiT \cap \TpTg\SigS)$, add the $2$-type
  $(\noe\tpTt \cup \tIN(\xx) \cup \tIN(\yy))$ to $\TpiTs\csps$.
  \item[\sticondIO]\label{sti-IO}
  For every (cosmic) $\tpTt \in \stps$, add the $2$-type
  $(\noe\tpTt \cup \tIN(\xx) \cup \tOUT(\yy))$ and its inverse to $\TpiTs\csps$.
  \item[\sticondOO]\label{sti-OO}
  For every $2$-type $\tpTt \in \TpiT$, add the $2$-type
  $(\noe\tpTt \cup \tOUT(\xx) \cup \tOUT(\yy))$ to $\TpiTs\csps$.
  \item[\sticondIB]\label{sti-IB}
  For every $1$-type $\tpIpp \in \TpiPs\csps$, $\tIN(\xx) \subseteq \tpIpp$ that
  comes from the inside, let $\tpTt$ be an arbitrary $2$-type connecting $\tpIpp$ and the black hole type $\tpIb$ that witnesses no
  message symbols for $\tpIpp$ and everything for $\tpIb$, that is
  $\xtp\tpTt = \tpIpp$, $\ytp\tpTt = \tpIb$, $(\lnot\sm(\xx,\yy)) \in \tpTt$ and
  $\sm(\yy,\xx) \in \tpTt$ for every $\sm \in \sms$. Add $\tpTt$ and its inverse
  to $\TpiTs\csps$.
  \item[\sticondOB]\label{sti-OB}
  For every $1$-type $\tpIpp \in \TpiPs\csps$, $\tOUT(\xx) \subseteq \tpIpp$
  that comes from the outside, let $\tpTt$ be an arbitrary $2$-type connecting
  $\tpIpp$ and the black hole type $\tpIb$ that witnesses every message symbol for $\tpIpp$ and for $\tpIb$, that is
  $\xtp\tpTt = \tpIpp$, $\ytp\tpTt = \tpIb$, $\sm(\xx,\yy) \in \tpTt$ and
  $\sm(\yy,\xx) \in \tpTt$ for every $\sm \in \sms$. Add $\tpTt$ and its inverse
  to $\TpiTs\csps$.
\end{itemize}
This completes the description of the spectral type instance
$\Tpi{\TpiPs\csps}{\TpiTs\csps}$.

The cosmic spectrum $\csps$ is \emph{locally consistent} if its spectral type
instance $\Tpi{\TpiPs\csps}{\TpiTs\csps}$ over the
$\Lvp\Fo2\nopow\Eea{(\sze-1)}\agrefine$-classified signature
$\ClSig{\SigS - \seq{\se} + \seq{\si, \sbh}}\sms$ is realizable.
\end{definition}
\begin{remark}\label{rem:csp-is-locally-consistent}
Let $\Tpi\TpiP\TpiT$ be a type instance over the
$\vFo2\Eea\sze\agrefine$-classified signature $\ClSig\SigS\sms$.
Let $\StrA$ be a model for $\Tpi\TpiP\TpiT$ such that $\relE =
\at\StrA\se$ is not full on $\domA$ and let $\eclX \in \Ecl\relE$ be any galaxy.
Then the cosmic spectrum $\csps = \cspIX\StrA\eclX$ is a locally consistent
cosmic spectrum over $\Tpi\TpiP\TpiT$.
\end{remark}
\begin{proof}
Let $\SigSp = \SigS - \seq{\se} + \seq{\si,\sbh}$.
We build a $\SigSp$-structure $\StrAp$ based on $\StrA$ that realizes the
spectral type instance $\Tpi{\TpiPs\csps}{\TpiTs\csps}$.
The domain of $\StrAp$ is 
$\domAp = \domA \cup \set{\eo}$, where $\eo$ is the new \emph{black hole}
element.
The $1$-type of every element $\ea \in \domAp$ is defined as follows:
\begin{itemize}
  \item[\refsticondI]
  If $\ea \in \eclX$ is inside the galaxy, then let
  $\tpIa\StrAp\ea = \noe{\tpIa\StrA\ea} \cup \tIN(\xx)$.
  \item[\refsticondO]
  If $\ea \in \domA \sub \eclX$ is outside the galaxy, then let
  $\tpIa\StrAp\ea = \noe{\tpIa\StrA\ea} \cup \tOUT(\xx)$.
  \item[\refsticondB] If $\ea = \eo$ is the black hole, then let
  $\tpIa\StrAp\ea = \beta$.
\end{itemize}
The $2$-type between every pair of distinct elements $\ea \neq \eb \in \domA'$
is defined as follows:
\begin{itemize}
  \item[\refsticondII]
  If both $\ea,\eb \in \eclX$ are inside, then let
  $\tpIab\StrAp\ea\eb = \noe{\tpIab\StrA\ea\eb} \cup \tIN(\xx) \cup \tIN(\yy)$.
  Note that this assignment is symmetric, that is we would assign
  $\inv{\tpIab\StrAp\ea\eb}$ to $\tpIab\StrAp\eb\ea$.
  \item[\refsticondIO]
  If $\ea \in \eclX$ is inside and $\eb \in \domA \sub
  \eclX$ is outside, then let
  $\tpIab\StrAp\ea\eb = \noe{\tpIab\StrA\ea\eb} \cup \tIN(\xx) \cup \tOUT(\yy)$.
  Note that this case covers the symmetric assignments where $\ea \in
  \domA\sub\eclX$ and $\eb \in \eclX$.
  \item[\refsticondOO]
  If both $\ea, \eb \in \domA\sub\eclX$ are outside, then let
  $\tpIab\StrAp\ea\eb = \noe{\tpIab\StrA\ea\eb} \cup \tOUT(\xx) \cup
  \tOUT(\yy)$. Note that this assignment is symmetric.
  \item[\refsticondIB]
  If $\ea \in \eclX$ is inside and $\eb = \eo$ is the black hole, let $\tpTt
  \in \TpiTs\csps$ be the unique $2$-type connecting $\tpIa\StrAp\ea$ and
  $\tpIb$ and assign $\tpIab\StrAp\ea\eb = \tpTt$.
  \item[\refsticondOB]
  If $\ea \in \domA\sub\eclX$ is outside and $\eb = \eo$ is the black hole, let
  $\tpTt \in \TpiTs\csps$ be the unique $2$-type connecting $\tpIa\StrAp\ea$ and
  $\tpIb$ and assign $\tpIab\StrAp\ea\eb = \tpTt$.
\end{itemize}
We have to verify that $\StrAp$ is indeed a $\ClSig\SigSp\sms$-structure, that
is that the star-type of every element of $\ea \in \domAp$ contains
$\sm(\xx,\yy)$ for every message symbol $\sm \in \sms$:
\begin{itemize}
\item[\refsticondI] If $\ea \in \eclX$ is inside, then every
$\sm \in \sms$ that is included in $\stpIa\StrA\ea$ is also included in
$\stpIa\StrAp\ea$, since the $2$-type $\noe\tpTt$ contains $\sm(\xx,\yy)$ iff
$\tpTt$ contains $\sm(\xx,\yy)$.
\item[\refsticondO] If $\ea \in \domA\sub\eclX$ is outside, then we just need to
recall that at stage~\refsticondOB, we have used a $2$-type $\tpIab\StrAp\ea\eo
= \tpTt$ that witnesses every message symbol for $\ea$: $\sm(\xx,\yy) \in
\tpTt$ for every $\sm \in \sms$.
\item[\refsticondB] If $\ea = \eo$ is the black hole, then we just need to
recall that $\eclX$ is nonempty and at stage~\refsticondIB~for every $\ea \in
\eclX$ we have used a $2$-type $\tpTt = \tpIab\StrAp\ea\eo$ that witnesses every
message symbol for $\eo$: $\sm(\yy,\xx) \in \tpTt$ for every $\sm \in \sms$.
\end{itemize}
Since $\StrA$ was a model for $\Tpi\TpiP\TpiT$, it is clear that $\StrAp$ is a
model for $\Tpi{\TpiPs\csps}{\TpiTs\csps}$.
\end{proof}

\begin{definition}
A certificate $\Cert$ for the type instance $\Tpi\TpiP\TpiT$ over the
$\vFo2\Eea\sze\agrefine$-classified signature $\ClSig\SigS\sms$ is a nonempty
set of cosmic spectrums for $\Tpi\TpiP\TpiT$ satisfying the following
conditions:
\begin{itemize}
  \item[\certcondIp]\label{cond:cert-Ip}
  If $\tpTt \in \csps$ for some $\csps \in \Cert$, then
  $\inv\tpTt \in \cspsp$ for some $\cspsp \in \Cert$, that is there are
  witnesses for the endpoints of every (cosmic) $2$-type used in the
  certificate. Equivalently, $\cup\Cert$ is closed under inversion.
  
  Let $\TpiTp = \cup\Cert$. Clearly, $\TpiTp \subseteq \TpiT$. We require that
  $\Tpi\TpiP\TpiTp$ is a type instance --- the filtered type instance --- and
  that all cosmic spectrums $\csps \in \Cert$ be cosmic spectrums over the
  filtered type instance. In this context we refer to the king types $\TKg =
  \TKgi\TpiP\TpiTp$, the noble types $\TNo = \TNoi\TpiP\TpiTp$ and the peasant
  types $\TPs = \TPsi\TpiP\TpiTp$ with respect to the filtered type instance.
  This is the analogue of Condition~\refcertcond1 from the case with no
  equivalence symbols.
  \item[\certcondIIp]\label{cond:cert-IIp}
  If $\tpIp \in \TpiP$ then some (possibly many) $\csps
  \in \Cert$ has $\tpIp \in (\xtp\restriction\csps)$, that is every
  $1$-type is witnessed. This is the analogue of
  Condition~\refcertcond2 from the case with no equivalence symbols.
  \item[\certcondIIIp]\label{cond:cert-IIIp}
  If $\tpIn \in \TNoi\TpiP\TpiTp$ is a noble type, then only one $\csps \in
  \Cert$ has $\tpIn \in (\xtp\restriction\csps)$, that is every noble type is
  witnessed once. This is the analogue of Condition~\refcertcond3 from the case
  with no equivalence symbols.
  \item[\certcondIVp]\label{cond:cert-IVp}
  If $\tpIp,\tpIpp \in \TpiP$ are (possibly the same) $1$-types that are not
  cosmically connectable, that is no cosmic $\tpTt \in \TpiTp$ has $\xtp\tpTt =
  \tpIp$ and $\ytp\tpTt = \tpIpp$, then $\tpIp, \tpIpp \in \TNoi\TpiP\TpiTp$ and
  there is a (noble, unique by~\refcertcondIIIp) cosmic spectrum $\csps \in
  \Cert$ such that $\tpIp, \tpIpp \in \csps$, that is two $1$-types are not
   cosmically connectable iff they are noble and have the same witnessing cosmic
   spectrum. This is the analogue of Condition~\refcertcond4 from the case with
   no equivalence symbols.
\end{itemize}
\end{definition}
Again, in general the size of a certificate is only exponentially bounded in
terms of the size of the type instance. But again, polynomial certificates
exist:
\begin{lemma}[Certificate extraction]\label{lem:cert-extr-many}
Let $\StrA$ be a nobly distinguished model for the type instance
$\Tpi\TpiP\TpiT$ such that $\relE = \at\StrA\se$ is not full on $\domA$.
Let $\Tpi\TpiP\TpiTp$ be the type instance of $\StrA$. For every cosmic $\tpTt
\in \TpiTp$, let $\eat\tpTt \neq \ebt\tpTt \in \domA$ be two distinct elements
realizing $\tpTt$: $\tpIab\StrA{\eat\tpTt}{\ebt\tpTt} = \tpTt$. Note that
necessarily the elements $\eat\tpTt$ and $\ebt\tpTt$ are in different galaxies.
The choice is made symmetrically, that is $\eat{\inv\tpTt} = \ebt\tpTt$ and
$\ebt\tpTt = \eat{\inv\tpTt}$. Let
\[
  \Cert = \setbd{\cspIX\StrA{\relE[\eat\tpTt]}}{\tpTt \in \TpiTp \cap
  \TpTc\SigS}.
\]
Then $\Cert$ is a certificate for $\Tpi\TpiP\TpiT$.
Moreover, its size is linearly bounded by $\card\TpiTp$, hence also by the size
of $\Tpi\TpiP\TpiT$.
\end{lemma}
\begin{proof}
TODO
\end{proof}