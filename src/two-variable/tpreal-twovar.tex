% Type realizibility of the two-variable first-order logic
Recall from~\Cref{sec:scott-nf} about normal forms that every $\vFo2$-sentence
$\fphi$ can be reduced in polynomial time sentence $\sctr\fphi$ in Scott normal
form:
\[
  \forall\xx\forall\yy (\falpp0(\xx,\yy) \lor \xx = \yy) \land
  \bigwedge_{1 \leq \ii \leq \nm} \forall\xx\exists\yy(
  \falpp\ii(\xx,\yy) \land \xx \neq \yy),
\]
where the formulas $\falpp\ii$ are quantifier-free and use at most linearly many
new unary predicate symbols. We refer to $\falpp0$ as the \emph{universal part},
or the $\forall\forall$-part of the formula $\sctr\fphi$ and to $\falpp\ii$ as
the \emph{existential parts}, or the $\forall\exists$-parts of the formula,
for $\ii \in [1,\nm]$.
For any formula in Scott normal form, we may replace its existential parts
by fresh binary predicate symbols: for $\ii \in [1,\nm]$, let $\smm\ii$ be a
fresh binary predicate symbol with the intended interpretation
$\forall\xx\forall\yy (\smm\ii(\xx,\yy) \lequ \falpp\ii(\xx,\yy))$.
Since this is a universal sentence, it can be incorporated into the
$\forall\forall$-part $\falpp0$ of the formula.
We refer to the symbols $\gls{message-symbol-m-i}$ as the \emph{message
symbols}.
Hence the formula can be transformed in polynomial time to the form:
\begin{equation}\label{eq:twovar-msg-nf}
  \forall\xx\forall\yy (\falp(\xx,\yy) \lor \xx = \yy) \land
  \bigwedge_{1 \leq \ii \leq \nm} \forall\xx\exists\yy
  (\smm\ii(\xx,\yy) \land \xx \neq \yy),
\end{equation}
where the $\forall\forall$-part $\falp$ is quantifier-free and over an extended
signature. For convenience, we make the existential parts of the formula part of
the signature, so we can focus only on the universal part. The following is a
term similar to the one defined in~\cite{MALQ:MALQ201400102}:
\begin{definition}
A \emph{classified signature} $\gls{classified-signature-S-m}$ for the
two-variable first-order logic $\vFo2$ is a predicate signature $\SigS$ together
with a sequence $\sms = \smm1\smm2\dots\smm\nm$ of distinct binary predicate
symbols from $\SigS$ having intended interpretation
\begin{equation}\label{eq:twovar-ms-iinter}
  \bigwedge_{1 \leq \ii \leq \nm} \forall\xx\exists\yy 
  (\smm\ii(\xx,\yy) \land \xx \neq \yy).
\end{equation}
\end{definition}
That is, a classified signature \emph{automatically includes} the
$\forall\exists$-part of formulas and $\ClSig\SigS\sms$-structures
automatically satisfy the $\forall\exists$-part.

\begin{definition}
The \emph{(finite) classified satisfiability problem for two-variable
first-order logic} is:
given a classified signature $\ClSig\SigS\sms$ and a quantifier-free
$\vFoF2\SigS$-formula $\falp(\xx,\yy)$, is there a (finite)
$\ClSig\SigS\sms$-structure satisfying~\cref{eq:twovar-msg-nf}.
Denote tha classified satisfiability problem for two-variable first-order logic
by $\ClSat{\vFo2}$ and its finite version by $\FinClSat{\vFo2}$.
\end{definition}

Scott normal form shows that (finite) satisfiability reduces in polynomial time
to (finite) classified satisfiability:
\[
  \FinASat{\vFo2} \red\cP \FinAClSat{\vFo2}.
\]

Let $\ClSig\SigS\sms$ be a classified signature for $\vFo2$.
\begin{definition}
A \emph{type instance} $\gls{type-instance-P-T}$ over $\ClSig\SigS\sms$ is a
pair of a set of $1$-types $\TpiP \subseteq \TpI\SigS$ and a set of $2$-types
$\TpiT \subseteq \TpT\SigS$ such that $\xtp\tpTt \in \TpiP$ and $\ytp\tpTt \in
\TpiP$ for all $\tpTt \in \TpiT$.

A $\ClSig\SigS\sms$-structure $\StrA$ \emph{realizes} (or is a model of)
$\Tpi\TpiP\TpiT$ if $\tpIa\StrA\ea \in \TpiP$ for all $\ea \in \domA$ and 
$\tpIab\StrA\ea\eb \in \TpiT$ for all $\ea \in \domA$ and $\eb \in \domA \sub \set{\ea}$.
The structure \emph{fully realizes} $\Tpi\TpiP\TpiT$ if additionally for every
$1$-type $\tpIp \in \TpiP$ there is $\ea \in \domA$ such that $\tpIa\StrA\ea =
\tpIp$ (we don't require an analogous condition for the $2$-types).
The \emph{canonical type instance} of $\StrA$ is the type instance
$\Tpi\TpiP\TpiT$ defined by:
\begin{align*}
\TpiP &= \setbd{\tpIa\StrA\ea}{\ea\in\domA} \\
\TpiT &= \setbd{\tpIab\StrA\ea\eb}{\ea\neq\eb \in \domA}.
\end{align*}

The \emph{(finite) (full) type realizibility problem for $\vFo2$} is the
following:
given a classified signature $\ClSig\SigS\sms$ and a type instance
$\Tpi\TpiP\TpiT$ over $\ClSig\SigS\sms$, is there a (finite)
$\ClSig\SigS\sms$-structure that (fully) realizes $\Tpi\TpiP\TpiT$.
Denote the type realizibility problem for $\vFo2$ by
$\Real{\vFo2}$ and its finite version by $\FinReal{\vFo2}$.
Denote the full type realizibility problem for $\vFo2$ by
$\FullReal{\vFo2}$ and its finite version by $\FinFullReal{\vFo2}$.
\end{definition}

\begin{remark}
Since the set of $1$-types realized in any model of $\Tpi\TpiP\TpiT$ is a subset
of $\TpiP$, by guessing this subset we can reduce the (finite) type
satisfiability problem to the (finite) full type satisfiability problem in
nondeterministic polynomial time:
\[
  \FinAReal{\vFo2} \red\cNP \FinAFullReal{\vFo2}.
\]
\end{remark}

\begin{remark}
Let $\alpha(\xx,\yy)$ be a quantifier-free $\vFoF2\SigS$-formula. Let $\TpiP$ be
the set of those $1$-types over $\ClSig\SigS\sms$ that are consistent with
$\alpha$ and $\TpiT$ be the set of those $2$-types over $\ClSig\SigS\sms$ that
are consistent with $\alpha$ and the intended
interpretation~\cref{eq:twovar-ms-iinter}. Then a $\ClSig\SigS\sms$-structure
$\StrA$ is a model for $\alpha$ iff it is a model for $\Tpi\TpiP\TpiT$.

Recall that the number of possible $1$-types or $2$-types over $\SigS$ is
exponentially bounded by the length $\sizes$ of $\SigS$ and that the cardinality
of a $1$-type or a $2$-type over $\SigS$ is polynomially bounded by $\sizes$.
Hence we can reduce the (finite) classified satisfiability problem to the
(finite) type realizibility problem in exponential time:
\[
  \FinASat{\vFo2} \red\cExpTime \FinAReal{\vFo2}.
\]
\end{remark}

We study the type realizibility problem instead of directly studying the
satisfiability problem because it is flexible enough to allow us to employ its
solutions in simpler logics to construct a solution in logics with more
equivalences.

Let $\Tpi\TpiP\TpiT$ be a type instance over $\ClSig\SigS\sms$.
\begin{definition}
Two $1$-types $\tpIp, \tpIpp \in \TpiP$ are \emph{connectible} (written
$\gls{connectible-p-pp}$), if $\tpIp = \xtp\tpTt$ and $\tpIpp = \ytp\tpTt$ for
some $\tpTt \in \TpiT$. A $1$-type $\tpIk \in \TpiP$ is a \emph{king type} if
$\tpIk \not\conn \tpIk$.
\end{definition}
\begin{remark}
If two distinct $1$-types $\tpIp, \tpIpp \in \TpiP$ are not connectible, then
$\Tpi\TpiP\TpiT$ is not fully realizible.

If $\StrA$ is a full model for $\Tpi\TpiP\TpiT$ then every two distinct elements
$\ea \neq \eb \in \domA$ realize connectible $1$-types. Hence every king type
$\tpIk \in \TpiP$ is realized once in $\StrA$.
\end{remark}