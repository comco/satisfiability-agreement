% Type realizibility of the two-variable first-order logic
Recall from~\Cref{sec:scott-nf} about normal forms that every $\vFo2$-sentence
$\fphi$ can be reduced in polynomial time sentence $\sctr\fphi$ in Scott normal
form:
\[
  \forall\xx\forall\yy (\falpp0(\xx,\yy) \lor \xx = \yy) \land
  \bigwedge_{1 \leq \ii \leq \nm} \forall\xx\exists\yy(
  \falpp\ii(\xx,\yy) \land \xx \neq \yy),
\]
where the formulas $\falpp\ii$ are quantifier-free and use at most linearly many
new unary predicate symbols. We refer to $\falpp0$ as the \emph{universal part},
or the $\forall\forall$-part of the formula $\sctr\fphi$ and to $\falpp\ii$ as
the \emph{existential parts}, or the $\forall\exists$-parts of the formula,
for $\ii \in [1,\nm]$.
For any formula in Scott normal form, we may replace its existential parts
by fresh binary predicate symbols: for $\ii \in [1,\nm]$, let $\smm\ii$ be a
fresh binary predicate symbol with the intended interpretation
$\forall\xx\forall\yy (\smm\ii(\xx,\yy) \lequ \falpp\ii(\xx,\yy))$.
Since this is a universal sentence, it can be incorporated into the
$\forall\forall$-part $\falpp0$ of the formula.
We refer to the symbols $\gls{message-symbol-m-i}$ as the \emph{message
symbols}.
Hence the formula can be transformed in polynomial time to the form:
\begin{equation}\label{eq:twovar-msg-nf}
  \forall\xx\forall\yy (\falp(\xx,\yy) \lor \xx = \yy) \land
  \bigwedge_{1 \leq \ii \leq \nm} \forall\xx\exists\yy
  (\smm\ii(\xx,\yy) \land \xx \neq \yy),
\end{equation}
where the $\forall\forall$-part $\falp$ is quantifier-free and over an extended
signature. For convenience, we make the existential parts of the formula part of
the signature, so we can focus only on the universal part. The following is a
term similar to the one defined in~\cite{MALQ:MALQ201400102}:
\begin{definition}
A \emph{classified signature} $\gls{classified-signature-S-m}$ for the
two-variable first-order logic $\vFo2$ is a predicate signature $\SigS$ together
with a sequence $\sms = \smm1\smm2\dots\smm\nm$ of distinct binary predicate
symbols from $\SigS$ having intended interpretation
\begin{equation}\label{eq:twovar-ms-iinter}
  \bigwedge_{1 \leq \ii \leq \nm} \forall\xx\exists\yy 
  (\smm\ii(\xx,\yy) \land \xx \neq \yy).
\end{equation}
\end{definition}
That is, a classified signature \emph{automatically includes} the
$\forall\exists$-part of formulas and $\ClSig\SigS\sms$-structures
automatically satisfy the $\forall\exists$-part.

\begin{definition}
The \emph{(finite) classified satisfiability problem for two-variable
first-order logic} is:
given a classified signature $\ClSig\SigS\sms$ and a quantifier-free
$\vFoF2\SigS$-formula $\falp(\xx,\yy)$, is there a (finite)
$\ClSig\SigS\sms$-structure satisfying~\cref{eq:twovar-msg-nf}.
Denote tha classified satisfiability problem for two-variable first-order logic
by $\ClSat{\vFo2}$ and its finite version by $\FinClSat{\vFo2}$.
\end{definition}

Scott normal form shows that (finite) satisfiability reduces in polynomial time
to (finite) classified satisfiability:
\[
  \FinASat{\vFo2} \red\cP \FinAClSat{\vFo2}.
\]

Let $\ClSig\SigS\sms$ be a classified signature for $\vFo2$.
\begin{definition}
A \emph{type instance} $\gls{type-instance-P-T}$ over $\ClSig\SigS\sms$ is a
pair of a set of $1$-types $\TpiP \subseteq \TpI\SigS$ and a set of $2$-types
$\TpiT \subseteq \TpT\SigS$ such that $\xtp\tpTt \in \TpiP$ and $\ytp\tpTt \in
\TpiP$ for all $\tpTt \in \TpiT$.

A $\ClSig\SigS\sms$-structure $\StrA$ \emph{realizes} (or is a model of)
$\Tpi\TpiP\TpiT$ if it contains at least $2$ elements, $\tpIa\StrA\ea \in \TpiP$
for all $\ea \in \domA$ and $\tpIab\StrA\ea\eb \in \TpiT$ for all $\ea \in \domA$
and $\eb \in \domA \sub \set{\ea}$.
The structure \emph{fully realizes} $\Tpi\TpiP\TpiT$ if additionally for every
$1$-type $\tpIp \in \TpiP$ there is $\ea \in \domA$ such that $\tpIa\StrA\ea =
\tpIp$ (we don't require an analogous condition for the $2$-types).
The \emph{characteristic type instance} of $\StrA$ is the type instance
$\Tpi\TpiP\TpiT$ defined by:
\begin{align*}
\TpiP &= \setbd{\tpIa\StrA\ea}{\ea\in\domA} \\
\TpiT &= \setbd{\tpIab\StrA\ea\eb}{\ea\neq\eb \in \domA}.
\end{align*}

The \emph{(finite) (full) type realizibility problem for $\vFo2$} is the
following:
given a classified signature $\ClSig\SigS\sms$ and a type instance
$\Tpi\TpiP\TpiT$ over $\ClSig\SigS\sms$, is there a (finite)
$\ClSig\SigS\sms$-structure that (fully) realizes $\Tpi\TpiP\TpiT$.
Denote the type realizibility problem for $\vFo2$ by
$\Real{\vFo2}$ and its finite version by $\FinReal{\vFo2}$.
Denote the full type realizibility problem for $\vFo2$ by
$\FullReal{\vFo2}$ and its finite version by $\FinFullReal{\vFo2}$.
\end{definition}

\begin{remark}
Since the set of $1$-types realized in any model of $\Tpi\TpiP\TpiT$ is a subset
of $\TpiP$, by guessing this subset we can reduce the (finite) type
satisfiability problem to the (finite) full type satisfiability problem in
nondeterministic polynomial time:
\[
  \FinAReal{\vFo2} \red\cNP \FinAFullReal{\vFo2}.
\]
\end{remark}

\begin{remark}
Let $\alpha(\xx,\yy)$ be a quantifier-free $\vFoF2\SigS$-formula. Let $\TpiP$ be
the set of those $1$-types over $\ClSig\SigS\sms$ that are consistent with
$\alpha$ and $\TpiT$ be the set of those $2$-types over $\ClSig\SigS\sms$ that
are consistent with $\alpha$ and the intended
interpretation~\cref{eq:twovar-ms-iinter}. Then a $\ClSig\SigS\sms$-structure
$\StrA$ is a model for $\alpha$ iff it is a model for $\Tpi\TpiP\TpiT$.

Recall that the number of possible $1$-types or $2$-types over $\SigS$ is
exponentially bounded by the length $\sizes$ of $\SigS$ and that the cardinality
of a $1$-type or a $2$-type over $\SigS$ is polynomially bounded by $\sizes$.
Hence we can reduce the (finite) classified satisfiability problem to the
(finite) type realizibility problem in exponential time:
\[
  \FinASat{\vFo2} \red\cExpTime \FinAReal{\vFo2}.
\]
\end{remark}

We study the type realizibility problem instead of directly studying the
satisfiability problem because it is flexible enough to allow us to employ its
solutions in simpler logics to construct a solution in logics with more
equivalences.

Let $\Tpi\TpiP\TpiT$ be a type instance over $\ClSig\SigS\sms$.
\begin{definition}
Two $1$-types $\tpIp, \tpIpp \in \TpiP$ are \emph{connectible} (written
$\gls{connectible-p-pp}$), if $\tpIp = \xtp\tpTt$ and $\tpIpp = \ytp\tpTt$ for
some $\tpTt \in \TpiT$. A $1$-type $\tpIk \in \TpiP$ is a \emph{king type} if
$\tpIk \not\conn \tpIk$.
\end{definition}
\begin{remark}\label{rem:twovar-king-once}
If two distinct $1$-types $\tpIp, \tpIpp \in \TpiP$ are not connectible, then
$\Tpi\TpiP\TpiT$ is not fully realizible.

If $\StrA$ is a full model for $\Tpi\TpiP\TpiT$ then every two distinct elements
$\ea \neq \eb \in \domA$ realize connectible $1$-types. Hence every king type
$\tpIk \in \TpiP$ is realized once in $\StrA$.

A type instance $\Tpi\TpiP\TpiT$ is \emph{connected} if every two distinct
$1$-types $\tpIp, \tpIpp \in \TpiP$ are connectible.
\end{remark}

The next definition characterizes the local structure of the ray of $2$-types
realized around an element of a structure.
\begin{definition}
A \emph{star-type} $\stps \subseteq \TpiT$ for the type instance
$\Tpi\TpiP\TpiT$ is a nonempty set of $2$-types satisfying the following conditions:
\begin{enumerate}
  \item\label{cond:star-1} If $\tpTt, \tpTtp \in \stps$ then $\xtp\tpTt =
  \xtp\tpTtp$, that is the $\xx$-type of every element of $\stps$ is the same.
  Denote the $\xx$-type of every element of $\stps$ by $\xtp\stps$.
  \item\label{cond:star-2} If $\tpIk = \xtp\stps$ is a king type, then no $\tpTt
  \in \stps$ has $\ytp\tpTt = \tpIk$.
  \item\label{cond:star-3} If $\tpIk \neq \xtp\stps$ is any king type, then one
  $\tpTt \in \stps$ has $\ytp\tpTt = \tpIk$.
  \item\label{cond:star-4} If $\sm \in \sms$, then $\sm \in \tpTt$ for some
  (possibly many) $\tpTt \in \stps$.
\end{enumerate}
Note that the size of a star-type is polynomially bounded by the size of the
type instance.

If $\StrA$ is a $\Tpi\TpiP\TpiT$-structure and $\ea \in \domA$, the
\emph{star-type realized by $\ea$} is:
\[
  \stpIa\StrA\ea = \setbd{\tpIab\StrA\ea\eb}{\eb \in \domA \sub \set{\ea}}.
\]
It is straighforward to check that this indeed defines a star-type.
\end{definition}

\begin{remark}[Star-type extension]
Let $\stps$ be a star-type, with $\xx$-type $\tpIp = \xtp\stps$.
Let $\tpTt \in \TpiT$ be a $2$-type and suppose that both $\xtp\tpTt$ and
$\ytp\tpTt$ are not king types.
Then $\stps \cup \set{\tpTt}$ is also a star-type.
\end{remark}

\begin{definition}
A \emph{certificate} $\Cert$ for the type instance $\Tpi\TpiP\TpiT$ is a
nonempty set of star-types satisfying the following conditions:
\begin{enumerate}
  \item If $\tpIk \in \TpiP$ is a king type, then one $\stps \in \Cert$ has
  $\xtp\stps = \tpIk$, that is every king type is witnessed once.
  \item If $\tpIp \in \TpiP$ is not a king type, then some (possibly many)
  $\stps \in \Cert$ has $\xtp\stps = \tpIp$, that is every other $1$-type is
  witnessed.
  \item If $\tpTt \in \cup\Cert$, then $\inv\tpTt \in \cup\Cert$, that is there
  are witnesses for the endpoints of every $2$-type used in the certificate.
\end{enumerate}
\end{definition}
Note that in general the size of a certificate may be exponential in terms of
the size of the type instance. However, polynomial certificates exist:
\begin{lemma}[Certificate extraction]\label{lem:cert-extract}
Let $\StrA$ be a full model for the type instance $\Tpi\TpiP\TpiT$.
Let $\TpiTp \subseteq \TpiT$ be a (nonempty) set of $2$-types realized in
$\StrA$.
For every $2$-type $\tpTt \in \TpiTp$ realized in $\StrA$, let $(\ea_\tpTt,
\eb_\tpTt) \in \domA\cprod\domA$ be a pair of elements realizing $\tpTt$.
Let 
\[
  \Cert = \setbd{\stpIa\StrA{\ea_\tpTt}, \stpIa\StrA{\eb_\tpTt}}{\tpTt \in
  \TpiTp}.
\]
Then $\Cert$ is a certificate for the type instance. Moreover, its size is
polynomially bounded by the size of the type instance.
\end{lemma}
\begin{proof}
We check the conditions for a certificate:
\begin{enumerate}
  \item If $\tpIk \in \TpiP$ is a king type, by~\Cref{rem:twovar-king-once} it
  is realized once in $\StrA$. If $\ec \in \domA$ realizes $\tpIk$ and $\stps
  \in \Cert$ has $\xtp\stps = \tpIk$, then $\ea_\stps = \ec$ and hence
  $\stps = \stpIa\StrA\ec$ is unique. On the other hand, such $\stps \in \Cert$
  exists since $\StrA$ is a full model for $\Tpi\TpiP\TpiT$ and $\domA$ has
  cardinality at least $2$.
  \item Let $\tpIp \in \TpiP$ be a $1$-type. Again, since $\StrA$ is a full
  model for $\Tpi\TpiP\TpiT$ and $\domA$ has cardinality at least $2$, we may
  choose a pair of distinct elements such that the first element realizes
  $\tpIp$.
  \item The third condition follows immediately from the definition of $\Cert$.
\end{enumerate}
\end{proof}

\begin{lemma}[Certificate expansion]\label{lem:cert-expand}
Let $\Cert$ be a certificate for the connected type instance $\Tpi\TpiP\TpiT$.
Then $\Tpi\TpiP\TpiT$ has a finite full model.
\end{lemma}
\begin{proof}
Let $t = \card{\TpiT}$.
We build a model $\StrA$ for $\Tpi\TpiP\TpiT$ as follows. For every star-type
$\stps \in \Cert$, if $\xtp\stps$ is a king type, there will be a unique element
$\ec \in \domA$ that wants to realize $\stps$. Otherwise, there will be
$(t+1)$ elements that want to realize $\stps$.
This construction fixes the $1$-type of every element the structure. 
Now we need to consistently assign $2$-types between all pairs of distinct
elements so that they realize their expected star-types.

Let $\ea \in \domA$ be an element having expected star-type $\stps$ and
expected $1$-type $\tpIp = \xtp\sigma$.
By the third condition for a certificate, for every $2$-type $\tau \in \sigma$
there is a star-type $\stps_\tau \in \Cert$ such that $\inv\tau \in \stps_\tau$.
We claim that we may choose distinct elements $\eb_\tau$ (also distinct from
$\ea$), such that $\eb_\tau$ has expected star-type $\stps_\tau$.
First suppose that $\tpIkp = \ytp\tpTt$ is a king type.
By conditions~\ref{cond:star-2} and~\ref{cond:star-3} in the definition of a
star-type, $\tpIkp \neq \tpIp$ and there is no other $\tpTtp \in \stps$ having
$\ytp\tpTtp = \tpIkp$. Then we take $\eb_\tau$ to be the unique element having
expected $1$-type $\tpIkp$.
Next suppose that $\tpIpp = \ytp\tpTt$ is not a king type.
Since $\stps \subseteq \TpiT$ has cardinality at most $t$, there are at most $t$
types $\tpTtp \in \stps$ having $\ytp\tpTtp = \tpIpp$.
For every such $\tpTtp$ the expected star-type of its target $\stps' =
\stps_{\tpTtp}$ has a $\xx$-type $\xtp\stps' = \tpIpp$, which is not a king
type, so there are $(t+1)$ elements that have expected star-type $\stps'$.
Accounting for the possibly of the element $\ea$ itself to have expected
star-type $\stps'$, we can see that there are enough distinct elements
$\eb_\tau$. Fixing $\tpIab\StrA\ea{\eb_\tau} = \tau$ for these elements takes
care of the star-type of $\ea$. By doing this step-by-step by readjusting the
remaining sets of $2$-types that need to be realized in order for an element
to realize its expected star-type, we may realize every expected star-type of
every element in the structure. However, there might still be unassigned
$2$-types between pairs of distict elements left.
Suppose that $\ea \neq \eb \in \domA$ be such a pair. We claim that
$\tpIp = \tpIa\StrA\ea$ and $\tpIpp = \tpIa\StrA\eb$ are not king types. Without
loss of generality, suppose towards a contradiction that $\tpIp$ is a king type.
Then $\tpIp \neq \tpIpp$ and since the type instance is connected, we may find a
$2$-type $\tau \in \TpiT$ connecting $\tpIp$ and $\tpIpp$.
We may fix $\tpIab\StrA\ea\eb = \tau$ and this will not destroy the model since
$\sigma \cup \set{\tau}$ remains a star-type.
\end{proof}

\begin{proposition}
The logic $\vFo2$ has the finite model property. The (finite) full type
realizibility problem for $\vFo2$ is in $\cNP$.
\end{proposition}
\begin{proof}
Let $\Tpi\TpiP\TpiT$ be a type instance. If it is not connected, then it has no
model by~\Cref{rem:twovar-king-once}. It is trivial to check satisfiability in
the class of structures of cardinality $1$. If this fails to yield a model,
guess a certificate of polynomial size (and verify in polynomial time that this
is indeed a certificate).
By~\Cref{lem:cert-extract} and~\Cref{lem:cert-expand} such a certificate exists
iff $\Tpi\TpiP\TpiT$ is fully satisfiable.
\end{proof}
This gives us another proof of a standard result:
\begin{corollary}
The logic $\vFo2$ has the finite model property and its (finite) satisfiability
problem is in $\cNExpTime$.
\end{corollary}