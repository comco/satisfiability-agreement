% Type realizibility of the two-variable first-order logic
Recall from~\Cref{sec:scott-nf} about normal forms that every $\vFo2$-sentence
$\fphi$ can be reduced in deterministic polynomial time to a sentence
$\sctr\fphi$ in Scott normal form:
\[
  \forall\xx\forall\yy (\falpp0(\xx,\yy) \lor \xx = \yy) \land
  \bigwedge_{1 \leq \ii \leq \nm} \forall\xx\exists\yy(
  \falpp\ii(\xx,\yy) \land \xx \neq \yy),
\]
where $\nm \geq 1$, all the formulas $\falpp\ii$ are quantifier-free and use at
most linearly many new unary predicate symbols.
The semantic connection between $\fphi$ and $\sctr\fphi$ is that they are
essentially equisatisfiable. More precisely, every model for $\fphi$ of
cardinality at least $2$ can be enriched to a model for $\sctr\fphi$ and also
every model of $\sctr\fphi$ (which by $\nm \geq 1$ must have cardinality at
least $2$) is a model for $\fphi$.
We refer to $\falpp0$ as the \emph{universal part} of the formula $\sctr\fphi$
and to $\falpp\ii$ for $\ii\in[1,\nm]$ as the \emph{existential parts} of
$\sctr\fphi$.

For any formula $\sctr\fphi$ in Scott normal form, we may replace its
existential parts by fresh binary predicate symbols:
for $\ii \in [1,\nm]$ let $\smm\ii$ be a fresh binary predicate symbol with
intended interpretation
$\forall\xx\forall\yy (\smm\ii(\xx,\yy) \lequ \falpp\ii(\xx,\yy))$.
Since this is a universal sentence, it can be added to the universal part
$\falpp0$.
The symbols $\gls{message-symbol-m-i}$ are the \emph{message symbols}.
Hence $\sctr\fphi$ can be transformed in deterministic polynomial time to the
form:
\begin{equation}\label{eq:twovar-msg-nf}
  \forall\xx\forall\yy (\falp(\xx,\yy) \lor \xx = \yy) \land
  \bigwedge_{1 \leq \ii \leq \nm} \forall\xx\exists\yy
  (\smm\ii(\xx,\yy) \land \xx \neq \yy),
\end{equation}
where the universal part $\falp$ is quantifier-free and over an extended
signature.
For convenience, we make the existential parts part of the signature, so we can
focus only on the universal part.
The following term is similar to the one defined in~\cite{MALQ:MALQ201400102}:
\begin{definition}
A \emph{classified signature} $\gls{classified-signature-S-m}$ for the
two-variable first-order logic $\vFo2$ is a predicate signature $\SigS$ together
with a nonempty sequence $\sms = \smm1\smm2\dots\smm\nm$ of distinct binary
predicate symbols from $\SigS$ having intended interpretation
\begin{equation}\label{eq:twovar-ms-iinter}
  \bigwedge_{1 \leq \ii \leq \nm} \forall\xx\exists\yy 
  (\smm\ii(\xx,\yy) \land \xx \neq \yy).
\end{equation}
\end{definition}
That is, a classified signature \emph{automatically includes} the
existential parts, so $\ClSig\SigS\sms$-structures \emph{automatically satisfy}
the the existential parts:
\begin{definition}
A structure $\StrA$ for the classified signature $\ClSig\SigS\sms$ is a
structure for the predicate signature $\SigS$ that satisfies the intended
interpretation~\cref{eq:twovar-ms-iinter} of the message symbols.
Note that $\StrA$ must have cardinality at least $2$ by $\nm \geq 1$.
\end{definition}

\begin{definition}\label{def:clsig-twovar}
The \emph{(finite) classified satisfiability problem for two-variable
first-order logic} is:
given a classified signature $\ClSig\SigS\sms$ and a quantifier-free
$\vFoF2\SigS$-formula $\falp(\xx,\yy)$, is there a (finite)
$\ClSig\SigS\sms$-structure $\StrA$ satisfying~\cref{eq:twovar-msg-nf}.
Note that since $\StrA$ is a $\ClSig\SigS\sms$-structure, it must also
satisfy~\cref{eq:twovar-ms-iinter} and must have cardinality at least $2$.
Denote the classified satisfiability problem by $\ClSat{\vFo2}$ and its finite
version by $\FinClSat{\vFo2}$.
\end{definition}

\begin{remark}
The problem of (finite) satisfiability reduces in nondeterministic polynomial
time to the problem of (finite) classified satisfiability:
\[
  \FinASat{\vFo2} \red\cNP \FinAClSat{\vFo2}.
\]
\end{remark}
\begin{proof}

Note that (finite) satisfiability in the class of models of cardinality $1$ is
trivially decidable in nondeterministic polynomial time --- just guess the
atomic $1$-type (whose size is polynomially bounded by the size of the
predicate signature) of the unique element of the structure and check (in
deterministic polynomial time) that it satisfies the original formula.

Scott normal form shows that (finite) satisfiability in the class of models of
cardinality at least $2$ reduces in deterministic polynomial time to (finite)
classified satisfiability.
Hence the following nondeterministic polynomial time procedure reduces an
instance $(\SigS, \fphi)$ of the (finite) satisfiability problem to an instance
$(\ClSig\SigSp\sms, \falp)$ of the (finite) classified satisfiability problem:
First check if $\fphi$ is satisfiable in the class of models of cardinality $1$.
If that is the case, then extend $\SigS$ to $\SigSp$ by adding a single message
symbol $\smm1$ and let $\falp = (\xx = \xx)$ be a fixed predicate tautology.
Otherwise transform $\fphi$ into the form~\cref{eq:twovar-msg-nf} and let
$\falp$ be the universal part of that normal form.
\end{proof}

A \emph{type instance} $\TI \subseteq \TpT\SigS$ over the classified signature
$\ClSig\SigS\sms$ is a nonempty set of \twotypes/ that is closed under
inversion.
The set of \onetypes/ included in the type instance $\TI$ is
$\TP\TI = \setbd{\xtp\tpt}{\tpt\in\TI}$.
Two \onetypes/ $\tpp,\tppp \in \TP\TI$ are \emph{connectable} if some
$\tpTt\in\TI$ connects them.
Connectability is symmetric, however it is not necessarily neither transitive
nor reflexive.
A \onetype/ $\tpk$ is a \emph{king type} if it is not connectable with itself;
the set of king types is $\TK\TI$.
A \onetype/ $\tpp$ that is not a king type is a \emph{worker type};
the set of worker types is $\TW\TI$.
So we have $\TP\TI = \TK\TI \cup \TW\TI$.

If $\tpp\in\TP\TI$, the \emph{neighbours} $\nb\TI\tpp\subseteq\TP\TI$ of $\tpp$
are defined by:
\[
  \nb\TI\tpp = \begin{cases}
    \TP\TI &\text{if } \tpp\in\TW\TI \text{ is a worker type} \\
    \TP\TI \sub \set\tpp &\text{otherwise, that is if }
    \tpIp\in\TK\TI \text{ is a king type.}
  \end{cases}
\]
Note that $\tpp \in \nb\TI\tpp$ iff $\tpp \in \TW\TI$ is a worker type.

If $\StrA$ is a $\ClSig\SigS\sms$-structure, the \emph{type instance} of $\StrA$
is:
\[
  \TIS\StrA = \setbd{\tpIab\StrA\ea\eb}{
    \ea \in \domA, \eb \in \domA \sub\set\ea}.
\]
That is $\TI = \TIS\StrA$ is the set of \twotypes/ realized in $\StrA$.
Note that this is indeed a type instance over $\ClSig\SigS\sms$:
$\TIS\StrA$ is nonempty since $\StrA$ has cardinality at least $2$
and $\TIS\StrA$ is closed under inversion by construction.
If $\TI$ is the type instance of $\StrA$, then $\StrA$ is a \emph{model} for
$\TI$.
An element realizing a king type is a \emph{king element}.
An element realizing a worker type is a \emph{worker element}.

\begin{definition}
The \emph{(finite) type realizability problem} for $\vFo2$ is the following:
given a classified signature $\ClSig\SigS\sms$ and a type instance $\TI$ over
$\ClSig\SigS\sms$, is there a (finite) model for $\TI$.
Denote the type realizability problem by $\Real{\vFo2}$ and its finite version
by $\FinReal{\vFo2}$.
\end{definition}

\begin{remark}\label{rem:red-sat-to-real}
Let $\ClSig\SigS\sms$ be a classified signature and let $\falp(\xx,\yy)$ be a
quantifier-free $\vFoF2\SigS$-formula.
Let $\TI^\falp \subseteq \TpT\SigS$ is the set of those \twotypes/ that are
consistent with $\falp(\xx,\yy)$ and the intended interpretation
for classified signatures~\cref{eq:twovar-ms-iinter}.
Then a $\ClSig\SigS\sms$-structure $\StrA$ is a classified model for
$\falp(\xx,\yy)$ iff $\TIS\StrA \subseteq \TI^\falp$.

Recall that the number of possible $1$-types or $2$-types over $\SigS$ is
exponentially bounded by the size of $\SigS$ and that the size of a $1$-type or
a $2$-type over $\SigS$ is linearly bounded by the size of $\SigS$.
Hence the (finite) classified satisfiability problem reduces to the
(finite) type realizability problem in nondeterministic exponential time:
\[
  \FinAClSat{\vFo2} \red\cNExpTime \FinAReal{\vFo2}.
\]
\end{remark}

\begin{lemma}[Model Characterization]\label{rem:type-char}
Let $\StrA$ be a model for $\TI$.
Then:
\begin{enumerate}
  \item
  If $\tpt\in\TI$
  then some $\ea\in\domA$ and $\eb\in\domA\sub\set\ea$
  have $\tpIab\StrA\ea\eb = \tpt$.
  
  If $\ea\in\domA$ and $\eb\in\domA\sub\set\ea$,
  then $\tpIab\StrA\ea\eb = \tpt$ for some $\tpt\in\TI$.
  
  Equivalently $\TI =
  \setbd{\tpIab\StrA\ea\eb}{\ea\in\domA,\eb\in\domA\sub\set\ea}$.
  \item
  If $\tpp\in\TP\TI$
  then some $\ea\in\domA$ has $\tpIa\StrA\ea = \tpp$.
  
  If $\ea\in\domA$
  then $\tpIa\StrA\ea = \tpp$ for some $\tpp\in\TP\TI$.
  
  Equivalently $\TP\TI = \setbd{\tpIa\StrA\ea}{\ea\in\domA}$.
  \item
  Let $\tpk\in\TP\TI$.
  Then $\tpk\in\TK\TI$ iff a unique $\ea\in\domA$ has $\tpIa\StrA\ea = \tpk$.
  
  \item
  Let $\tpp\in\TP\TI$.
  Then $\tpp\in\TW\TI$ iff for every $\ea\in\domA$ such that $\tpIa\StrA\ea =
  \tpp$ there is some $\eb\in\domA\sub\set\ea$ having $\tpIa\StrA\eb = \tpp$.
  
  \item
  Let $\ea\in\domA$ and let $\tpp = \tpIa\StrA\ea$.
  
  If $\tppp \in \nb\TI\tpp$ then some $\eb\in\domA\sub\set\ea$ has
  $\tpIa\StrA\eb = \tppp$.
  
  If $\eb\in\domA\sub\set\ea$ then $\tpIa\StrA\eb = \tppp$ for some
  $\tppp\in\nb\TI\tpp$.
\end{enumerate}
We will be applying this lemma implicitly.
\end{lemma}
\begin{proof}

\begin{enumerate}
  \item
  By definition $\TI = \TIS\StrA =
  \setbd{\tpIab\StrA\ea\eb}{\ea\in\domA,\eb\in\domA\sub\set\ea}$.
  
  \item
  If $\tpp\in\TP\TI$ then some $\tpt\in\TI$ has $\tpx\tpt = \tpp$,
  so some $\ea\in\domA$ and $\eb\in\domA\sub\set\ea$ has
  $\tpIab\StrA\ea\eb = \tpt$, so $\tpIa\StrA\ea = \tpp$.
  
  If $\ea\in\domA$, then note that $\StrA$ has cardinality at least $2$
  and let $\eb\in\domA\sub\set\ea$ be any other element.
  Then $\tpIab\StrA\ea\eb = \tpt \in \TI$,
  so $\tpp = \tpIa\StrA\ea = \tpx\tpt \in \TP\TI$.
  
  \item
  First let $\tpk\in\TK\TI$, so some $\ea\in\domA$ has $\tpIa\StrA\ea = \tpk$.
  Suppose towards a contradiction that
  some $\eb\in\domA\sub\set\ea$ has $\tpIa\StrA\eb = \tpk$.
  Then $\tpt = \tpIab\StrA\ea\eb \in \TI$ connects $\tpk$ with itself
  --- a contradiction.
  
  Next suppose that $\tpp\in\TP\TI\sub\TK\TI = \TW\TI$,
  so some $\tpt\in\TI$ connects $\tpk$ with itself.
  Then some $\ea\in\domA$ and $\eb\in\domA\sub\set\ea$
  have $\tpIab\StrA\ea\eb = \tpt$,
  so $\tpIa\StrA\ea = \tpIa\StrA\eb = \tpp$,
  so there is not a unique $\ea\in\domA$ having $\tpIa\StrA\ea = \tpp$.
  
  \item
  First suppose that $\tpp\in\TW\TI$ and that $\ea\in\domA$
  has $\tpIa\StrA\ea = \tpp$.
  Since $\tpp\not\in\TK\TI$, such $\ea$ is not unique, so there is some
  $\eb\in\domA\sub\set\ea$ having $\tpIa\StrA\eb = \tpp$.
  
  Next suppose that $\tpp\in\TP\TI$ and that for every $\ea\in\domA$ such that
  $\tpIa\StrA\ea = \tpp$ there is some $\eb\in\domA\sub\set\ea$ having
  $\tpIa\StrA\eb = \tpp$.
  Since $\tpp\in\TP\TI$, some $\ea\in\domA$ has $\tpIa\StrA\ea = \tpp$, so some
  $\eb\in\domA\sub\set\ea$ has $\tpIa\StrA\eb = \tpp$, so there is not a unique
  $\ea\in\domA$ having $\tpIa\StrA\ea = \tpp$, so $\tpp\not\in\TK\TI$, so
  $\tpp\in\TW\TI$.
  \item
  Let $\ea\in\domA$ and $\tpp = \tpIa\StrA\ea$.
  
  First let $\tppp\in\nb\TI\tpp$.
  If $\tppp \neq \tpp$, then some $\eb \in \domA$ has
  $\tpIa\StrA\eb = \tppp$, so $\eb \in \domA\sub\set\ea$.
  If $\tppp = \tpp$, then $\tpp \in \TW\TI$ is a worker type,
  so some $\eb\in\domA\sub\set\ea$ has
  $\tpIa\StrA\eb = \tpp = \tppp$.
  
  Next suppose that some $\eb\in\domA\sub\set\ea$ has $\tpIa\StrA\eb = \tppp$.
  If $\tppp \neq \tpp$, then $\tppp \in \nb\TI\tpp$.
  If $\tppp = \tpp$, then $\tpp$ must be a worker type,
  so $\tppp \in \nb\TI\tpp = \TP\TI$.
\end{enumerate}
\end{proof}

\begin{definition}
Let $\TI$ be a type instance over $\ClSig\SigS\sms$.
A \emph{star-type} $\stps \subseteq \TI$ over $\TI$ is a nonempty
set of \twotypes/ satisfying the following conditions:
\begin{itemize}
  \item[\condstpx]\label{cond:stpx}
  If $\tpt, \tptp \in \stps$, then $\tpx\tpt = \tpx\tptp$.
  Denote $\tpx\tpt$ for any $\tpt \in \stps$ by $\tpp=\tpx\stps$.
  
  \item[\condstppy]\label{cond:stppy}
  If $\tppp \in \nb\TI\tpp$, then some $\tpt\in\stps$ has
  $\tpy\tpt = \tppp$.
  
  \item[\condstpky]\label{cond:stpky}
  If $\tpkp \in \nb\TI\tpp\cap\TK\TI$
  and if $\tpt,\tptp \in \stps$ have $\tpy\tpt = \tpy\tptp = \tpkp$,
  then $\tpt = \tptp$.
  
  \item[\condstpm]\label{cond:stpm}
  If $\sm \in \sms$, then some $\tpt\in\stps$ has $\sm(\xx,\yy) \in \tpTt$.
\end{itemize}
\end{definition}
A star-type $\stps$ is a \emph{king star-type} if $\tpx\stps$ is a king type.
Otherwise the star-type is a \emph{worker star-type}.
Note that the size of a star-type is linear with respect to the size of the type
instance.
\begin{remark}
If $\stps$ is a star-type over $\TI$, then:
\begin{itemize}
  \item[\condstpkyu]\label{cond:stpkyu}
  If $\tpkp \in \nb\TI\tpp\cap\TK\TI$,
  then a unique $\tpt\in\stps$ has $\tpy\tpt = \tpk$.
\end{itemize}
\end{remark}
\begin{proof}
By~\refcondstppy{}, some $\tpt\in\stps$ has $\tpy\tpt = \tpk$.
By~\refcondstpky{}, such $\tpt$ is unique.
\end{proof}

\begin{definition}
Let $\StrA$ be a model for $\TI$
and let $\ea\in\domA$.
The \emph{star-type} $\stpIa\StrA\ea$ of $\ea$ is defined by:
\[
  \stpIa\StrA\ea = \setbd{\tpIab\StrA\ea\eb}{\eb \in \domA \sub \set\ea}.
\]
\end{definition}
\begin{remark}
Indeed $\stps = \stpIa\StrA\ea$ is a star-type over $\TI$.
\end{remark}
\begin{proof}
The set $\stps$ is nonempty since $\StrA$ has cardinality at least $2$.
We check the conditions for a star-type $\stps$ over $\TI$:
\begin{itemize}
  \item[\refcondstpx]
  If $\tpt,\tptp\in\stps$,
  then $\tpt=\tpIab\StrA\ea\eb$ and $\tptp=\tpIab\StrA\ea\ebp$ for some
  $\eb,\ebp\in\domA\sub\set{\ea}$.
  Then $\tpx\tpt = \tpx\tptp = \tpIa\StrA\ea$.
  
  Let $\tpp = \tpx\stps = \tpIa\StrA\ea$.
  \item[\refcondstppy]
  If $\tppp\in\nb\TI\tpp$,
  then some $\eb\in\domA\sub\set\ea$ has $\tpIa\StrA\eb = \tppp$, so
  $\tpt = \tpIab\StrA\ea\eb \in \stps$ has $\tpy\tpt = \tppp$.
  \item[\refcondstpky]
  If $\tpkp\in\nb\TI\tpp\cap\TK\TI$, then $\tpkp\neq\tpp$.
  Suppose towards a contradiction that some $\tpt\neq\tptp\in\stps$ have
  $\tpy\tpt=\tpy\tptp=\tpkp$.
  Then $\tpt=\tpIab\StrA\ea\eb$ and $\tptp=\tpIab\StrA\ea\ebp$ for some
  $\eb\neq\ebp\in\domA\sub\set\ea$.
  Then $\tpIa\StrA\eb=\tpIa\StrA\ebp=\tpkp$ --- a contradiction.
  \item[\refcondstpm]
  Let $\sm\in\sms$.
  Since $\StrA$ is a $\ClSig\SigS\sms$-structure, some $\eb\in\domA\sub\set\ea$
  has $\sm(\xx,\yy) \in \tpIab\StrA\ea\eb \in \stps$.
\end{itemize}
\end{proof}

\begin{definition}
Let $\TI$ be a type instance over $\ClSig\SigS\sms$.
A \emph{certificate} $\Cert$ for $\TI$ is a nonempty set of star-types over
$\TI$ satisfying the following conditions:
\begin{itemize}
  \item[\condcertT]\label{cond:certT}
  If $\tpt\in\TI$, then some $\stps\in\Cert$ has $\tpt\in\stps$,
  that is there is a star-type containing each \twotype/.
  \item[\condcertk]\label{cond:certk}
  If $\tpk\in\TK\TI$ and if $\stps,\stpsp\in\Cert$
  have $\tpx\stps = \tpx\stpsp = \tpk$, then $\stps = \stpsp$.
\end{itemize}
\end{definition}
\begin{remark}
Let $\Cert$ be a certificate over $\TI$. Then:
\begin{itemize}
  \item[\condcertp]\label{cond:certp}
  If $\tpp \in \TP\TI$, then some $\stps\in\Cert$ has $\tpx\stps = \tpp$.
  \item[\condcertku]\label{cond:certku}
  If $\tpk \in \TK\TI$, then a unique $\stps\in\Cert$ has $\tpx\stps = \tpp$.
\end{itemize}
\end{remark}
\begin{proof}
\begin{itemize}
  \item[\refcondcertp]
  If $\tpp\in\TP\TI$, then some $\tpt\in\TI$ has $\tpx\tpt = \tpp$,
  so by~\refcondcertT{} some $\stps\in\Cert$ has $\tpt\in\stps$, so
  $\tpx\stps = \tpx\tpt = \tpp$.
  \item[\refcondcertku]
  If $\tpk\in\TK\TI$, then by~\refcondcertp{} some $\stps\in\Cert$ has
  $\tpx\stps = \tpk$.
  By~\refcondcertk, such $\stps$ is unique.
\end{itemize}
\end{proof}

Note that the size of a certificate may be exponential with respect to the size
of the type instance.
However, we may extract polynomial certificates:
\begin{lemma}[Certificate extraction]\label{lem:cert-extract}
Let $\StrA$ be a model for the type instance $\TI$.
For each \twotype/ $\tpt\in\TI$,
let $\eat\tpt\in\domA$ and $\ebt\tpt\in\domA\sub\set{\eat\tpt}$
have $\tpIab\StrA{\eat\tpt}{\ebt\tpt} = \tpt$.
Let 
\[
  \Cert = \setbd{\stpIa\StrA{\eat\tpt}}{\tpt \in \TI}.
\]
Then $\Cert$ is a certificate for $\TI$.
The size of $\Cert$ is quadratic with respect to the size of $\TI$.
\end{lemma}
\begin{proof}
Since $\TI$ is nonempty, $\Cert$ is nonempty.
We check the conditions for $\Cert$ to be a certificate for $\TI$:
\begin{itemize}
  \item[\refcondcertT]
  If $\tpt\in\TI$,
  then $\tpt=\tpIab\StrA{\eat\tpt}{\ebt\tpt} \in \stpIa\StrA{\eat\tpt} \in
  \Cert$.
  \item[\refcondcertk]
  Let $\tpk\in\TK\TI$ and let $\stps,\stpsp \in \Cert$ have
  $\tpx\stps = \tpx\stpsp = \tpk$.
  Then $\stps = \stpIa\StrA{\eat\tpt}$ and $\stpsp = \stpIa\StrA{\eat\tptp}$ for
  some $\tpt,\tptp\in\TI$.
  Then $\tpx\tpt = \tpx\tptp = \tpk$.
  Then $\tpIa\StrA{\eat\tpt} = \tpx\tpk = \tpIa\StrA{\eat\tptp}$.
  Since $\tpk$ is a king type, $\eat\tpt = \eat\tptp$, so
  $\stps = \stpsp$.
\end{itemize}
\end{proof}

\begin{theorem}[Certificate expansion]\label{lem:cert-expand}
Let $\Cert$ be a certificate for the type instance $\TI$ over the classified
signature $\ClSig\SigS\sms$.
Then $\TI$ has a finite model.
More precisely, let $\pt \geq \card\TI$ be a parameter.
Then $\TI$ has a finite model in which each worker type is realized at least
$\pt$ times.
\end{theorem}
\begin{proof}
We adapt the standard strategy\footnote{with the slight difference that our
approach doesn't need \emph{a court}, since the information about it is implicit
in the certificate} used in the proof of the finite model property for the logic
$\vFo2$, as presented in~\cite{gradel1999logics}.
We build a model $\StrA$ for $\TI$ as follows.
The domain $\domA$ of $\StrA$ is the union of the following disjoint sets of
elements:
\begin{itemize}
  \item
  The singleton set $\As\stps = \set{\as\stps}$ for every king star-type
  $\stps\in\Cert$, $\xtp\stps \in \TK\TI$.
  The elements $\as\stps$ are the \emph{kings}.
  \item 
  The three disjoint copies of $\pt$ elements
  $\As\stps = \Asi\stps0 \cup \Asi\stps1 \cup \Asi\stps2$ for every
  worker star-type $\stps \in \Cert$, $\xtp\stps \in \TW\TI$,
  where $\Asi\stps\ii = \set{\asij\stps\ii1, \asij\stps\ii2, \dots,
  \asij\stps\ii\pt}$ for $\ii \in \set{0,1,2}$.
  The elements $\asij\stps\ii\jj$ are the \emph{workers}.
\end{itemize}
Let $\itpsOP : \domA \to \Cert$ denote the intended star-type of the elements:
$\itps\ea = \stps$ on $\As\stps$.
Let $\itpiOP : \domA \to \TP\TI$ denote the intended \onetype/ of the elements:
$\itpi\ea = \tpx(\stps(\ea))$.
We consistently assign \twotypes/ between distinct elements on stages.
\begin{description}
  \item[Realization of kings]
  We first assign \twotypes/ consistently between the kings and any other
  element.
  Let $\ea\in\domA$ be any king,
  so $\ea = \as\stpsp$ for some king star-type $\stpsp\in\Cert$.
  Let $\tpkp = \tpp(\ea) = \tpx\stpsp \in \TK\TI$
  be the intended (king) type of $\ea$.
  Let $\eb \in \domA \sub \set\ea$ be any other element and let
  $\stps = \stps(\eb)$ and $\tpp = \tpp(\eb) = \tpx\stps$ be its intended
  star-type and \onetype/, respectively.
  Since $\domA$ contains a unique element for each king star-type,
  $\stps\neq\stpsp$.
  By~\refcondcertk, $\tpp \neq \tpkp$,
  so $\tpkp \in \nb\TI\tpp \cap \TK\TI$.
  By~\refcondstpkyu, a unique $\tpt\in\stps$ has $\tpy\tpt = \tpkp$.
  We assign $\tpIab\StrA\eb\ea = \tpt$.
  We must check that these assignments are consistent.
  
  First, these assignments are symmetric over the kings.
  Suppose that $\eb$ is a king, so $\tpp = \tpk$ is a king type.
  Since $\tpkp \neq \tpk$, $\tpk \in \nb\TI\tpkp \cap \TK\TI$.
  By~\refcondstpkyu, a unique $\tptp\in\stpsp$ has $\tpy\tptp = \tpk$ and we
  would want to assign $\tpIab\StrA\ea\eb = \tptp$.
  We claim that $\tptp = \inv\tpt$.
  Indeed, by~\refcondcertT, $\inv\tpt \in \stpspp$ for some $\stpspp\in\Cert$.
  Then $\tpx\stpspp = \tpx{(\inv\tpt)} = \tpk$, so by~\refcondcertk, $\stpspp =
  \stps$.
  Then $\inv\tpt \in \stps$ has $\tpy{\inv\tpt} = \tpk$, so $\inv\tpt = \tptp$.
  
  Next, these assignments cover $\stpsp$.
  Let $\tptp\in\stpsp$ be any.
  Then by~\refcondcertT, some $\stps\in\Cert$ has $\tpt = \inv\tptp\in\stps$.
  If $\stps = \stpsp$, then $\tpy\tptp = \tpx{\tpt} = \tpx\stps = \tpk$,
  so $\tptp$ would connect $\tpk$ with itself --- a contradiction.
  So $\stps \neq \stpsp$.
  By~\refcondcertk, $\tpp \neq \tpk$, so $\tpk \in \nb\TI\tpp$.
  Then by~\refcondstpkyu, $\tpt\in\stps$ is the unique having $\tpy\tpt = \tpk$.
  Since $\StrA$ contains some element for each star-type,
  some $\eb\in\domA\sub\set\ea$ has $\stps(\eb) = \stps$, so we had assigned
  $\tpIab\StrA\eb\ea = \tpt$.
  
  \item[Realization of workers]
  Next we consistently assign \twotypes/ between workers.
  Let $\ea\in\domA$ be any worker and let $\stps=\stps(\ea)$ and
  $\tpp=\tpp(\ea)$ be its intended star-type and \onetype/, respectively.
  Then $\ea = \asij\stps\ii\jj$ for some $\ii\in\set{0,1,2}$ and
  $\jj\in[1,\pt]$.
  Let $\iip=(\ii+1\bmod3)\in\set{0,1,2}$ be the index of \emph{the next copy} of
  the workers.
  Let $\tpt\in\stps$ be any \twotype/.
  
  First suppose that $\tpy\tpt = \tpkp \in \TK\TI$ is a king type.
  By~\refcondcertku, let $\stpsp\in\Cert$ be the unique star-type having
  $\tpx\stpsp = \tpkp$ and let $\eb = \as\stpsp$ be the unique king having
  $\tpp(\eb) = \tpkp$.
  Since $\tpkp\neq\tpp$, $\tpkp \in \nb\TI\tpp$ and by~\refcondstpkyu,
  $\tpt\in\stps$ is the unique having $\tpy\tpt = \tpkp$.
  So we had already assigned $\tpIab\StrA\ea\eb = \tpt$ during the realization
  of kings.
  
  Next suppose that $\tpy\tpt = \tppp \in \TW\TI$ is a worker type.
  Let $\TU = \setbd{\tpu\in\stps}{\tpy\tpu = \tppp}$ be the set of all
  \twotypes/ from $\stps$ parallel to $\tpt$.
  We simultaneously find distinct elements $\ebt\tpu$ that are distinct from
  $\ea$ for the assignments $\tpIab\StrA\ea{\ebt\tpu} = \tpu$.
  By~\refcondcertT{}, for each
  $\tpu\in\TU$ there is some star-type $\stpspt\tpu \in \Cert$ such that
  $\inv\tpu \in \stpspt\tpu$.
  Note that $\tpx{\stpspt\tpu} = \tppp$ is a worker type.
  Since $\TU\subseteq\TI$ we have $\card{\TU} \leq \pt$, so there are enough
  distinct workers from the next copy $\ebt\tpu \in \Asi{\stpspt\tpu}\iip$ for
  the assignments $\tpIab\StrA\ea{\ebt\tpu} = \tpu$.
  These assignments do not clash with each other, since they are made between
  \emph{consecutive copies} of worker elements.
  \item[Completion]
  Suppose that $\ea\neq\eb \in \domA$ are any two distinct elements such that
  $\tpIab\StrA\ea\eb$ has not yet been assigned. 
  Then both $\tpp(\ea)$ and $\tpp(\eb)$ are worker types, so $\tpp(\eb) \in
  \nb\TI{\tpp(\ea)} = \TP\TI$.
  By~\refcondstppy, some $\tpt\in\stps(\ea)$ has $\tpy\tpt = \tpp(\eb)$, so we
  may assign $\tpIab\StrA\ea\eb = \tpt$.
  Note that this may extend the actual star-type of $\ea$ and $\eb$, but this is
  appropriate.
\end{description}
The structure $\StrA$ is a $\ClSig\SigS\sms$-structure by~\refcondstpm{} and is
a model for $\TI$ by~\refcondcertT.
\end{proof}

\begin{proposition}\label{prop:tprealiz-1}
The type realizability problem for $\vFo2$ coincides with the finite type
realizability problem and is in $\cNP$.
\end{proposition}
\begin{proof}
Let $\TI$ be a type instance for the classified signature
$\ClSig\SigS\sms$.
Guess a polynomial certificate for $\TI$.
By~\Cref{lem:cert-extract} and~\Cref{lem:cert-expand}, such a certificate exists
iff $\TI$ has a model.
The general version coincides with the finite version since the model
constructed in~\Cref{lem:cert-expand} is finite.
\end{proof}
\begin{corollary}[\cite{gradel1997decision}]
The logic $\vFo2$ has the finite model property and its (finite) satisfiability
problem is in $\cNExpTime$.
\end{corollary}