% Monadic first-order cells

Let $\szu, \sze \in \Nats$, $\sze \geq 1$ and $\SigS = \ueSigS\szu\sze =
\seq{\suu1,\suu2,\dots,\suu\szu,\see1,\see2,\dots,\see\sze}$
be a predicate signature.
Abbreviate the finest equivalence symbol $\sd = \see1$.

\begin{definition}
Define the quantifier-free $\vFoF2\SigS$-formula $\gls{fcell-S-x-y}$ by:
\[
  \fcell\SigS(\xx,\yy) = \sd(\xx,\yy) \land
  \bigwedge_{1 \leq i \leq \szu} (\suu\ii(\xx) \lequ \suu\ii(\yy)).
\]
If $\StrA$ is a $\SigS$-structure and $\relD = \at\StrA\sd$, then the
interpretation $\relC = \at\StrA{\fcell\SigS} \subseteq \domAA$ is an
equivalence relation on $\domA$ that refines $\relD$.
The \emph{cells} of $\StrA$ are the equivalence classes of $\relC$.
That is, a cell is a maximal set of $\relD$-equivalent elements satisfying the
same $\su$-predicates.
\end{definition}

\begin{remark}\label{rem:monadic-same-cell-r}
Let $\StrA$ be a $\SigS$-structure, $\nr \in \Nats$,
$\many\ea = \eaa1\eaa2\dots\eaa\nr \in \domA^\nr$,
$\many\eb = \ebb1\ebb2\dots\ebb\nr \in \domA^\nr$ and $\eaa\ii$ and $\ebb\ii$
are in the same $\StrA$-cell for all $\ii \in [1,\nr]$.
Suppose that $\eaa\ii = \eaa\jj$ iff $\ebb\ii = \ebb\jj$ for all $\ii, \jj \in
[1,\nr]$.
Then $\many\ea \mapsto \many\eb$ is a partial isomorphism.
\end{remark}
\begin{proof}
Direct consequence of the fact that the cell equivalence relation refines the
finest equivalence relation $\relD$ and that the elements in the same cell
satisfy the same $\su$-predicates.
The equality condition ensures that the mapping is a bijection.
\end{proof}

\begin{lemma}\label{lem:monadic-cell}
Let $\StrA$ be a $\SigS$-structure and $\nr \in \pNats$.
There is $\StrB \subseteq \StrA$ such that
$\StrB \requiv\nr \StrA$ and every $\StrB$-cell has cardinality at most $\nr$.
\end{lemma}
\begin{proof}
Let $\relC \subseteq \domAA$ be the $\StrA$-cell equivalence relation.
Execute the following process: for every $\StrA$-cell, if it has cardinality
less than $\nr$, select all elements from that cell;
otherwise select $\nr$ distinct elements from that cell.
Let $\domB \subseteq \domA$ be the set of selected elements and let
$\StrB = \StrA \restriction \domB$.
By construction, every $\StrB$-cell has cardinality at most $\nr$.
We claim that $\StrA \requiv\nr \StrB$.
Let $\selh = \relC \cap (\domA \times \domB)$ relates elements from $\domA$ with
elements from $\domB$ in the same cell.
Note that for all $\ea \in \domA$:
\begin{equation}\label{eq:monadic-cell-little}
  \card{\selh[\ea]} = \min(\card{\relC[\ea]}, \nr).
\end{equation}
For $\ii\in[0,\nr]$ let $\pisoII\ii$ be the set of partial isomorphisms from
$\StrA$ to $\StrB$ that have length $\ii$ and that are included in $\selh$.
The set $\pisoII0$ is nonempty since it contains the empty partial isomorphism.
We claim that the sequence $\pisoII0, \pisoII1, \dots, \pisoII\nr$
satisfies the back-and-forth conditions of \Cref{thm:game-ef}.
Let $\ii \in [0,\nr-1]$ and let 
\[
  \pisop = \many\ea \mapsto \many\eb =
  \eaa1\eaa2\dots\eaa\ii \mapsto \ebb1\ebb2\dots\ebb\ii
  \in
  \pisoII\ii
\]
be any partial isomorphism.
\begin{enumerate}
  \item For the forth condition, let $\ea \in \domA$.
  We have to find some $\eb \in \domB$ such that
  $\many\ea\ea \mapsto \many\eb\eb \in \pisoII{\ii+1}$.
  If $\ea = \eaa\kk$ for some $\kk \in [1,\ii]$, then $\eb = \ebb\kk$ is
  appropriate.
  
  Suppose that $\ea \neq \eaa\kk$ for all $\kk \in [1,\ii]$.
  Let $\setS \subseteq \relC[\ea]$ be the set of $\many\ea$-elements in the same
  $\StrA$-cell as $\ea$:
  \[
    \setS = \setbd{\eaa\kk \in \relC[\ea]}{\kk \in [1,\ii]}.
  \]
  Note that $\card\setS \leq \nr-1$ and
  $\card{\relC[\ea]} \geq \card{\setS} + 1$.
  By \cref{eq:monadic-cell-little}, $\card{\selh[\ea]} \geq \card\setS + 1$.
  Hence there is an element $\eb \in \selh[\ea]$ that is distinct from
  $\ebb\kk$ for all $\kk \in [1,\ii]$ and this $\eb$ is appropriate.

  \item For the back condition, let $\eb \in \domB$.
  We have to find some $\ea \in \domA$ such that
  $\many\ea\ea \mapsto \many\eb\eb \in \pisoII{\ii+1}$.
  If $\eb = \ebb\kk$ for some $\kk \in [1,\ii]$, then $\ea = \eaa\kk$ is
  appropriate.
  
  Suppose that $\eb \neq \ebb\kk$ for all $\kk \in [1,\ii]$. Since $\eb \in
  \selh[\eb]$, $\ea = \eb$ is appropriate.
\end{enumerate}
By \Cref{thm:game-ef}, $\StrA \requiv\nr \StrB$.
\end{proof}