% Monadic first-order organs

Let $\szu, \sze \in \Nats$, $\sze \geq 2$ and
$\SigS = \SigS(\szu, \sze) = 
\seq{\suu1,\suu2,\dots,\suu\szu,\see1,\see2,\dots,\see\sze}$
be a predicate signature.
Abbreviate the finest two equivalence symbols $\sd = \see1$ and $\se = \see2$.
\begin{definition}
Let $\StrA$ be a $\SigS$-structure and let $\relD = \at\StrA\sd$ and
$\relE = \at\StrA\se$.
Recall that the set of $\relD$-classes is $\Ecl\relD$.
Two $D$-classes $\eclX, \eclY \in \Ecl\relD$ are \emph{organ-equivalent}
if they are included in the same $E$-class (equivalently $X \times Y \subseteq
E$), and the induced substructures $(\StrA \restriction \eclX)$ and $(\StrA
\restriction \eclY)$ are isomorphic.
The organ-equivalence relation is
$\gls{organ-equivalence} \subseteq \domXX{\Ecl\relD}$.
Since $\relD$ refines $\relE$, organ-equivalence is an equivalence relation
on $\Ecl\relD$.
An \emph{organ} is an organ-equivalence-class. That is, an organ
is a maximal set of isomorphic $\relD$-classes, included in the same
$\relE$-class.

For any two organ-equivalent $\relD$-classes $(\eclX, \eclY) \in \Org$,
fix an isomorphism
\[
  \isoh\eclX\eclY : (\StrA \restriction \eclX) \bto (\StrA \restriction \eclY)
\]
consistently, so that $\isoh\eclX\eclX = \id\eclX$,
$\isoh\eclY\eclX = \inv{\isoh\eclX\eclY}$
and if $(\eclY, \eclZ) \in \Org$ then
$\isoh\eclX\eclZ = \isoh\eclY\eclZ \comp \isoh\eclX\eclY$.
Two elements $\ea, \eb \in \domA$ are \emph{sub-organ-equivalent}
if $(\relD[\ea], \relD[\eb]) \in \Org$ and
$\isoh{\relD[\ea]}{\relD[\eb]}(\ea) = \eb$.
Since the isomorphisms $\isoh\eclX\eclY$ are chosen consistently,
sub-organ-equivalence 
$\gls{sub-organ-equivalence} \subseteq \domAA$ is an equivalence relation on
$\domA$ that refines $\relE$.
\end{definition}

\begin{remark}\label{rem:monadic-same-organ-iso}
Let $\StrA$ be a $\SigS$-structure, $\nr \in \Nats$,
$\many\ea = \eaa1\eaa2\dots\eaa\nr \in \domA^\nr$,
$\many\eb = \ebb1\ebb2\dots\ebb\nr \in \domA^\nr$,
$\eaa\ii$ and $\ebb\ii$ are sub-organ-equivalent for all $\ii \in [1,\nr]$.
Suppose that
$\StrA \vDash \sd(\eaa\ii, \eaa\jj)$ iff $\StrA \vDash \sd(\ebb\ii, \ebb\jj)$
for all $\ii, \jj \in [1,\nr]$.
Then $\many\ea \mapsto \many\eb$ is a partial isomorphism.
\end{remark}
\begin{proof}
The condition about the finest equivalence symbol $\sd$ ensures that the
interpretation of $\sd$ is preserved.
Since sub-organ-equivalence relates isomorphic elements, the interpretation of
the unary symbols and the formal equality is preserved.
Since the sub-organ-equivalence $\sOrg \subseteq \domAA$ refines the
second finest equivalence relation $\relE$, the interpretation of all remaining
equivalence symbols $\see\jj$ is preserved.
\end{proof}

\begin{lemma}\label{lem:monadic-organ}
Let $\StrA$ be a $\SigS$-structure and $\nr \in \pNats$.
There is $\StrB \subseteq \StrA$ such that $\StrB \requiv\nr \StrA$ and every
$\StrB$-organ has cardinality at most $\nr$.
\end{lemma}
\begin{proof}
Let $\relD = \at\StrA\sd$, $\relE = \at\StrA\se$ and
let $\EclA = \Ecl\relD$ be the set of $\relD$-classes.
Let $\Org \subseteq \domXX\EclA$ be the $\StrA$-organ-equivalence relation
on $\EclA$.
Execute the following process: for every $\StrA$-organ, if it has cardinality at
most $\nr$, select all $\relD$-classes from that organ; otherwise select $\nr$
distinct $\relD$-classes from that organ (note that these will be isomorphic).
Let $\EclB \subseteq \EclA$ be the set of selected $\relD$-classes.
Let $\domB = \cup \EclB \subseteq \domA$ be the set of elements in the selected
classes and let $\StrB = (\StrA \restriction \domB)$.
By construction, every $\StrB$-organ has cardinality at most $\nr$.
We claim that $\StrA \requiv\nr \StrB$.
Let $\selH = \Org \cap (\EclA\cprod\EclB)$ relates the $\relD$-classes with the
isomorphic $\relD$-classes from $\EclB$ in the same organ.
Let $\selh$ relates the elements of $\domA$ with their isomorphic elements from
$\domB$.
Note that for all elements $\ea \in \domA$:
\begin{equation}\label{eq:monadic-organ}
  \card{\selh[\ea]} = \min(\card{\Org[\relD[\ea]]}, \nr).
\end{equation}
For $\ii\in[0,\nr]$ let $\pisoII\ii$ be the set of partial isomorphisms from
$\StrA$ to $\StrB$ that have length $\ii$ and that are included in $\selh$.
The set $\pisoII0$ is nonempty since it contains the empty partial isomorphism.
We claim that the sequence $\pisoII0,\pisoII1,\dots,\pisoII\nr$ satisfies the
back-and-forth conditions of \Cref{thm:game-ef}.
Let $\ii \in [0,\nr-1]$ and let 
\[
  \pisop = \many\ea \mapsto \many\eb =
  \eaa1\eaa2\dots\eaa\ii \mapsto \ebb1\ebb2\dots\ebb\ii \in \pisoI
\]
be any partial isomorphism.
\begin{enumerate}
  \item For the forth condition, let $\ea \in \domA$.
  We have to find some $\eb \in \domB$ such that
  $\many\ea\ea \mapsto \many\eb\eb \in \pisoII{\ii+1}$.
  If $\ea \in \relD[\eaa\kk]$ for some $\kk \in [1,\ii]$,
  then $\eb = \isoh{\relD[\eaa\kk]}{\relD[\ebb\kk]}(\ea)$ is appropriate.
  
  Suppose $\ea \not\in \relD[\eaa\kk]$ for all $\kk \in [1,\ii]$.
  Let $\EclS \subseteq \Org[\relD[\ea]]$ be the set of $\relD$-classes of
  $\many\ea$-elements in the same $\StrA$-organ as $\relD[\ea]$:
  \[
    \EclS = \setbd{\relD[\eaa\kk] \in \Org[\relD[\ea]]}{\kk \in [1,\ii]}.
  \]
  Note that $\card\EclS \leq \nr-1$ and
  $\card{\Org[\relD[\ea]]} \geq \card\EclS + 1$.
  By \cref{eq:monadic-organ}, $\card{\selh[\ea]} \geq \card\EclS + 1$.
  Hence there is some $\eb \in \selh[\ea]$ such that $\eb \not\in
  \relD[\ebb\kk]$ for all $\kk \in [1,\ii]$. This $\eb$ is appropriate.
  \item For the back condition, let $\eb \in \domB$.
  We have to find some $\ea \in \domA$ such that
  $\many\ea\ea \mapsto \many\eb\eb \in \pisoI$.
  If $\eb \in \relD[\ebb\kk]$ for some $\kk \in [1,\ii]$,
  then $\ea = \isoh{\relD[\ebb\kk]}{\relD[\eaa\kk]}(\eb)$ is appropriate.
  
  Suppose that $\eb \not\in \relD[\ebb\kk]$ for all $\kk \in [1,\ii]$.
  Since $\eb \in \selh[\eb]$, $\ea = \eb$ is appropriate.
\end{enumerate}
By \Cref{thm:game-ef}, $\StrA \requiv\nr \StrB$.
\end{proof}