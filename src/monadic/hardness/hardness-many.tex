% Hardness with many equivalences in refinement
The argument from the previous section can be iterated to yield the hardness of
the (finite) satisfiability of the monadic first-order logic with several
builtin equivalence symbols in refinement $\Lvp\Fo\nonv1\Eea\sze\agrefine$.
Our strategy is to encode $(\sze+1)$-exponential numbers at every equivalence
class of the coarsest relation by thinking of the $\sze$-exponential numbers at
the classes of the second-to-coarsest relation as bit positions.

 For $\sze \in \pNats$, consider the predicate
signature $\ES(\sze) = \seq{\see1,\see2,\dots,\see\sze}$ consisting of the builtin equivalence
symbols $\see\ii$ in refinement. Abbreviate the \emph{coarsest} equivalence 
symbol $\sd = \see\sze$.
\begin{definition}
Let $\sze \in \pNats$. An \emph{$\sze$-exponential setup} is a uniform
effective polynomial-time process for creating the following data structure.
For every $\szu \in \pNats$, there is a predicate signature $\DS(\sze,\szu)$
having length polynomial in $\szu$, consisting of unary predicate symbols and
containing $\ES(\sze)$. The following data is effectively defined:
\begin{enumerate}[label=E\arabic*]
  \item\label{eset:1} There is a $\vFoF3{\DS(\sze,\szu)}$-sentence
  $\fposfull{\DS(\sze,\szu)}$, whose length grows polynomially as $\szu$ grows.
  \item\label{eset:2} If $\StrA$ is a $\DS(\sze,\szu)$-structure,
  $\StrA \vDash \fposfull{\DS(\sze,\szu)}$ and $\relD = \at\StrA\sd$, then there
  is a function
  $\Data{\DS(\sze,\szu)}\StrA : \Ecl\relD \to \tBitstrings{\tetr\sze2\szu}$
  that assigns an $\sze$-exponential bitstring to every $\relD$-class.
  \item\label{eset:3} There is a $\vFoF3{\DS(\sze,\szu)}$-formula
  $\fEq{\DS(\sze,\szu)}(\xx,\yy)$ whose length grows polynomially as $\szu$
  grows, such that for all $\ea,\eb \in \domA$:
  \[
    \miff{
      \StrA \vDash \fEq{\DS(\sze,\szu)}(\ea,\eb)}{
      \bdec{\Data{\DS(\sze,\szu)}\StrA\relD[\ea]} =
      \bdec{\Data{\DS(\sze,\szu)}\StrA\relD[\eb]}}.
  \]
  \item\label{eset:4} There is a $\vFoF3{\DS(\sze,\szu)}$-formula
  $\fZero{\DS(\sze,\szu)}(\xx)$, whose length grows polynomially as $\szu$
  grows, such that for all $\ea \in \domA$:
  \[
    \miff{
      \StrA \vDash \fZero{\DS(\sze,\szu)}(\ea)}{
      \bdec{\Data{\DS(\sze,\szu)}\StrA\relD[\ea]} = 0}.
  \]
  \item\label{eset:5} There is a $\vFoF3{\DS(\sze,\szu)}$-formula
  $\fLargest{\DS(\sze,\szu)}(\xx)$, whose length grows polynomially as $\szu$ grows,
  such that for all $\ea \in \domA$:
  \[
    \miff{
      \StrA \vDash \fLargest{\DS(\sze,\szu)}(\ea)}{
      \bdec{\Data{\DS(\sze,\szu)}\StrA\relD[\ea]} = \largtbit{\tetr\sze2\szu} =
      \tetr{\sze+1}2\szu-1}.
  \]
  \item\label{eset:6} There is a $\vFoF3{\DS(\sze,\szu)}$-formula
  $\fLess{\DS(\sze,\szu)}(\xx,\yy)$, whose length grows polynomially as $\szu$
  grows, such that for all $\ea, \eb \in \domA$:
  \[
    \miff{
      \StrA \vDash \fLess{\DS(\sze,\szu)}(\ea,\eb)}{
      \bdec{\Data{\DS(\sze,\szu)}\StrA\relD[\ea]} <
      \bdec{\Data{\DS(\sze,\szu)}\StrA\relD[\eb]}}.
  \]
  \item\label{eset:7} There is a $\vFoF3{\DS(\sze,\szu)}$-formula
  $\fSucc{\DS(\sze,\szu)}(\xx,\yy)$, whose length grows polynomially as $\szu$
  grows, such that for all $\ea, \eb \in \domA$:
  \[
    \miff{
      \StrA \vDash \fSucc{\DS(\sze,\szu)}(\ea,\eb)}{
      \bdec{\Data{\DS(\sze,\szu)}\StrA\relD[\eb]} =
      \bdec{\Data{\DS(\sze,\szu)}\StrA\relD[\ea]} + 1}.
  \]
  \item\label{eset:8} For every $\tetr\sze2\szu$-bit number $\nM$, there is a
  $\vFoF3{\DS(\sze,\szu)}$-formula
  $\fDataA{\DS(\sze,\szu)}\nM(\xx)$, whose length grows polynomially as $\szu$
  and $\nM$ grow, such that for all $\ea \in  \domA$:
  \[
    \miff{
      \StrA \vDash \fDataA{\DS(\sze,\szu)}\nM(\ea)}{
      \bdec{\Data{\DS(\sze,\szu)}\StrA\relD[\ea]} = \nM}.
  \]
\end{enumerate}
\end{definition}
The previous section defines a $1$-exponential setup.
Suppose that we have an $\sze$-exponential setup having predicate signature
$\DS = \DS(\sze,\szu)$. Analogously to the previous section, we will describe an
$(\sze+1)$-exponential setup $\DSp = \DS(\sze+1,\szu) = \DS + \seq{\se}$ which
is based on $\DS$, where $\se = \see{\sze+1}$ is the new coarsest builtin
equivalence symbol in $\DSp$. Define the following formulas:
\begin{align}
  &\fposeq\DSp(\xx,\yy) = \fEq\DS(\xx,\yy) \nonumber \\
  &\fbitO\DSp(\xx) = \forall\yy(\se(\yy,\xx) \land 
    \fposeq\DSp(\yy,\xx) \limp \sd(\yy, \xx)) \nonumber \\
  &\fbitI\DSp(\xx) = \exists\yy(\se(\yy,\xx) \land
    \fposeq\DSp(\yy,\xx) \land \lnot\sd(\yy,\xx)) \nonumber \\
  &\fposzero\DSp(\xx) = \fZero\DS(\xx) \nonumber \\
  &\fposmax\DSp(\xx) = \fLargest\DS(\xx) \nonumber \\
  &\fposless\DSp(\xx,\yy) = \se(\xx,\yy) \land
    \fLess\DS(\xx,\yy) \nonumber \\
  &\fpossucc\DSp(\xx,\yy) = \se(\xx,\yy) \land
    \fSucc\DS(\xx,\yy) \nonumber \\
  &\fposfull\DSp = \forall\xx\exists\yy\Big(\se(\yy,\xx) \land 
    \fposzero\DSp(\yy)\Big) \land \tag{\ref{eset:1}}\\
    &\quad\forall\xx\Big(\lnot\fposmax\DSp(\xx) \limp
    \exists\yy\fpossucc\DSp(\xx,\yy)\Big) \nonumber \\
  &\fposA\DSp0(\xx) = \fposzero\DSp(\xx) \nonumber \\
  &\fposA\DSp{(\posp+1)}(\xx) = \exists\yy\Big(
    \fposA\DSp\posp(\yy) \land \fpossucc\DSp(\yy,\xx)
  \Big) \nonumber \\
    &\quad\text{ for } \posp \in [0,\largtbit{\tetr\sze2\szu}-1].\nonumber
\end{align}

Let $\StrA$ be a $\DSp$-structure, $\StrA \vDash \fposfull\DSp$ and let
$\relE=\at\StrA\se$.
Define the function $\Data\DSp\StrA : \Ecl\relE \to
\tBitstrings{\tetr{\sze+1}2\szu}$ assiging a $\tetr{\sze+1}2\szu$-bit
bitstring to any $\relE$-class $\eclX$ by:
\begin{equation}
  \Data\DSp\StrA_\posp\eclX = \begin{cases}
  1 \text{ if } \StrA \vDash \fposA\DSp{(\posp-1)}(\ea) 
  \text{ implies } \StrA \vDash \fbitI\DSp(\ea)
  \text{ for all } \ea \in \eclX
  \\
  0 \text{ otherwise}
  \end{cases}\tag{\ref{eset:2}}
\end{equation}
for $\posp \in [1,\tetr{\sze+1}2\szu]$.

Define the following formulas:
\begin{align}
  &\fEq\DSp(\xx,\yy) = \forall\xxp\forall\yyp\Big(\se(\xxp,\xx) \land
  \se(\yyp,\yy) \land \tag{\ref{eset:3}} \\
  &\quad\fposeq\DSp(\xxp,\yyp) \limp (\fbitO\DSp(\xxp) \lequ
  \fbitO\DSp(\yyp))\Big)
  \nonumber \\
  &\fZero\DSp(\xx) = \forall\yy\Big(\se(\yy,\xx) \limp \fbitO\DSp(\yy)\Big)
  \tag{\ref{eset:4}} \\
  &\fLargest\DSp(\xx) = \forall\yy\Big(\se(\yy,\xx) \limp \fbit1\DSp(\yy)\Big)
  \tag{\ref{eset:5}} \\
  &\fLess\DSp(\xx,\yy) = \exists\xxp\exists\yyp\Bigg(\se(\xxp,\xx) \land
  \se(\yyp,\yy) \land \tag{\ref{eset:6}} \\
  &\quad\Big(\fposeq\DSp(\xxp,\yyp) \land \fbitO\DSp(\xxp) \land
  \fbitI\DSp(\yyp)\Big) \land \nonumber \\
  &\quad\forall\xxpp\Big(\fposless\DSp(\xxp,\xxpp) \limp \exists\yypp\Big(
  \se(\yypp,\yyp) \land \nonumber \\
  &\quad\fposeq\DSp(\yypp,\xxpp) \land (\fbitO\DSp(\yypp) \lequ
  \fbitO\DSp(\xxpp)) \Big)\Big)\Bigg) \nonumber \\
  &\fSucc\DSp(\xx,\yy) = \exists\xxp\exists\yyp\Bigg(\se(\xxp,\xx) \land
  \se(\yyp,\yy) \land \tag{\ref{eset:7}} \\
  &\quad\Big(\fposeq\DSp(\xxp,\yyp) \land \fbitO\DSp(\xxp) \land
  \fbitI\DSp(\yyp)\Big) \land \nonumber \\
  &\quad\forall\xxpp\Big(\fposless\DSp(\xxpp,\xxp) \limp \fbitI\DSp(\xxpp)\Big)
  \land \nonumber \\
  &\quad\forall\yypp\Big(\fposless\DSp(\yypp,\yyp) \limp \fbitO\DSp(\yypp)\Big)
  \land \nonumber \\
  &\quad\forall\xxpp\Big(\fposless\DSp(\xxp,\xxpp) \limp \exists\yypp\Big(
  \se(\yypp,\yyp) \land \nonumber \\
  &\quad\fposeq\DSp(\yypp,\xxpp) \land (\fbitO\DSp(\yypp) \lequ
  \fbitO\DSp(\xxpp))\Big)\Big)\Bigg). \nonumber
\end{align}

If $\nM \in \tBitnums{\tetr{\sze+1}2\szu}$ is an $\tetr{\sze+1}2\szu$-bit
number, let $\tbit = \bsz\nM$ and define the formula:
\begin{align}
&\fDataA\DSp\nM(\xx) = \forall\yy\Bigg(\se(\yy,\xx) \limp
\bigwedge_{0 \leq \posp < \tbit} \Big(\fposA\DSp\posp(\yy) \limp
\fbit\DSp{\benc{\nM}_{\posp+1}}(\yy)\Big) \land \tag{\ref{eset:8}} \\
&\quad\forall\xx\Big(\fposA\DSp{(\tbit-1)}(\yy) \land 
\fposless\DSp(\yy,\xx) \limp \fbitO\DSp(\xx)\Big)\Bigg). \nonumber
\end{align}
This completes the definition of the $(\sze+1)$-exponential setup.

We can encode an instance of the $(\sze+1)$-exponential tiling problem into a
(finite) satisfiability $\DS$-formula completely analogously to the previous
section. Thus we have:
\begin{proposition}
The (finite) satisfiability problem for the monadic first-order logic with
$\sze$ equivalence symbols in refinement $\Lvp\Fo\nonv1\Eea\sze\agrefine$ is
$\ceNExpTime{(\sze+1)}$-hard. Even the three-variable fragment
$\Lvp\Fo31\Eea\sze\agrefine$ has this property.

By \Cref{prop:global-to-refine-n} and \Cref{prop:local-to-refine-n}, the same
holds for $\Lvp\Fo{(3)}1\Eea\sze\agglobal$ and
$\Lvp\Fo{(3)}1\Eea\sze\aglocal$.
\end{proposition}
\begin{proposition}
The (finite) satisfiability problem for the monadic first-order logic with many
equivalence symbols in refinement $\Lvp\Fo\nonv1\Eea\nosze\agrefine$ is
$\cElementary$-hard. Even the three-variable fragment
$\Lvp\Fo31\Eea\nosze\agrefine$ has this property.

By \Cref{prop:global-to-refine} and \Cref{prop:local-to-refine}, the same 
holds for $\Lvp\Fo{(3)}1\Eea\nosze\agglobal$ and
$\Lvp\Fo{(3)}1\Eea\nosze\aglocal$.
\end{proposition}