% Granularity

In this section we demonstrate how to replace the finest equivalence from a
sequence of equivalences in refinement with a counter setup. This works if the
structures are \emph{granular},
that is if the finest equivalence doesn't have many classes within a single
bigger equivalence class.

\begin{definition}
Let $\seq{\relD,\relE}$ be a sequence of two equivalence relations on $\domA$ in
refinement.
Let $\gls{granular-g} \in \pNats$.
The sequence is \emph{$\gr$-granular} if every $\relE$-equivalence class
includes at most $\gr$ $\relD$-equivalence classes.
\end{definition}
\begin{definition}
Let $\gr \in \pNats$ and let $\seq{\relD,\relE}$ be $\gr$-granular.
The function $\gcol : \domA \to [1,\gr]$ is a \emph{$\gr$-granular coloring} for
the sequence,
if two $\relE$-equivalent elements have the same color iff they are
$\relD$-equivalent.
That is, for every $(\ea,\eb) \in \relE$ we have
$\gcoll\ea = \gcoll\eb$ iff $(\ea,\eb) \in \relD$.
\end{definition}
\begin{remark}\label{rem:granular-exists}
Let $\gr \in \pNats$ and let $\seq{\relD,\relE}$ be $\gr$-granular.
Then there is a $\gr$-granular coloring for the sequence.
\end{remark}
\begin{proof}
Let $\eclX$ be an $\relE$-class.
Since $\relD \subseteq \relE$ is $\gr$-granular,
the set $\setS = \setbd{\relD[\ea]}{\ea \in \eclX}$ has cardinality at most
$\gr$.
Let $\inji : \setS \ito [1,\gr]$ be any injective function.
Define the color $\gcol$ on $\eclX$ as $\gcoll\ea = \inji(\relD[\ea])$.
\end{proof}

\begin{remark}\label{rem:granular-color}
Let $\relE \subseteq \domAA$ be an equivalence relation on $\domA$,
$\gr \in \pNats$ and $\gcol : \domA \to [1,\gr]$.
Then there is an equivalence relation $\relD \subseteq \relE$ on $\domA$ such
that $\seq{\relD,\relE}$ is $\gr$-granular, having $\gcol$ as a $\gr$-granular
coloring.
\end{remark}
\begin{proof}
Take $\relD = \setbd{(\ea,\eb) \in \relE}{\gcoll\ea = \gcoll\eb}$.
\end{proof}

\begin{definition}
Let $\gr \in \pNats$ and let $\tbit = \bsz\gr$ be the bitsize of $\gr$.
A \emph{$\gr$-color setup}
${\gls{color-setup} = \seq{\suu1,\suu2,\dots,\suu\tbit}}$ is just a $\tbit$-bit
counter setup.
\end{definition}

Let $\gr \in \pNats$ and let $\ColS = \seq{\suu1,\suu2,\dots,\suu\tbit}$ be a
$\gr$-color setup.
Let $\SigS$ be a predicate signature containing the binary symbols
$\sd$ and $\se$ and not containing any symbols from $\ColS$.
Let $\SigSp = \SigS + \ColS$ and $\SigG = \SigSp - \set{\sd}$.

\begin{definition}
Define the quantifier-free $\vFoF2\SigG$-formula
$\gls{finer-granularity-G}(\xx,\yy)$ by:
\[
  \fd\SigG(\xx,\yy) = \se(\xx,\yy) \land \feq\ColS(\xx,\yy).
\]
Note that $\fd\SigG(\xx,\yy)$ is constructible in deterministic polynomial time
from $\SigS$ and $\ColS$.
\end{definition}
\begin{definition}
Define the syntactic operation
$\gls{granularity-tr} : \FoF\SigS \to \FoF\SigG$ by:
\[
  \grtr\fphi = \fphip \land \fbetw\ColS1\gr,
\]
where $\fphip$ is obtained from the formula $\fphi$ by replacing all subformulas
of the form $\sd(\gx,\gy)$ by $\fd\SigG(\gx,\gy)$, where $\gx$ and $\gy$ are
(not necessarily distinct) variable symbols.
Note that this translation can be performed in deterministic polynomial time.
\end{definition}

\begin{lemma}\label{lem:granular-f-to-tr}
Let $\StrA$ be a $\SigS$-structure and suppose that the sequence of
symbols $\seq{\sd,\se}$ is interpreted in $\StrA$ as a
$\gr$-granular sequence $\seq{\relD,\relE}$.
Suppose that $\StrA \vDash \fphi$.
Then there is a $\SigSp$-enrichment $\StrAp$ of $\StrA$ such that
$\StrAp \vDash \grtr\fphi$.
\end{lemma}
\begin{proof}
By \Cref{rem:granular-exists}, there exists a $\gr$-granular coloring 
$\gcol : \domA \to [1,\gr]$.
We interpret the unary symbols in $\ColS$ so that $\data\ColS\StrA = \gcol$.
Since $\StrAp$ is an enrichment of $\StrA$, we have $\StrAp \vDash \fphi$.
Let $\ea, \eb \in \domA$.
Then $\StrAp \vDash \fd\SigG(\ea,\eb)$ is equivalent to:
\[
  \StrAp \vDash \se(\ea,\eb) \text{ and } \StrAp \vDash \feq\ColS(\ea,\eb),
\]
which is equivalent to:
\[
  (\ea,\eb) \in \relE \text{ and } \data\ColS\StrAp\ea = \data\ColS\StrAp\eb,
\]
which, since $\data\ColS\StrAp = \gcol$ is a $\gr$-granular coloring, is
equivalent to:
\[
  (\ea,\eb) \in \relD.
\]
Hence $\StrAp \vDash \forall\xx\forall\yy(\sd(\xx,\yy) \lequ \fd\SigG(\xx,\yy))$
and since $\StrAp \vDash \fphi$, we have $\StrAp \vDash \grtr\fphi$.
\end{proof}

\begin{lemma}\label{lem:granular-tr-to-f}
Let $\StrA$ be a $\SigG$-structure and suppose that the binary symbol $\se$ is
interpreted in $\StrA$ as an equivalence relation on $\domA$.
Suppose that $\StrA \vDash \grtr\fphi$.
Then there is a $\SigSp$-structure $\StrAp$ enriching $\StrA$ such that 
$\StrAp \vDash \fphi$ and the sequence of binary symbols $\seq{\sd,\se}$ is
interpreted in $\StrAp$ as a $\gr$-granular sequence $\seq{\relD,\relE}$.
\end{lemma}
\begin{proof}
Since $\StrA \vDash \fbetw\ColS1\gr$, we have 
$\data\ColS\StrA\ea \in [1,\gr]$ for all $\ea \in \domA$.
Define ${\gcol : \domA \to [1,\gr]}$ by $\gcoll\ea = \data\CS\StrA\ea$.
By \Cref{rem:granular-exists}, we can find $\relD \subseteq \relE$ such that 
the sequence $\seq{\relD,\relE}$ is $\gr$-granular, having $\gcol$ as a
$\gr$-granular coloring.
Consider the $\SigSp$-structure $\StrAp$, where $\at\StrAp\sd = \relD$.
Since $\StrAp$ is an enrichment of $\StrA \vDash \grtr\fphi$, we have
$\StrAp \vDash \grtr\fphi$.
Let $\ea,\eb \in \domA$. 
Then $\StrAp \vDash \fd\SigG(\ea,\eb)$ is equivalent to:
\[
  \StrAp \vDash \se(\ea,\eb) \text{ and } \StrAp \vDash \feq\ColS(\ea,\eb),
\]
which is equivalent to:
\[
  (\ea,\eb) \in \relE \text{ and } \gcoll\ea = \gcoll\eb,
\]
which, since $\gcol$ is a $\gr$-granular coloring, is equivalent to:
\[
  (\ea,\eb) \in \relD.
\]
Hence
$\StrAp \vDash \forall\xx\forall\yy
\left(\se(\xx,\yy) \lequ \fd\SigG(\xx,\yy)\right)$
and since $\StrAp \vDash \grtr\fphi$, we have $\StrAp \vDash \fphi$.
\end{proof}
