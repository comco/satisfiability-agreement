% Reductions introduction
In this chapter we provide polynomial-time reductions from the case of
equivalence symbols in global or local agreement to the case of equivalence
symbols in refinement.

We restrict our attention to binary predicate signatures only consisting of
unary and binary predicate symbols.
To denote various logics with built-in equivalence symbols, we use the notation
\[
  \gls{LvpEea}
\]
where:
\begin{itemize}
  \item $\LL$ is the \emph{ground logic}
  \item $\nv$, if given, bounds the number of variables
  \item $\sze$, if given, bounds the number of built-in equivalence symbols
  \item $\ag \in \set{\agrefine, \agglobal, \aglocal}$, if given, gives the
  agreement condition between the built-in equivalence symbols
  \item $\pow$, the \emph{signature power}, specifies constraints on the
  signature:
  \begin{itemize}
    \item if $\pow = 0$, the signature consists of only constantly many unary
    predicate symbols in addition to the built-in equivalence symbols
    \item if $\pow = 1$, the signature consists of unboundedly many unary
    predicate symbols in addition to the built-in equivalence symbols
    \item if $p$ is not given, the signature consists of unboundedly many unary
    and binary predicate symbols in addition to the built-in equivalence symbols.
    This is the commonly investigated fragment with respect to satisfiability of
    the two-variable logics with or without counting quantifiers.
  \end{itemize}
\end{itemize}

For example $\Lvp\Fo\nonv1$ is the monadic first-order logic, featuring only
unary predicate symbols.
$\Lvp\Fo\nonv0\Eea1\noag$ is the first-order logic of a single equivalence
relation.
$\Lvp\Fo2\nonv\Eea2\noag$ is the two-variable logic, featuring unary, binary
predicate symbols and two built-in equivalence symbols.
$\Lvp\Fo21\Eea2\aglocal$ is the two-variable logic,
featuring unary predicate symbols and two built-in equivalence symbols in local
agreement.
$\Lvp\Fo\nonv1\Eea\nosze\agglobal$ is the monadic first-order logic featuring
many equivalence symbols in global agreement.

When we working with a concrete logic, for example
$\Lvp\Fo2\nopow\Eea2\aglocal$,
we implicitly assume an appropriate generic predicate signature $\SigS$ for it.
In this case, there are two built-in equivalence symbols $\sd$ and $\se$ in
$\SigS$ and in addition $\SigS$ contains arbitrary many unary
and binary predicate symbols.
The \emph{intended interpretation} of the built-in equivalence symbols is fixed
by an appropriate condition $\fthe$.
In this case:
\[
  \fthe = \fequiv\sd \land \fequiv\se \land \flocal{\sd,\se}.
\] 
Note that the interpretation condition might in general be a first-order formula
outside the logic in interest, as in this case, since for instance $\fequiv\sd$
uses the variables $\xx,\yy$ and $\zz$ and the logic
$\Lvp\Fo2\nopow\Eea2\aglocal$ is a two-variable logic.
Recall that when talking about semantics, we include the intended interpretation
condition in the definition of $\SigS$-structures.