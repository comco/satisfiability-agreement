% Local to refine

In this section demonstrate how (finite) satisfiability in logics featuring
builtin equivalence symbols in local agreement reduces to (finite)
satisfiability in logics featuring builtin equivalence symbols in refinement.
Our strategy is to start with the level equivalences which form a refinement,
and to encode a permutation specifying the local chain structure for every
element in the structure.

Fix an arbitrary ground logic $\aL \in \set{\Fo, \Foc}$ and think of $\SigS$ as
a predicate signature for the logics $\aL\Eea\sze\aglocal$ and
$\aL\Eea\sze\agrefine$. The $\sze$ builtin equivalence symbols of $\SigS$ are
$\see1, \see2, \dots, \see\sze$.

Let $\fphi$ be a $\aLF\SigS$-sentence.
The class of $\aL\Eea\sze\agrefine$-structures satisfying $\fphi$ coincides with
the class of $\aL\Eea\sze\aglocal$-structures satisfying
\[
  \fphi \land \frefine{\see1,\see2,\dots,\see\sze}.
\] 
Hence:
\[
  \FinASat\aL\Eea\sze\agrefine \red\cP \FinASat\aL\Eea\sze\aglocal.
\]
Since the size of the formula $\frefine{\see1,\see2,\dots,\see\sze}$ grows 
polynomially as $\sze$ grows, we have:
\[
  \FinASat\aL\Eea\nosze\agrefine \red\cP \FinASat\aL\Eea\nosze\aglocal.
\]

Consider the opposite direction.
Let $\ES = \seq{\see1,\see2,\dots,\see\sze}$ be a predicate signature consisting
of the binary predicate symbols $\see\ii$ (not necessarily interpreted as
equivalences).
Let $\StrA$ be an $\ES$-structure and suppose that the symbols $\see\ii$ are
interpreted in $\StrA$ as equivalence relations on $\domA$ in local agreement.
Let $\relEE\ii = \at\StrA{\see\ii}$ for $\ii \in [1,\sze]$.
Recall that for every $\ea \in \domA$ there is a permutation
$\permnu \in \Perms\sze$ satisfying \cref{eq:local-perm}:
\begin{equation}\label{eq:local-perm-1}
  \relEE{\permnu(1)}[\ea] \subseteq
  \relEE{\permnu(2)}[\ea] \subseteq \dots \subseteq
  \relEE{\permnu(\sze)}[\ea].
\end{equation}
\begin{definition}
The \emph{characteristic $\ES$-permutation} of $\ea$ in $\StrA$ is
the lexicographically smallest permutation $\permnu \in \Perms\sze$
satisfying \cref{eq:local-perm-1}.
Define the function $\gls{chperm-E-A} : \domA \to \Perms\sze$ so
that $\chperm\ES\StrA\ea$ is the characteristic $\ES$-permutation of $\ea$ in
$\StrA$.
\end{definition}
\begin{remark}\label{rem:local-eq-perm}
Let $\ea \in \domA$, $\permnu = \chperm\ES\StrA\ea$ and
$\ii < \jj \in [1,\sze]$.
Suppose that $\relEE{\permnu(\ii)}[\ea] = \relEE{\permnu(\jj)}[\ea]$.
Then $\permnu(\ii) < \permnu(\jj)$.
\end{remark}
\begin{proof}
Suppose not. For some $\ii < \jj \in [1,\sze]$ we have
$\permnu(\ii) \geq \permnu(\jj)$.
Since $\permnu$ is a permutation and $\ii \neq \jj$,
we have $\permnu(\ii) > \permnu(\jj)$.
Since $\relEE{\permnu(\ii)}[\ea] = \relEE{\permnu(\jj)}[\ea]$,
by \cref{eq:local-perm-1} we have $\relEE{\permnu(\kk)} = \relEE{\permnu(\ii)}$
for all $\kk \in [\ii, \jj]$.
Consider the permutation $\permmu \in \Perms\sze$ defined by:
\[
  \permmu(\kk) = \begin{cases}
    \permnu(\jj) &\text{ if } \kk = \ii \\
    \permnu(\ii) &\text{ if } \kk = \jj \\
    \permnu(\kk) &\text{ otherwise.}
  \end{cases}
\]
Clearly, $\permmu$ is a permutation satisfying \cref{eq:local-perm-1} that is
lexicographically smaller than $\permnu$ --- a contradiction.
\end{proof}

\begin{remark}\label{rem:local-inv-perm}
Let $\ea, \eb \in \domA$ and let
$\perma = \chperm\ES\StrA\ea$ and $\permb = \chperm\ES\StrA\eb$.
Let $\ii \in [1,\sze]$ and suppose that $(\ea, \eb) \in \relEE\ii$.
Then $\inv\perma(\ii) = \inv\permb(\ii)$.
\end{remark}
\begin{proof}
Suppose not, so $\inv\perma(\ii) \neq \inv\permb(\ii)$.
Let $\posp = \inv\perma(\ii)$ and $\posq = \inv\permb(\ii)$.
Without loss of generality, suppose that $\posp < \posq$.
Thus $\posp$ is the position of $\ii$ in the permutation $\perma$ and 
$\posq > \posp$ is the position of $\ii$ in the permutation $\permb$. 
By the pigeonhole principle, there is $\kk \in [1,\sze]$ that occurs after $\ii$
in $\perma$ and before $\jj$ in $\permb$:
$\posp < \inv\perma(\kk)$ and $\inv\permb(\kk) < \posq$.
Since $\permb$ is the characteristic $\ES$-permutation of $b$ in $\StrA$, by
\cref{eq:local-perm-1} we have $\relEE\kk[\eb] \subseteq \relEE\ii[\eb]$.
Since $(\ea, \eb) \in \relEE\ii$, we have $\relEE\kk[\eb] \subseteq
\relEE\ii[\ea]$.
Since $\relEE\kk[\eb] \subseteq \relEE\ii[\ea]$ are equivalence classes,
$\relEE\kk[\ea] \subseteq \relEE\ii[\ea]$.
Since $\kk$ occurs after $\ii$ in $\perma$, which is the characteristic
$\ES$-permutation of $\ea$ in $\StrA$, 
by \cref{eq:local-perm-1} we have $\relEE\kk[\ea] = \relEE\ii[\ea]$.
By \Cref{rem:local-eq-perm}, $\ii < \kk$.
By the contrapositive of \Cref{rem:local-eq-perm},
$\relEE\kk[\eb] = \relEE\ii[\eb]$ is impossible.
Since $\kk$ occurs before $\ii$ in $\permb$, 
by \cref{eq:local-perm-1} we have $\relEE\kk[\eb] \subset \relEE\ii[\eb]$.
Hence
\[
  \relEE\kk[\eb] \subset \relEE\ii[\eb] = \relEE\ii[\ea] = \relEE\kk[\ea]
\]
--- a contradiction --- since the equivalence classes $\relEE\kk[\eb]$ and
$\relEE\kk[\ea]$ are either equal or disjoint.
\end{proof}

Let
$\seqL = \seq{\relLL1,\relLL2,\dots,\relLL\sze} \subseteq \domAA$ be
the levels of $\seqE = \seq{\relEE1, \relEE2, \dots, \relEE\sze}$.
Recall that by \Cref{rem:local-lvl-refine}, the levels are equivalence relations
on $\domA$ in refinement.

\begin{remark}\label{rem:local-lvl-perm}
Let $\ea \in \domA$, $\perma = \chperm\ES\StrA\ea$ and let $\kk \in [1,\sze]$.
Then $\relLL\kk[\ea] = \relEE{\perma(\kk)}[\ea]$.
\end{remark}
\begin{proof}
Since $\perma$ satisfies \cref{eq:local-perm-1}, by \Cref{lem:local-lvl-perm}:
\[
  \relLL\kk[\ea] = 
  \relLL{\inv\perma(\perma(\kk))}[\ea] =
  \relEE{\perma(\kk)}[\ea].
\]
\end{proof}

\begin{remark}\label{rem:local-lvl-agree}
Let $\ea, \eb \in \domA$, $\perma = \chperm\ES\StrA\ea$,
$\permb = \chperm\ES\StrA\eb$ and $\kk \in [1,\sze]$.
Suppose that $(\ea, \eb) \in \relLL\kk$.
Then $\perma(\kk) = \permb(\kk)$.
That is, the elements connected at level $\kk$ agree at position $\kk$ in their
characteristic permutations.
\end{remark}
\begin{proof}
By \Cref{rem:local-lvl-perm}, $\relLL\kk[\ea] = \relEE{\perma(\kk)}[\ea]$, thus
$(\ea, \eb) \in \relEE{\perma(\kk)}$. 
By \Cref{rem:local-eq-perm},
\[
  \kk = \inv\perma(\perma(\kk)) = \inv\permb(\perma(\kk)).
\]
Hence $\permb(\kk) = \perma(\kk)$.
\end{proof}

Let $\PS = \seq{\suu{11}, \suu{12}, \dots, \suu{\sze\tbit}}$ be an
$\sze$-permutation setup.
Let $\LS = \seq{\slee1, \slee2, \dots, \slee\sze} + \PS$ be a predicate
signature containing the binary predicate symbols $\slee\kk$ (not necessarily
interpreted as equivalence relations) together with the symbols from $\PS$.
\begin{definition}
Define the $\vFoF2\LS$-sentence $\gls{ffixperm-L}$ by:
\[
  \ffixperm\LS = \forall\xx \forall\yy \bigwedge_{1 \leq \kk \leq \sze}
  \left(\slee\kk(\xx,\yy) \limp \feq{\PSp{\kk}}(\xx,\yy)\right).
\]
\end{definition}
\begin{definition}
Define the $\vFoF2\LS$-sentence $\gls{flocperm-L}$ by:
\[
  \flocperm\LS = \fperm\PS \land \ffixperm\LS.
\]
\end{definition}
\begin{remark}\label{rem:local-perm-val-fixed}
Let $\StrA$ be an $\LS$-structure and suppose that $\StrA \vDash \flocperm\LS$.
Let $\ea, \eb \in \domA$, $\kk \in [1,\sze]$ and suppose that
$\StrA \vDash \slee\kk(\ea, \eb)$.
Let $\perma = \data\PS\StrA\ea$ and $\permb = \data\PS\StrA\eb$ be the
$\sze$-permutations at $\ea$ and $\eb$,
\emph{encoded by the permutation setup $\PS$}.
Then $\perma(\kk) = \permb(\kk)$.
\end{remark}
\begin{proof}
Since $\StrA \vDash \ffixperm\LS$ and $\StrA \vDash \slee\kk(\ea,\eb)$, we have
$\StrA \vDash \feqA\PS\kk(\ea,\eb)$, which means $\perma(\kk) = \permb(\kk)$.
\end{proof}

\begin{definition}
For $\ii \in [1,\sze]$, define the quantifier-free $\vFoF2\LS$-formula
$\gls{fel-L-i}$ by:
\[
  \fel\LS\ii(\xx,\yy) = \bigwedge_{1 \leq \kk \leq \nn}
  \left(\feqAA\LS\kk\ii(\xx) \limp \slee\kk(\xx,\yy)\right).
\]
\end{definition}

\begin{remark}\label{rem:local-e-m}
Let $\StrA$ be an $\LS$-structure and suppose that $\StrA \vDash \flocperm\LS$
and that the binary symbols $\slee\kk$ are interpreted in $\StrA$ as equivalence
relations on $\domA$ in refinement.
Define $\permnu : \domA \to \Perms\sze$ by $\permnu(\ea) = \data\PS\StrA\ea$
for $\ea \in \domA$.
Let $\ea \in \domA$ be arbitrary.
Then for all $\ii \in [1,\sze]$:
\[
  \at\StrA{\fel\LS\ii}[\ea] = \at\StrA{\slee{\inv{\permnu(\ea)}(\ii)}}[\ea].
\]
\end{remark}
\begin{proof}
Let $\relEE\ii = \at\StrA{\fel\LS\ii}$ and $\relLL\ii = \at\StrA{\slee\ii}$
for every $\ii \in [1,\sze]$.
Let $\ii \in [1,\sze]$ be arbitrary. 
Let $\perma = \permnu(\ea)$ and $\kk = \inv\perma(\ii)$,
so $\perma = \data\PS\StrA\ea$ and $\perma(\kk) = \ii$.
We have to show that $\relEE\ii[\ea] = \relLL\kk[\ea]$.
Since $\perma$ is a permutation, for every $\kkp \in [1,\sze]$ we have:
\begin{equation}\label{eq:local-e-equiv}
  \mifff{
  \StrA \vDash \feqAA\PS\kkp\ii(\ea)}{
  \perma(\kkp) = \ii}{ 
  \kkp = \kk}.
\end{equation}
First, suppose $\eb \in \relEE\ii[\ea]$.
Then $\StrA \vDash \fel\LS\ii(\ea,\eb)$ and by \cref{eq:local-e-equiv} we have
$\StrA \vDash \slee\kk(\ea,\eb)$, hence $\eb \in \relLL\kk[\ea]$.

Next, suppose $\eb \not\in \relEE\ii[\ea]$.
Then $\StrA \vDash \lnot\fel\LS\ii(\ea,\eb)$, so there is some
$\kkp \in [1,\sze]$ such that 
$\StrA \vDash \lnot(\feqAA\LS\kkp\ii(\ea) \limp \slee\kkp(\ea,\eb)) 
\equiv \feqAA\LS\kkp\ii(\ea) \land \lnot\slee\kkp(\ea,\eb)$.
By \cref{eq:local-e-equiv} we have $\kkp = \kk$.
Hence $\StrA \vDash \lnot \slee\kk(\ea,\eb)$, so $\eb \not\in \relLL\kk[\ea]$.
\end{proof}

\begin{remark}\label{rem:local-e-local}
Let $\StrA$ and $\permnu$ are declared as in \Cref{rem:local-e-m}.
Then the sequence of interpretations
$\seqsm{\at\StrA{\fel\LS1},\at\StrA{\fel\LS2},\dots,\at\StrA{\fel\LS\sze}}$ is a
sequence of equivalence relations on $\domA$ in local agreement.
\end{remark}
\begin{proof}
Let $\relEE\ii = \at\StrA{\fel\LS\ii}$ and $\relLL\ii = \at\StrA{\slee\ii}$ for
every $\ii \in [1,\sze]$.
Let $\ii \in [1,\sze]$ be arbitrary.
We check that $\relEE\ii$ is reflexive, symmetric and transitive.
\begin{itemize}
  \item For reflexivity, let $\ea \in \domA$.
  By \Cref{rem:local-e-m},
  $\relEE\ii[\ea] = \relLL\kk[\ea]$ for $\kk = \inv{\permnu(\ea)}(\ii)$. 
  But $\relLL\kk[\ea]$ is an equivalence class, hence $\ea \in \relLL\kk[\ea]$,
  so $(\ea,\ea) \in \relEE\ii$.

  \item For symmetry, let $\ea, \eb \in \domA$ and $(\ea,\eb) \in \relEE\ii$.
  Let $\kk = \inv{\permnu(\ea)}(\ii)$ so that $\ii = \permnu(\kk)$.
  By \Cref{rem:local-e-m}, $\relEE\ii[\ea] = \relLL\kk[\ea]$.
  Thus $\StrA \vDash \slee\kk(\ea,\eb)$ and by
  \Cref{rem:local-perm-val-fixed}, $\ii = \permnu(\ea)(\kk) =
  \permnu(\eb)(\kk)$.
  By \Cref{rem:local-e-m}:
  \[
    \relEE\ii[\eb] =
    \at\StrA{\fel\LS\ii}[\eb] =
    \at\StrA{\slee{\inv{\permnu(\eb)}(\ii)}}[\eb] =
    \relLL\kk[\eb] = \relLL\kk[\ea].
  \]
  Since $\ea \in \relLL\kk[\ea] = \relEE\ii[\eb]$, we have $(\eb,\ea) \in
  \relEE\ii$.
   
  \item For transitivity, continue the argument for symmetry.
  Let $\ec \in \relEE\ii[\eb]$.
  Then $\ec \in \relEE\ii[\eb] = \relLL\kk[\ea] = \relEE\ii[\ea]$,
  thus $(\ea,\ec) \in \relEE\ii$.
\end{itemize}
By \Cref{rem:local-e-m}, since the relations $\relLL\kk$ are in refinement, we
have that $\relEE1, \relEE2, \dots, \relEE\sze$ are in local agreement.
\end{proof}

Let $\ES = \seq{\see1,\see2,\dots,\see\sze}$ be a predicate signature consisting
of binary predicate symbols.
Let $\SigS$ be a predicate signature enriching $\ES$
and not containing any symbols from $\LS$. 
Let $\SigSp = \SigS + \LS$ and $\LSp = \SigSp - \ES$.

\begin{definition}
Define the syntactic operation $\gls{ltr} : \aLF\SigS \to \aLF\LSp$
by:
\[
  \ltr\fphi = \fphip \land \flocperm\LS,
\]
where $\fphip$ is obtained from $\fphi$ by replacing all occurrences of a
subformula of the form $\see\ii(\gx,\gy)$ by the formula $\fel\LS\ii(\gx,\gy)$, 
where $\gx$ and $\gy$ are (not necessarily distinct) variable symbols and
$\ii \in [1,\sze]$.
\end{definition}
\begin{remark}
Let $\fphi$ be a $\aLF\SigS$-formula and let $\StrA$ be a $\SigS$-structure.
Suppose that $\StrA \vDash \fphi$ and that the symbols 
$\see1, \see2, \dots, \see\sze$ are interpreted in $\StrA$ as equivalence
relations on $\domA$ in local agreement.
Then there is a $\SigSp$-enrichment $\StrAp$ of $\StrA$ such that
$\StrAp \vDash \ltr\fphi$ and that the symbols 
$\slee1, \slee2, \dots, \slee\sze$ are interpreted in $\StrAp$ as equivalence
relations on $\domA$ in refinement.
\end{remark}
\begin{proof}
Since the binary symbols $\see1, \see2, \dots, \see\sze$ are interpreted as
equivalence relations on $\domA$ in local agreement in $\StrA$, we may define
the levels $\relLL1, \relLL2, \dots, \relLL\sze \subseteq \domAA$
and the characteristic $\ES$-permutation mapping 
$\permnu = \chperm\ES\StrA : \domA \to \Perms\sze$.
Consider an enrichment $\StrAp$ of $\StrA$ where 
$\at\StrAp{\slee\ii} = \relLL\ii$.
By \Cref{rem:local-lvl-refine}, $\relLL\ii$ are equivalences on $\domA$ in
refinement.
We interpret the unary symbols from the
permutation setup $\PS$ so that $\data\PS\StrAp\ea = \permnu(\ea)$ for all
$\ea \in \domA$.
By \Cref{rem:local-lvl-agree}, $\StrAp \vDash \ffixperm\LS$.
By \Cref{rem:local-e-m}, followed by \Cref{lem:local-lvl-perm},
for every $\ii \in [1,\sze]$ and $\ea \in \domA$ we have:
\[
  \at\StrAp{\fel\LS\ii}[\ea] =
  \at\StrAp{\slee{\inv{\permnu(\ea)}(\ii)}}[\ea] =
  \at\StrAp{\see{\permnu(\ea)(\inv{\permnu(\ea)}(\ii))}}[\ea] =
  \at\StrAp{\see\ii}[\ea].
\]
By \Cref{rem:local-e-local}, the interpretations $\fel\LS\ii$ are
equivalence relations and since they have the same classes as the
interpretations of $\see\ii$, we have 
$\StrAp \vDash \forall\xx \forall\yy
\left(\see\ii(\xx,\yy) \lequ \fel\LS\ii(\xx,\yy)\right)$
and since $\StrAp \vDash \fphi$, we have $\StrAp \vDash \ltr\fphi$.
\end{proof}

\begin{remark}
Let $\fphi$ be a $\aLF\SigS$-formula and let $\StrA$ be an $\LSp$-structure.
Suppose that $\StrA \vDash \ltr\fphi$ and that the symbols
$\slee1, \slee2, \dots, \slee\sze$ are interpreted as equivalence relations on
$\domA$ in refinement in $\StrA$.
Then there is a $\SigSp$-enrichment $\StrAp$ of $\StrA$ such that
$\StrAp \vDash \fphi$ and that the binary symbols
$\see1, \see2, \dots, \see\sze$
are interpreted as equivalence relations on $\domA$ in global agreement in
$\StrAp$.
\end{remark}
\begin{proof}
Consider an enrichment $\StrAp$ of $\StrA$ to a $\SigSp$-structure where
$\at\StrAp{\see\ii} = \at\StrA{\fel\LS\ii}$.
By \Cref{rem:local-e-local}, $\at\StrAp{\see\ii}$ are equivalence
relations on $\domA$ in local agreement. 
For every $\ii \in [1,\sze]$ we have
$\StrAp \vDash \forall\xx\forall\yy \left(\see\ii(\xx,\yy) \lequ
\fel\LS\ii(\xx,\yy)\right)$ by definition.
Since $\StrAp \vDash \ltr\fphi$ we have $\StrAp \vDash \fphi$.
\end{proof}

The last two remarks show that a $\aL\Eea\sze\aglocal$-formula $\fphi$
has essentially the same models as the $\aL\Eea\sze\agrefine$-formula
$\ltr\fphi$, so we have shown:
\begin{proposition}\label{prop:local-to-refine-n}
The logic $\aL\Eea\sze\aglocal$ has the finite model property iff
the logic $\aL\Eea\sze\agrefine$ has the finite model property.
The corresponding satisfiability problems are polynomial-time equivalent:
$\FinASat\aL\Eea\sze\aglocal \redeq\cP \FinASat\aL\Eea\sze\agrefine$.
\end{proposition}

Since the relative size of $\ltr\fphi$ with respect to $\fphi$ grows
polynomially as $\sze$ grows, we have shown:
\begin{proposition}\label{prop:local-to-refine}
The logic $\aL\Eea\nosze\aglocal$ has the finite model property iff
the logic $\aL\Eea\nosze\agrefine$ has the finite model property.
The corresponding satisfiability problems are polynomial-time equivalent:
$\FinASat\aL\Eea\nosze\aglocal \redeq\cP \FinASat\aL\Eea\nosze\agrefine$.
\end{proposition}

The reduction is two-variable first-order and uses additional $(\sze\tbit)$
unary predicate symbols for the permutation setup $\PS$, so it is also valid
for the two-variable fragments $\Lvp\aL20\Eea\sze\ag$, $\Lvp\aL21\Eea\sze\ag$
and $\Lvp\aL21\Eea\nosze\ag$ for $\ag \in \set{\aglocal, \agrefine}$
respectively.