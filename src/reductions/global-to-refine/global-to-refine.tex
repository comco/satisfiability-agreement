% Global to refine glossary

In this section we demonstrate how (finite) satisfiability in logics featuring
builtin equivalence symbols in global agreement reduces to (finite)
satisfiability in logics featuring builtin equivalence symbols in refinement.
Our strategy is to encode the permutation of the builtin equivalence symbols in
global agreement that turns them in refinement into a permutation setup.

Let $\SigS$ be a predicate signature for the logics $\Fo\Eea\sze\agglobal$
or $\Fo\Eea\sze\agrefine$.
The $\sze$ builtin equivalence symbols of $\SigS$ are $\see1, \see2, \dots,
\see\sze$.

Let $\fphi$ be a $\SigS$-sentence.
The class of $\Fo\Eea\sze\agrefine$-structures satisfying $\fphi$
coincides with the class of $\Fo\Eea\sze\agglobal$-structures satisfying
\[
  \fphi \land \frefine{\see1,\see2,\dots,\see\sze}.
\]
Hence:
\[
  \FinASat{\Fo\Eea\sze\agrefine} \red\cP
  \FinASat{\Fo\Eea\sze\agglobal}.
\]
Since the length of the formula $\frefine{\see1,\see2,\dots,\see\sze}$ grows
polynomially as $\sze$ grows:
\[
  \FinASat{\Fo\Eea\nosze\agrefine} \red\cP
  \FinASat{\Fo\Eea\nosze\agglobal}.
\]

Consider the opposite direction.
Let $\PS = \seq{\suu{11}, \suu{12}, \dots, \suu{\sze\tbit}}$ be an
$\sze$-permutation setup (where $\tbit = \bsz\sze$).
\begin{definition}
Define the $\vFoF2\PS$-sentence $\gls{falleq-P}$ by:
\[
  \falleq\PS = \forall\xx \forall\yy \bigwedge_{1 \leq \ii \leq \sze}
  \feq{\PS(\ii)}(\xx,\yy).
\]
\end{definition}
If $\StrA$ is a $\PS$-structure, then $\StrA \vDash \falleq\PS$ iff
$\data\PS\StrA\ea = \data\PS\StrA\eb$ for all $\ea, \eb \in \domA$.
If $\domA$ is a nonempty set and $\vectv \in \ntBitnums\sze\tbit$ is any
$\sze$-dimensional $\tbit$-vector, there is a $\PS$-structure $\StrA$ over
$\domA$ such that $\StrA \vDash \falleq\PS$ and
$\data\PS\StrA\ea = \vectv$ for all $\ea \in \domA$.

\begin{definition}
Define the $\vFoF2\PS$-sentence $\gls{fglobperm-P}$ by:
\[
  \fglobperm\PS = \fperm\PS \land \falleq\PS.
\]
\end{definition}
If $\StrA$ be a $\PS$-structure then $\StrA \vDash \fglobperm\PS$ iff there
is a permutation $\permnu \in \Perms\sze$ such that
$\data\PS\StrA\ea = \permnu$ for all $\ea \in \domA$.

If $\domA$ be a nonempty set and $\permnu \in \Perms\sze$ is any permutation,
there is a $\PS$-structure $\StrA$ over $\domA$ such that
$\StrA \vDash \fglobperm\PS$ and $\data\PS\StrA\ea = \permnu$ for all 
$\ea \in \domA$.

Let $\LS = \seq{\slee1, \slee2, \dots, \slee\sze} + \PS$ be a predicate
signature consisting of the binary predicate symbols $\slee\kk$ in addition to
the symbols from $\PS$.

\begin{definition}
For $\ii \in [1, \sze]$, define the quantifier-free $\vFoF2\LS$-formula
$\gls{feg-L-i-x-y}$ by:
\[
  \feg\LS\ii(\xx,\yy) = \bigwedge_{1 \leq \kk \leq \sze}
  \left(\feqA{\PS(\kk)}\ii(\xx) \limp \slee\kk(\xx, \yy)\right).
\]
\end{definition}
\begin{remark}\label{rem:global-e-m}
Let $\StrA$ be an $\LS$-structure and suppose that 
$\StrA \vDash \fglobperm\PS$
and that the binary symbols $\slee\kk$ are interpreted as equivalence relations
on $\domA$ in refinement.
Recall that there is a permutation $\permnu \in \Perms\sze$ such that
$\data\PS\StrA\ea =\permnu$ for all $\ea \in \domA$.
Then for all $\ii \in [1, \sze]$:
\[
  \at\StrA{\feg\LS\ii} = \at\StrA{\slee{\inv\permnu(\ii)}}.
\]
In particular,
$\seq{\at\StrA{\feg\LS1}, \at\StrA{\feg\LS2}, \dots, \at\StrA{\feg\LS\sze}}$ is
a sequence of equivalence relations on $\domA$ in global agreement.
\end{remark}
\begin{proof}
Let $\kk = \inv\permnu(\ii)$, so $\permnu(\kk) = \ii$ and
$\data{\PSp\kk}\StrA\ea = \ii$.
Since $\permnu$ is a permutation, for every $\kkp \in [1, \sze]$:
\begin{equation}\label{eq:global-e}
\mifff{
  \StrA \vDash \feqA{\PS(\kkp)}\ii(\ea)}{
  \data{\PSp\kkp}\StrA\ea = \ii}{
  \kkp = \kk}.
\end{equation}

Let $\ea, \eb \in \domA$.
First suppose that $\StrA \vDash \feg\LS\ii(\ea,\eb)$.
By \cref{eq:global-e} we must have that ${\StrA \vDash
\feqA{\PS(\kk)}\ii(\ea)}$, hence $\StrA \vDash \slee\kk(\ea,\eb)$.

Now suppose that $\StrA \vDash \lnot\feg\LS\ii(\ea,\eb)$. There is some
$\kkp \in [1, \sze]$ such that:
\[
  \StrA \vDash \lnot\left(\feqA{\PS(\kkp)}\ii(\ea) \limp
  \slee\kkp(\ea,\eb)\right) \equiv \feqA{\PS(\kkp)}\ii(\ea) \land
  \lnot\slee\kkp(\ea,\eb).
\]
By \cref{eq:global-e} we have $\kkp = \kk$, hence
$\StrA \vDash \lnot\slee\kk(\ea,\eb)$.
\end{proof}

Let $\ES = \seq{\see1,\see2,\dots,\see\sze}$ be a predicate signature consisting
of the binary predicate symbols $\see\ii$.
Let $\SigS$ be a predicate signature enriching $\ES$ and not
containing any symbols from $\LS$.
Let $\SigSp = \SigS \cup \LS$ and
$\LSp = \SigSp - \ES$.

\begin{definition}
Define the syntactic operation $\gls{gtr} : \FoF\SigS \to \FoF\LSp$ by:
\[
  \gtr\fphi = \fphip \land \fglobperm\PS,
\]
where $\fphip$ is obtained from $\fphi$ by replacing all occurrences of a
subformula of the form $\see\ii(\gx,\gy)$ by the formula $\feg\LS\ii(\gx,\gy)$, 
where $\gx$ and $\gy$ are (not necessarily distinct) variables and
$\ii \in [1,\sze]$.
\end{definition}

\begin{remark}\label{rem:global-e-to-m}
Let $\fphi$ be a $\SigS$-formula and let $\StrA$ be a $\SigS$-structure.
Suppose that $\StrA \vDash \fphi$ and that the symbols 
$\see1,\see2,\dots,\see\sze$ are interpreted in $\StrA$ as equivalence
relations on $\domA$ in global agreement.
Then there is a $\SigSp$-enrichment $\StrAp$ of $\StrA$
such that $\StrAp \vDash \gtr\fphi$ and that the symbols
$\slee1,\slee2,\dots,\slee\sze$ are interpreted in $\StrAp$ as equivalence
relations on $\domA$ in refinement.
\end{remark}
\begin{proof}
There is a permutation $\permnu \in \Perms\sze$ such that
$\at\StrA{\see{\permnu(1)}} \subseteq
\at\StrA{\see{\permnu(2)}} \subseteq
\dots \subseteq
\at\StrA{\see{\permnu(\sze)}}$.
Consider an enrichment $\StrAp$ of $\StrA$
to a $\SigSp$-structure where 
$\at\StrAp{\slee\kk} = \at\StrA{\see{\permnu(\kk)}}$,
so the interpretations of $\slee\kk$ in $\StrAp$ 
are equivalence relations on $\domA$ in refinement.
We can interpret the unary predicate symbols from permutation setup 
$\PS$ in $\StrAp$ so that 
$\StrAp \vDash \fglobperm\PS$ and 
$\data\PS\StrA\ea = \permnu$ for all $\ea \in \domA$.
By \Cref{rem:global-e-m}, for every $\ii \in [1,\sze]$:
\[
  \at\StrAp{\feg\LS\ii} =
  \at\StrAp{\slee{\inv\permnu(\ii)}} =
  \at\StrAp{\see{\permnu(\inv\permnu(\ii))}} =
  \at\StrAp{\see\ii} =
  \at\StrA{\see\ii}.
\]
Hence $\StrAp \vDash
  \forall\xx\forall\yy
  \left(\see\ii(\xx,\yy) \lequ \feg\LS\ii(\xx,\yy)\right)$.
Since $\StrAp \vDash \fphi$ we have $\StrAp \vDash \gtr\fphi$.
\end{proof}

\begin{remark}\label{rem:global-m-to-e}
Let $\fphi$ be a $\SigS$-formula and let $\StrA$ be an $\LSp$-structure.
Suppose that $\StrA \vDash \gtr\fphi$ and that the symbols
$\slee1,\slee2,\dots,\slee\sze$ are interpreted in $\StrA$ as equivalence
relations on $\domA$ in refinement.
Then there is a $\SigSp$-enrichment $\StrAp$ of $\StrA$ such that
$\StrAp \vDash \fphi$ and that the symbols $\see1,\see2,\dots,\see\sze$ are
interpreted as equivalence relations on $\domA$ in global agreement in $\StrAp$.
\end{remark}
\begin{proof}
Consider an enrichment $\StrAp$ of $\StrA$ to a $\SigSp$-structure
where $\at\StrAp{\see\ii} = \at\StrA{\feg\LS\ii}$.
By \Cref{rem:global-e-m},
$\seqsm{\at\StrAp{\see1}, \at\StrAp{\see2}, \dots, \at\StrAp{\see\sze}}$
is a sequence of equivalence relations on $\domA$ in global agreement.
For every $\ii \in [1,\sze]$ we have 
$\StrAp \vDash \forall\xx\forall\yy(\see\ii(\xx,\yy) \lequ
\feg\LS\ii(\xx,\yy))$ by definition.
Since $\StrAp \vDash \gtr\fphi$ we have $\StrAp \vDash \fphi$.
\end{proof}

The last two remarks show that a $\Fo\Eea\sze\agglobal$-formula $\fphi$
has essentially the same models as the $\Fo\Eea\sze\agrefine$-formula
$\gtr\fphi$, so we have shown:
\begin{proposition}\label{prop:global-to-refine-n}
The logic $\Fo\Eea\sze\agglobal$ has the finite model property iff
the logic $\Fo\Eea\sze\agrefine$ has the finite model property. 
The corresponding satisfiability problems are polynomial-time equivalent:
$\FinASat{\Fo\Eea\sze\agglobal} \redeq\cP \FinASat{\Fo\Eea\sze\agrefine}$.
\end{proposition}

Since the relative size of $\gtr\fphi$ with respect to $\fphi$ grows
polynomially as $\sze$ grows, we have shown:
\begin{proposition}\label{prop:global-to-refine}
The logic $\Fo\Eea\nosze\agglobal$ has the finite model property iff
the logic $\Fo\Eea\nosze\agrefine$ has the finite model property. 
The corresponding satisfiability problems are polynomial-time equivalent:
$\FinASat{\Fo\Eea\nosze\agglobal} \redeq\cP \FinASat{\Fo\Eea\nosze\agrefine}$.
\end{proposition}

The reduction is two-variable first-order and uses additional $(\sze\tbit)$
unary predicate symbols for the permutation setup $\PS$, so it is also valid
for the two-variable fragments $\Lvp\Fo20\Eea\sze\ag$, $\Lvp\Fo21\Eea\sze\ag$
and $\Lvp\Fo21\Eea\nosze\ag$ for $\ag \in \set{\agglobal, \agrefine}$ (but not
for the fragment $\Lvp\Fo20\Eea\nosze\ag$).
