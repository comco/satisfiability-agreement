% Counting base case
Recall from~\Cref{sec:scott-nf} about normal forms that every $\vFoc2$-sentence
$\fphi$ can be reduced in deterministic polynomial time to a sentence
\[
  \forall\xx\forall\yy(\falpp0(\xx,\yy) \lor \xx=\yy) \land
  \bigwedge_{1 \leq \ii \leq \nm} \forall\xx\existseq{\nMM\ii}\yy
  (\falpp\ii(\xx,\yy) \land \xx\neq\yy),
\]
where $\nm\geq1$, $\nMM\ii\geq1$, all the formulas $\falpp\ii$ are
quantifier-free and use at most linearly many new unary and binary predicate symbols.
Let $\nM = 1 + \max\set{\nMM1,\nMM2,\dots,\nMM\nm}$.
The semantic connection between $\fphi$ and $\prtr\fphi$ is that they are
essentially equisatisfiable.
More precisely, every model for $\fphi$ of cardinality at least $2$ can be
enriched to a model for $\prtr\fphi$ and every model for $\prtr\fphi$ of
cardinality at least $\nM$ is a model for $\fphi$.
We refer to $\falpp0$ as the universal part of $\fphi$ and to $\falpp\ii$ for
$\ii\in[1,\nm]$ as the existenial parts of $\fphi$.

We replace the existential parts of a formula in normal form with fresh binary
predicate symbols $\smm\ii$ (the message symbols) for $\ii\in[1,\nm]$
having intended interpretation
$\forall\xx\forall\yy(\smm\ii(\xx,\yy)\lequ\falpp\ii(\xx,\yy))$.
Hence any $\vFoc2$-sentence can be transformed in deterministic polynomial time
into the form:
\begin{equation}\label{eq:counting-base}
\forall\xx\forall\yy(\falp(\xx,\yy)\lor\xx=\yy) \land
\bigwedge_{1\leq\ii\leq\nm} \forall\xx\existseq{\nMM\ii}\yy
(\smm\ii(\xx,\yy)\land\xx\neq\yy),
\end{equation}
where the universal part $\falp$ is quantifier-free and over an extended
signature.

\begin{definition}
A \emph{classified signature} $\ClSig\SigS\sms$ for the two-variable logic with
counting quantifiers $\vFoc2$ is a predicate signature $\SigS$ together with a
nonempty sequence
\[
\sms = \underbrace{\smm1\smm1\dots\smm1}_{\nMM1}
\underbrace{\smm2\smm2\dots\smm2}_{\nMM2}
\dots
\underbrace{\smm\nm\smm\nm\dots\smm\nm}_{\nMM\nm}
\]
($\nMM\ii\geq1$) of distinct binary predicate symbols $\smm\ii$ from $\SigS$
having intended interpretation
\begin{equation}\label{eq:counting-base-clsig}
\bigwedge_{1\leq\ii\leq\nm}\forall\xx\existseq{\nMM\ii}\yy
(\smm\ii(\xx,\yy)\land\xx\neq\yy).
\end{equation}
Whenever $\ClSig\SigS\sms$ is a classified signature, we will denote
\[
  \nM = 1 +\max\set{\nMM1,\nMM2,\dots,\nMM\nm}
\].
\end{definition}
Note that a classified signature may be exponentially larger than its intended
interpretation, since the counting quantifiers are coded succinctly in binary.
Note that $\nM$ is polynomial with respect to the size of the classified
signature.
\begin{definition}
A structure $\StrA$ over the classified signature $\ClSig\SigS\sms$ is a
structure over $\SigS$ having cardinality at least $\nM$ and satisfying the
intended interpretation~\cref{eq:counting-base-clsig} of the message symbols.
Note that $\StrA$ must have cardinality at least $\nM$.
\end{definition}
\begin{definition}
The (finite) classified satisfiability problem for $\vFoc2$ is the following:
given a classified signature $\ClSig\SigS\sms$ and a quantifier-free
$\vFocF2\SigS$-formula $\falp(\xx,\yy)$, is there a (finite)
$\ClSig\SigS\sms$-structure $\StrA$ satisfying~\ref{eq:counting-base}.
Denote the classified satisfiability problem by $\ClSat{\vFoc2}$ and its finite
version by $\FinClSat{\vFoc2}$.
\end{definition}

Similarly to~\Cref{ch:twovar} that a type instance $\TI\subseteq\TIS\SigS$
over $\ClSig\SigS\sms$ is a set of $2$-types that is closed under inversion.
If $\StrA$ is a $\ClSig\SigS\sms$-structure, its type instance is:
\[
  \TIS\StrA = \setbd{\tpIab\StrA\ea\eb}{\ea\in\domA,\eb\in\domA\sub\set\ea}.
\]

\begin{definition}
The (finite) type realizability problem for $\vFoc2$ is the following:
given a classified signature $\ClSig\SigS\sms$ and a type instance $\TI$ over
$\ClSig\SigS\sms$, is there a (finite) model for $\TI$.
Denote the type realizability problem for $\vFoc2$ by $\Real{\vFoc2}$
and its finite version by $\FinReal{\vFoc2}$.
\end{definition}

\begin{remark}
\[
\FinASat{\vFoc2} \red\cNExpTime \FinAReal{\vFoc2}.
\]
\end{remark}
\begin{proof}
Let $\SigS$ be a predicate signature and let $\fphi$ be a $\SigS$-sentence.
Generate $\prtr\fphi$ of the form~\cref{eq:counting-base} in deterministic
polynomial time and extract a classified signature $\ClSig\SigS\sms$ out of it.
Note that $\nM$ is exponentially bounded by the length of $\fphi$.
First check if some $\SigS$-structure of cardinality less than $\nM$ satisfies
$\fphi$. This can be done in nondeterministic exponential time by guessing such
a structure, if it exists.
If such a structure exists, map the input $(\SigS,\fphi)$ to a fixed positive
instance of the (finite) type realizability problem.
Otherwise $\fphi$ is (finitely) satisfiable iff $\prtr\fphi$ is finitely
satisfiable.
Consider $\TI^\falp \subseteq \TpT\SigS$ to be the set of those \twotypes/ that
are consistent with $\falp(\xx,\yy)$ and the indended interpretation for
classified signature~\cref{eq:counting-base-clsig}, where $\falp$ is the
universal part of $\prtr\fphi$.
Then $\ClSig\SigS\sms$-structure $\StrA$ is a classified model for
$\falp(\xx,\yy)$ iff $\TIS\StrA \subseteq \TI^\falp$.
So in this case $\fphi$ is (finitely) satisfiable iff some
type instance $\TI\subseteq\TI^\falp$ is (finitely) realizable.
\end{proof}

%\section{Message types and Silent types}
\begin{definition}
Let $\TI$ be a type instance over the $\vFoc2$-classified signature
$\ClSig\SigS\sms$ and let $\tpt\in\TI$ be any \twotype/.
Then:
\begin{itemize}
  \item 
  $\tpt$ is an \emph{$\xx$-message type} if some $\smm\ii\in\sms$
  has $\smm\ii(\xx,\yy)\in\tpt$.
  \item
  $\tpt$ is an \emph{$\xx$-silent type} if it is not an $\xx$-message type.
  \item
  $\tpt$ is a \emph{$\yy$-message type} if some $\smm\ii\in\sms$
  has $\smm\ii(\yy,\xx)\in\tpt$.
  \item
  $\tpt$ is a \emph{$\yy$-silent type} if it is not a $\yy$-message type.
\end{itemize}
We use the following notation to denote particular sets of types:
\[
\TIall.
\]
The left index denotes the $\xx$-kind of a type;
the right index denotes the $\yy$-kind.
The symbol $\arn$ means no restriction, the symbol $\arm$ means message and the
symbol $\ars$ means silent.
For example $\TIann = \TI$ is the set of all types and $\TIasm$ is the set of
$\xx$-silent $\yy$-message types.
The set of \emph{silent types} is $\TIass$.
The set of \emph{invertable message types} is $\TIamm$.
Clearly the sets $\TIann, \TIass$ and $\TIamm$ are closed under inversion;
the remaining sets come in inversion pairs, for example
$\TIasm = \setbd{\inv\tpt}{\tpt\in\TIams}$ and 
$\TIams = \setbd{\inv\tpt}{\tpt\in\TIasm}$.
\end{definition}
