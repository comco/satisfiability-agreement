% Counting base case
Recall from~\Cref{sec:scott-nf} about normal forms that every $\vFoc2$-sentence
$\fphi$ can be reduced in deterministic polynomial time to a sentence
\[
  \forall\xx\forall\yy(\falpp0(\xx,\yy) \lor \xx=\yy) \land
  \bigwedge_{1 \leq \ii \leq \nm} \forall\xx\existseq{\nMM\ii}\yy
  (\falpp\ii(\xx,\yy) \land \xx\neq\yy),
\]
where $\nm\geq1$, $\nMM\ii\geq1$, all the formulas $\falpp\ii$ are
quantifier-free and use at most linearly many new unary and binary predicate
symbols.
Let $\nM = \max\set{\nMM1,\nMM2,\dots,\nMM\nm}$.
The semantic connection between $\fphi$ and $\prtr\fphi$ is that they are
essentially equisatisfiable.
More precisely, every model for $\fphi$ of cardinality more than $\nM$ can be
enriched to a model for $\prtr\fphi$ and every model for $\prtr\fphi$ of
cardinality at least $\nM$ is a model for $\fphi$.
We refer to $\falpp0$ as the universal part of $\fphi$ and to $\falpp\ii$ for
$\ii\in[1,\nm]$ as the existenial parts of $\fphi$.

We replace the existential parts of a formula in normal form with fresh binary
predicate symbols $\smm\ii$ (the message symbols) for $\ii\in[1,\nm]$
having intended interpretation
$\forall\xx\forall\yy(\smm\ii(\xx,\yy)\lequ\falpp\ii(\xx,\yy))$.
Hence any $\vFoc2$-sentence can be transformed in deterministic polynomial time
into the form:
\begin{equation}\label{eq:counting-base}
\forall\xx\forall\yy(\falp(\xx,\yy)\lor\xx=\yy) \land
\bigwedge_{1\leq\ii\leq\nm} \forall\xx\existseq{\nMM\ii}\yy
(\smm\ii(\xx,\yy)\land\xx\neq\yy),
\end{equation}
where the universal part $\falp$ is quantifier-free and over an extended
signature.

\begin{definition}
A \emph{classified signature} $\ClSig\SigS\sms$ for the two-variable logic with
counting quantifiers $\vFoc2$ is a predicate signature $\SigS$ together with a
nonempty sequence
\[
\sms = \underbrace{\smm1\smm1\dots\smm1}_{\nMM1}
\underbrace{\smm2\smm2\dots\smm2}_{\nMM2}
\dots
\underbrace{\smm\nm\smm\nm\dots\smm\nm}_{\nMM\nm}
\]
($\nMM\ii\geq1$) of distinct binary predicate symbols $\smm\ii$ from $\SigS$
having intended interpretation
\begin{equation}\label{eq:counting-base-clsig}
\bigwedge_{1\leq\ii\leq\nm}\forall\xx\existseq{\nMM\ii}\yy
(\smm\ii(\xx,\yy)\land\xx\neq\yy).
\end{equation}
Whenever $\ClSig\SigS\sms$ is a classified signature, we will denote
\[
  \nM = \max\set{\nMM1,\nMM2,\dots,\nMM\nm}.
\]
\end{definition}
Note that a classified signature may be exponentially larger than its intended
interpretation, since the counting quantifiers are coded succinctly in binary.
Note that $\nM$ is polynomial with respect to the size of the classified
signature.
\begin{definition}
A structure $\StrA$ over the classified signature $\ClSig\SigS\sms$ is a
structure over $\SigS$ satisfying the intended
interpretation~\cref{eq:counting-base-clsig} of the message symbols.
Note that $\StrA$ must have cardinality greater than $\nM$.
\end{definition}
\begin{definition}
The (finite) classified satisfiability problem for $\vFoc2$ is the following:
given a classified signature $\ClSig\SigS\sms$ and a quantifier-free
$\vFocF2\SigS$-formula $\falp(\xx,\yy)$, is there a (finite)
$\ClSig\SigS\sms$-structure $\StrA$ satisfying~\ref{eq:counting-base}.
Denote the classified satisfiability problem by $\ClSat{\vFoc2}$ and its finite
version by $\FinClSat{\vFoc2}$.
\end{definition}

Similarly to~\Cref{ch:twovar}, we define a type instance
$\TI\subseteq\TIS\SigS$ over $\ClSig\SigS\sms$ as a set of $2$-types that is
closed under inversion.
If $\StrA$ is a $\ClSig\SigS\sms$-structure, its type instance again is:
\[
  \TIS\StrA = \setbd{\tpIab\StrA\ea\eb}{\ea\in\domA,\eb\in\domA\sub\set\ea}.
\]

\begin{definition}
The (finite) type realizability problem for $\vFoc2$ is the following:
given a classified signature $\ClSig\SigS\sms$ and a type instance $\TI$ over
$\ClSig\SigS\sms$, is there a (finite) model for $\TI$.
Denote the type realizability problem for $\vFoc2$ by $\Real{\vFoc2}$
and its finite version by $\FinReal{\vFoc2}$.
\end{definition}

\begin{remark}
\[
\FinASat{\vFoc2} \red\cNExpTime \FinAReal{\vFoc2}.
\]
\end{remark}
\begin{proof}
Let $\SigS$ be a predicate signature and let $\fphi$ be a $\SigS$-sentence.
Generate $\prtr\fphi$ of the form~\cref{eq:counting-base} in deterministic
polynomial time and extract a classified signature $\ClSig\SigS\sms$ out of it.
Note that $\nM$ is exponentially bounded in the length of $\fphi$.
First check if some $\SigS$-structure of cardinality less than or equal to $\nM$
satisfies $\fphi$. This can be done in nondeterministic exponential time by
guessing such a structure if it exists.
If such a structure exists, map the input $(\SigS,\fphi)$ to a fixed positive
instance of the (finite) type realizability problem.
Otherwise $\fphi$ is (finitely) satisfiable iff $\prtr\fphi$ is finitely
satisfiable.
Consider $\TI^\falp \subseteq \TpT\SigS$ to be the set of those \twotypes/ that
are consistent with $\falp(\xx,\yy)$ and the indended interpretation for
classified signature~\cref{eq:counting-base-clsig}, where $\falp$ is the
universal part of $\prtr\fphi$.
Then $\ClSig\SigS\sms$-structure $\StrA$ is a classified model for
$\falp(\xx,\yy)$ iff $\TIS\StrA \subseteq \TI^\falp$.
So in this case $\fphi$ is (finitely) satisfiable iff some
type instance $\TI\subseteq\TI^\falp$ is (finitely) realizable.
\end{proof}

\begin{definition}
Let $\TI$ be a type instance over the $\vFoc2$-classified signature
$\ClSig\SigS\sms$ and let $\tpt\in\TI$ be any \twotype/.
Then:
\begin{itemize}
  \item 
  $\tpt$ is an \emph{$\xx$-message type} if some $\smm\ii\in\sms$
  has $\smm\ii(\xx,\yy)\in\tpt$.
  \item
  $\tpt$ is an \emph{$\xx$-silent type} if it is not an $\xx$-message type.
  \item
  $\tpt$ is a \emph{$\yy$-message type} if some $\smm\ii\in\sms$
  has $\smm\ii(\yy,\xx)\in\tpt$.
  \item
  $\tpt$ is a \emph{$\yy$-silent type} if it is not a $\yy$-message type.
\end{itemize}

We use the following notation to denote particular sets of types:
\[
  \TIall.
\]
The left index denotes the $\xx$-kind of a type;
the right index denotes the $\yy$-kind.
The symbol $\arn$ means no restriction, the symbol $\arm$ means message and the
symbol $\ars$ means silent.
For example $\TIann = \TI$ is the set of all types and $\TIasm$ is the set of
$\xx$-silent $\yy$-message types.
The set of \emph{silent types} is $\TIass$.
The set of \emph{message types} is $\TI\sub\TIass = \TIanm\cup\TIamn$.
The set of \emph{invertable message types} is $\TIamm$.
Clearly the sets $\TIann, \TIass$ and $\TIamm$ are closed under inversion;
the remaining sets come in inversion pairs, for example
$\TIasm = \setbd{\inv\tpt}{\tpt\in\TIams}$ and 
$\TIams = \setbd{\inv\tpt}{\tpt\in\TIasm}$.
\end{definition}
We are now going to perform a sequence of reductions of the general type
realizability problem to type realizability over a class of very well-behaved
special structures, which then we can attack directly, in a way similar to what
we did in~\Cref{ch:twovar}.
% TODO: Clarify or remove
A special structure is a structure that is:
\begin{enumerate}
  \item Chromatic
  \item Has at least two kings
  \item Has silent types
  \item Kings send invertible message types --- so that we don't multiply them
  \item Message distinguished --- for star-types
  \item Origin distinguished --- crucial for the invertible simulation lemma.
\end{enumerate}

\begin{definition}
Let $\TI$ be a type instance over $\ClSig\SigS\sms$ and let $\nZ\geq1$.
A model $\StrA$ for $\TI$ is \emph{$\nZ$-differentiated}
if each \onetype/ $\tpp\in\TP\TI$ realized in $\StrA$ is realized by either a
unique element or more than $\nZ$ elements in $\StrA$.
\end{definition}

\begin{definition}
Let $\SigS$ be a predicate signature and let $\nZ\geq1$ be a parameter.
Define $\df\SigS\nZ = \SigS + \seq{\spp1,\spp2,\dots,\spp\nZ}$ as an
enrichment of $\SigS$ with $\nZ$ new unary predicate symbols.
For any $\ii\in[1,\nZ]$, let $\sPe\ii(\xx)$ be the following set of literals
over $\df\SigS\nZ$:
\[
\sPe\ii(\xx) =
\set{\spp\ii(\xx)}\cup\setbd{\lnot\spp\kk(\xx)}{\kk\in[1,\nZ]\sub\set\ii}.
\]
That is, $\sPe\ii(\xx)$ \emph{satisfies just the $\ii$-th predicate symbol}.

For any \onetype/ $\tpp$ over $\SigS$ and for $\ii\in[1,\nZ]$,
define $\mr\tpp\ii = \tpp\cup\sPe\ii(\xx)$ as a \onetype/ over $\df\SigS\nZ$.
For any \twotype/ $\tpt$ over $\SigS$ and for $\ii,\jj\in[1,\nZ]$,
define $\mrr\tpt\ii\jj = \tpt\cup\sPe\ii(\xx)\cup\sPe\jj(\yy)$ as a \twotype/
over $\df\SigS\nZ$.

Consider any classified signature $\ClSig\SigS\sms$ over $\SigS$.
For any type instance $\TI$ over $\ClSig\SigS\sms$,
define $\df\TI\nZ = \setbd{\mrr\tpt\ii\jj}{\ii,\jj\in[1,\nZ]}$ as a type
instance over $\ClSig{\df\SigS\nZ}\sms$.

A \emph{$\nZ$ promotion} $\TIpr\subseteq\df\TI\nZ$ of the type instance $\TI$ is
a type instance over $\ClSig{\df\SigS\nZ}\sms$ such that for each $\tpt\in\TI$
there are some $\ii,\jj\in[1,\nZ]$ such that $\mrr\tpt\ii\jj\in\TIpr$.
\end{definition}

\begin{definition}
Let $\TI$ be a type instance over $\ClSig\SigS\sms$.
A model $\StrA$ for $\TI$ is \emph{chromatic}
if the following conditions are satisfied:
\begin{enumerate}
  \item If $\ea\in\domA$, $\eb\in\domA\sub\set\ea$ and
  $\tpIab\StrA\ea\eb\in\TIamm$ is an invertible message type,
  then $\tpIa\StrA\ea\neq\tpIa\StrA\eb$, that is invertible message types
  connect distict \onetypes/.
  \item If $\ea\in\domA$, $\eb\in\domA\sub\set\ea$,
  $\ec\in\domA\sub\set{\ea,\eb}$ and $\tpIab\StrA\ea\eb, 
  \tpIab\StrA\eb\ec\in\TIamm$, then $\tpIa\StrA\ea\neq\tpIa\StrA\ec$.
\end{enumerate}
That is, no chain of one or two invertible message types connects elements of
the same \onetype/.
\end{definition}
\begin{remark}[Chromaticity]\label{rem:prop-chroma}
The type instance $\TI$ over $\ClSig\SigS\sms$ has a (finite) model
iff some $\nZZ1$-promotion $\TIprr1$ of $\TI$ has a (finite)
\emph{chromatic model}, where $\nZZ1 = \nM^2+1$.
\end{remark}
\begin{proof}
First suppose that $\StrA$ is a (finite) model for $\TI$.
We will define a chromatic enrichment $\StrAp$ of $\StrA$ to a
$\ClSig{\df\SigS{\nZZ1}}\sms$-structure such that $\TIprr1 = \TIS\StrAp$ is a
$\nZZ1$-promotion of $\TI$.
Consider the graph over $\domA$ where two elements are connected iff there is a
chain of one or two invertible message types connecting them.
This is an undirected graph of maximal degree $\nM+\nM(\nM-1)=\nM^2$, so it can
be colored using $\nM^2+1$ colors so that no two adjacent vertices have the same
color. We use the $\nZZ1$ fresh unary symbols of $\df\SigS\nZ$ to color the
vertices appropriately. More precisely, let $c : \domA\to[1,\nZ]$ be a vertex
coloring of the graph. Then define the enrichment as
$\tpIa\StrAp\ea = \mr{\tpIa\StrA\ea}{c(\ea)}$.

Next the reduct of any model of any $\nZZ1$-promotion $\TIprr1$ of $\TI$ to a
$\SigS$-structure is a model for $\TI$ by the promotion condition.
\end{proof}

\begin{remark}\label{rem:prop-twokings}
The type instance $\TI$ over $\ClSig\SigS\sms$ has a (finite) model
iff some $\nZZ2$-promotion $\TIprr2$
of some $\nZZ1$-promotion $\TIprr1$ 
of $\TI$ has a (finite) chromatic model \emph{with at least two kings}, that is
$\card{\TK{\TIprr2}}\geq2$,
where $\nZZ1 = \nM^2+1$ and $\nZZ2 = 3$.
\end{remark}
\begin{proof}
First suppose that $\TI$ has a (finite) model.
By~\Cref{rem:prop-chroma},
some $\nZZ1$-promotion $\TIprr1$
of $\TI$ has a (finite) chromatic model $\StrA$.
Let $\ea\in\domA$ and $\eb\in\domA\sub\set\ea$ be two arbitrary elements of
$\StrA$ (recall that $\StrA$ has cardinality at least $2$).
We enrich $\StrA$ to an $\StrAp$ by making $\ea$ and $\eb$ kings:
$\tpIa\StrAp\ea = \mr{\tpIa\StrA\ea}1$,
$\tpIa\StrAp\eb = \mr{\tpIa\StrA\eb}2$
and $\tpIa\StrAp\ec = \mr{\tpIa\StrA\ec}3$ for $\ec\in\domA\sub\set{\ea,\eb}$.
Then indeed $\TIprr2 = \TIS\StrAp$ is a $\nZZ2$-promotion of $\TIprr1$.
The enrichment remains chromatic, since we only alter the \onetypes/ of elements
by differentiating them; we do not change the kind (message or silent) of the
two-types between them, so no chain of one or two invertible message types
connecting elements of the same \onetype/ could be created.

Next, the reduct of any model $\StrAp$ of any $\nZZ2$-promotion $\TIprr2$ of
$\TIprr1$ to a $\SigS$-structure is a model for $\TI$ by the promotion
condition.
\end{proof}

\begin{definition}
Let $\TI$ be a type instance over $\ClSig\SigS\sms$ and let $\nZ\geq1$.
A model $\StrA$ for $\TI$ is \emph{$\nZ$-differentiated} if any \onetype/
$\tpp\in\TP\TI$ realized in $\StrA$ is realized by either a unique or more than
$\nZ$ elements.
\end{definition}

\begin{definition}
A type instance $\TI$ \emph{has silent types} if whenever $\tpp,\tppp\in\TW\TI$
are worker types, then there is some silent $\tpt\in\TIass$ connecting $\tpp$
and $\tppp$.
\end{definition}
\begin{remark}\label{rem:silent}
Let $\TI$ be a type instance over $\ClSig\SigS\sms$ and let $\nZ\geq2\nM+2$.
Suppose that $\StrA$ is a $\nZ$-differentiated model for $\TI$.
Then $\TI$ has silent types.
\end{remark}
\begin{proof}
Note that there are more than $\nZ$ elements realizing $\tpp$ and there are
more than $\nZ$ elements realizing $\tppp$ in $\StrA$, since otherwise $\tpp$ or
$\tppp$ would be king types.

First suppose that $\tpp=\tppp$.
Let $\domB\subseteq\domA$ be any set of $(\nZ+1)$ elements realizing $\tpp$ in
$\StrA$.
We claim that some pair of elements from $\domB$ is connected by a silent type,
that is $\ea\in\domB$ and $\eb\in\domB\sub\set\ea$ have
$\tpIab\StrA\ea\eb\in\TIass$.
Suppose not towards a contradiction.
Then for every $\ea\in\domB$ and $\eb\in\domB\sub\set\ea$ we would have
$\tpIab\StrA\ea\eb\in\TIamn\cup\TIanm$.
Note that any element can sent at most $\nM$ $\xx$-message types.
Hence the number of message types between elements from $\domB$ is
$\nE\leq(\nZ+1)\nM$.
But the total number of edges between elements from $\domB$ is $(\nZ+1)\nZ/2 >
\nE$ --- a contradiction.

Next suppose that $\tpp\neq\tppp$.
Let $\domB\subset\domA$ be a set of $(\nZ+1)$ elements realizing $\tpp$ and let
$\domC\subset\domA$ be a set of $(\nZ+1)$ elements realizing $\tppp$.
Suppose towards a contradiction that no $\eb\in\domB$ and $\ec\in\domC$ have
$\tpIab\StrA\eb\ec\in\TIass$.
Hence the number of message types between elements from $\domB$ and elements
from $\domC$ is $\nE\leq2(\nZ+1)\nM$.
But the total number of edges between elements from $\domB$ and $\domC$ is
$(\nZ+1)^2 > \nE$ --- a contradiction.
\end{proof}

\begin{remark}[Silent types]\label{rem:prop-silent}
The type instance $\TI$ over $\ClSig\SigS\sms$ has a (finite) model
iff some $\nZZ3$-promotion $\TIprr3$
of some $\nZZ2$-promotion $\TIprr2$ 
of some $\nZZ1$-promotion $\TIprr1$ 
of $\TI$ has a (finite) chromatic model with at least two kings and
\emph{with silent types},
where $\nZZ1 = \nM^2+1$, $\nZZ2 = 3$ and $\nZZ3 = 2\nM+2$.
\end{remark}
\begin{proof}
First, suppose that $\TI$ has a (finite) model.
By~\Cref{rem:prop-chroma}, some
$\nZZ2$-promotion $\TIprr2$
of some $\nZZ1$-promotion $\TIprr1$ of $\TI$ has a (finite) chromatic model
$\StrA$ with at least two kings.
We define an enrichment $\StrAp$ of $\StrA$ to a chromatic
$\nZZ3$-differentiated structure which by~\Cref{rem:silent} has silent types,
such that $\TIprr3 = \TIS\StrA$ is a $\nZZ3$-promotion of $\TIprr2$.
Indeed, consider the graph over $\domA$ where two elements are connected iff
they have the same \onetype/ $\tpp$ and if this $\tpp$ is realized by at most
$\nZZ3$ elements in $\domA$.
This graph is an undirected graph with maximal degree $(\nZZ3-1)$, so it can be
colored using $\nZZ3$ colors such that no two adjacent vertices have the same
color.
Let $c : \domA\to[1,\nZZ3]$ be such coloring.
Define $\tpIa\StrAp\ea = \mr{\tpIa\StrA\ea}{c(\ea)}$.
It is evident that the resulting structure is $\nZZ3$-differentiated.
The enrichment remains chromatic and the kings remain kings,
since we only alter the \onetypes/ of elements by differentiating them.

 Next, the reduct of any model $\StrAp$ of any $\nZZ3$-promotion $\TIprr3$ of
$\TIprr2$ to a $\SigS$-structure is clearly a model for $\TI$ by the promotion
condition.
\end{proof}

\begin{remark}[At least $2$ kings]\label{rem:prop-twokings}
The type instance $\TI$ over $\ClSig\SigS\sms$ has a (finite) model
iff some $\nZZ3$-promotion $\TIprr3$
of some $\nZZ2$-promotion $\TIprr2$
of some $\nZZ1$-promotion $\TIprr1$ 
of $\TI$ has a (finite) chromatic model with silent types
\emph{that has at least $2$ kings},
where $\nZZ1 = \nM^2+1$, $\nZZ2 = 2\nM+2$ and $\nZZ3 = 3$.
\end{remark}
\begin{proof}
First suppose that $\TI$ has a (finite) model.
By~\Cref{rem:prop-silent}
some $\nZZ2$-promotion $\TIprr2$
of some $\nZZ1$-promotion $\TIprr1$
of $\TI$ has a (finite) chromatic model $\StrA$ with silent types.
Let $\ea\in\domA$ and $\eb\in\domA\sub\set\ea$ be two arbitrary elements of
$\StrA$. We enrich $\StrA$ to $\StrAp$ by making $\ea$ and $\eb$ kings:
$\tpIa\StrAp\ea = \mr{\tpIa\StrA\ea}1$,
$\tpIa\StrAp\eb = \mr{\tpIa\StrA\eb}2$
and $\tpIa\StrAp\ec = \mr{\tpIa\StrA\ec}3$ for $\ec\in\domA\sub\set{\ea,\eb}$.
Then indeed $\TIprr3 = \TIS\StrAp$ is a $\nZZ3$-promotion of $\TIprr2$.
We know that $\StrAp$ remains chromatic since we did not alter the kind of
\twotypes/ between elements.
\end{proof}

\begin{definition}
Let $\TI$ be a type instance over $\ClSig\SigS\sms$ such that
$\card{\TK\TI}\geq2$ and let $\nK = \card{\TK\TI}-1$.
Define the enrichment
$\SigSs = \SigS + \seq{\skm0,\skm1}$
of $\SigS$ with two new message symbols,
let $\skms = \sms + \seq{\skm0,\skm1}$
and consider the classified signature $\ClSig\SigSs\skms$.
For any \twotype/ $\tpt$ over $\SigS$
and for $\sts,\stsp\subseteq\set{0,1}$,
define the \twotype/ $\mrr\tpt\sts\stsp$ over $\SigSs$ as:
\begin{align*}
\mrr\tpt\sts\stsp = \tpt
\cup\setbd{\skm\ii(\xx,\yy)}{\ii\in\sts}
\cup\setbd{\lnot\skm\ii(\xx,\yy)}{\ii\in\set{0,1}\sub\sts}\cup \\
\cup\setbd{\skm\ii(\yy,\xx)}{\ii\in\stsp}
\cup\setbd{\lnot\skm\ii(\yy,\xx)}{\ii\in\set{0,1}\sub\stsp}\cup \\
\cup\setbd{\lnot\skm\ii(\xx,\xx)}{\ii\in\set{0,1}}
\cup\setbd{\lnot\skm\ii(\yy,\yy)}{\ii\in\set{0,1}}.
\end{align*}
That is, $\mrr\tpt\sts\stsp$ is an $\xx$-message type exactly for the fresh
message symbols from $\sts$ and a $\yy$-message type exactly for the fresh
message symbols from $\stsp$.

Define the set of \twotypes/ $\TI'$ over $\SigSs$ as:
\[
\TI' = \setbd{\mrr\tpt\sts\stsp}{\tpt\in\TI;\sts,\stsp\subseteq\set{0,1}}.
\]

A \emph{king promotion} $\TIpr\subseteq\TI'$ for $\TI$ is a type instance over
$\ClSig\SigSs\skms$ such that for each $\tpt\in\TI$ there are some
$\sts,\stsp\subseteq\set{0,1}$ such that $\mrr\tpt\sts\stsp\in\TIpr$.
\end{definition}

\begin{definition}
Let $\SigS$ be a predicate signature and let $\nW\geq1$ be a parameter.
Define $\enII\SigS\nW = \SigS + \seq{\sqq1,\sqq2,\dots,\sqq\nW}$ as an
enrichment of $\SigS$ with $\nW$ new binary predicate symbols.
For any $\ii\in[1,\nW]$, define $\sQ\ii(\xx,\yy)$ as the following set of
literals over $\enII\SigS\nW$:
\[
\sQ\ii(\xx,\yy) = \set{\sqq\ii(\xx,\yy)} \cup
\setbd{\lnot\sqq\jj(\xx,\yy)}{\jj\in[1,\nW]\sub\set\ii} \cup
\setbd{\lnot\sqq\kk(\xx,\xx)}{\kk\in[1,\nW]}.
\]
Define $\sQ\bot(\xx,\yy)$ as the following set of literals over $\enII\SigS\nW$:
\[
\sQ\bot(\xx,\yy) = \setbd{\lnot\sqq\jj(\xx,\yy)}{\jj\in[1,\nW]} \cup
\setbd{\lnot\sqq\kk(\xx,\xx)}{\kk\in[1,\nW]}.
\]
For any \twotype/ $\tpt$ over $\SigS$ and for $\ii,\jj\in[1,\nW]\cup\set\bot$,
define the \twotype/ $\mrr\tpt\ii\jj$ over $\enII\SigS\nW$ as follows:
\[
\mrr\tpt\ii\jj = \tpt \cup \sQ\ii(\xx,\yy) \cup \sQ\jj(\yy,\xx).
\]
Let $\ClSig\SigS\sms$ be any classified signature over $\SigS$
and let $\TI$ be any type instance over $\ClSig\SigS\sms$.
Define the set of \twotypes/ $\TI'$ over $\enII\SigS\nW$ as follows:
\[
\TI' = \setbd{\mrr\tpt\ii\jj}{\tpt\in\TI,\ii,\jj\in[1,\nW]\cup\set\bot}.
\]

A \emph{$\nW$-\twotype/-promotion} $\TIpr\subseteq\TI'$ for $\TI$ is a type
instance over $\ClSig{\enII\SigS\nW}\sms$ such that for each $\tpt\in\TI$ there
are some $\ii,\jj\in[1,\nW]\cup\set\bot$ such that $\mrr\tpt\ii\jj\in\TIpr$.

A \emph{king-promotion} $\TIpr\subseteq\TI'$ for $\TI$ is a type instance over
$\ClSig{\enII\SigS\nW}{\sms+\seq{\sqq1}}$ for $\nW = 1$.
\end{definition}

\begin{remark}[Kings sending invertible messages]\label{rem:prop-kingsinv}
The type instance $\TI$ over $\ClSig\SigS\sms$ has a (finite) model
iff some king-promotion $\TIprr4$
of some $\nZZ3$-promotion $\TIprr3$
of some $\nZZ2$-promotion $\TIprr2$
of some $\nZZ1$-promotion $\TIprr1$ 
of $\TI$ has a (finite) chromatic model with silent types
\emph{where each king sends an invertible message type},
where $\nZZ1 = \nM^2+1$, $\nZZ2 = 2\nM+2$ and $\nZZ3 = 3$.
\end{remark}
\begin{proof}
First suppose that $\TI$ has a (finite) model.
By~\Cref{rem:prop-twokings},
some $\nZZ3$-promotion $\TIprr3$
of some $\nZZ2$-promotion $\TIprr2$
of some $\nZZ1$-promotion $\TIprr1$
of $\TI$ has a (finite) chromatic model $\StrA$ with silent types and at least
two kings.
Consider the classified signature $\ClSig\SigSp{\sms+\seq{\sqq1}}$ for the king
promotion of $\TIprr3$.
We will enrich $\StrA$ into a $\ClSig\SigSp{\sms+\seq{\sqq1}}$-structure
$\StrAp$ such that $\TIprr4 = \TIS\StrAp$ is a king promotion of $\TIprr3$.
Let $\domC = \set{\ecc1,\ecc2,\dots,\ecc\nn}\subseteq\domA$ be an enumeration
of the kings in $\StrA$, where $\nn=\card{\TK{\TIprr3}}$ is the number of kings.
Define the function $s : \domA\to\domA$ as follows:
\begin{itemize}
  \item $s(\ecc\ii) = \ecc{\ii+1}$ if $\ii\in[1,\nn-1]$
  \item $s(\ecc\nn) = \ecc1$
  \item if $\ea\in\domA\sub\domC$, let $s(\ea)\in\domA\sub\set\ea$
  be any element such that $\tpIab\StrA\ea{s(\ea)}$ is an $\xx$-message type
\end{itemize}
Define the enrichment as follows:
let $\ea\in\domA$, $\eb\in\domA\sub\set\ea$ and let $\tpt=\tpIab\StrA\ea\eb$.
Then:
\begin{itemize}
  \item $\tpIab\StrAp\ea\eb = \mrr\tpt\bot\bot$
  if $c(\ea)\neq\eb$ and $c(\eb)\neq\ea$
  \item $\tpIab\StrAp\ea\eb = \mrr\tpt1\bot$
  if $c(\ea)=\eb$ and $c(\eb)\neq\ea$
  \item $\tpIab\StrAp\ea\eb = \mrr\tpt\bot1$
  if $c(\ea)\neq\eb$ and $c(\eb)=\ea$
  \item $\tpIab\StrAp\ea\eb = \mrr\tpt11$
  if $c(\ea)=\eb$ and $c(\eb)=\ea$
\end{itemize}
It is routine to check that this assignment is sound and that $\StrAp$ is a
$\ClSig\SigSp{\sms+\seq{\sqq1}}$-structure.
This model remains chromatic and has silent types since we did not alter the
kind (message or silent) of any \twotype/ between a worker element and another
element.

Next, if $\StrAp$ is a model for any king-promotion $\TIprr4$, then the reduct
of $\StrAp$ to a $\SigS$-structure is a model for $\TI$ by the promotion
condition.
\end{proof}

\begin{definition}
Let $\TI$ be a type instance over $\ClSig\SigS\sms$.
A model $\StrA$ for $\TI$ has \emph{differentiated message types}
if whenever $\ea\in\domA$, $\eb\in\domA\sub\set\ea$
and $\tpt = \tpIab\StrA\ea\eb\in\TIamn$ is an $\xx$-message type,
no $\ec\in\domA\sub\set{\ea,\eb}$ has $\tpIab\StrA\ea\ec = \tpt$.
That is, an element is allowed to send any $\xx$-message type at most once.
\end{definition}
\begin{remark}[Differentiated message types]\label{rem:prop-diff-msg-tp}
The type instance $\TI$ over $\ClSig\SigS\sms$ has a (finite) model
iff some $\nWW5$-\twotype/-promotion $\TIprr5$
of some king-promotion $\TIprr4$
of some $\nZZ3$-promotion $\TIprr3$
of some $\nZZ2$-promotion $\TIprr2$
of some $\nZZ1$-promotion $\TIprr1$ 
of $\TI$ has a (finite) chromatic model with silent types
and \emph{with differentiated message types} 
where each king sends an invertible message type,
where $\nZZ1 = \nM^2+1$, $\nZZ2 = 2\nM+2$, $\nZZ3 = 3$ and $\nWW5 = \nM$.
\end{remark}
\begin{proof}
First suppose that $\TI$ has a (finite) model.
By~\Cref{rem:prop-kingsinv},
some king-promotion $\TIprr4$
of some $\nZZ3$-promotion $\TIprr3$
of some $\nZZ2$-promotion $\TIprr2$
of some $\nZZ1$-promotion $\TIprr1$
of $\TI$ has a (finite) chromatic model $\StrA$ with silent types where each
king sends an invertible message type.
Note that since $\StrA$ is chromatic, no $\ea\in\domA$ sends the same invertible
message type twice.
Hence we only have to take care of differentiating the $\xx$-message
$\yy$-silent types sent by any element.
But any element sends any $\xx$-message type at most $\nM$ times, so we can use
the $\nWW5 = \nM$ fresh binary predicate symbols to differentiate them.
The model stays chromatic, still has silent types, the kings are preserved and
each king still sends an inveritble message type
since we did not alter the kind of types between elements and the \onetypes/ of
elements.
\end{proof}

\begin{definition}
Let $\StrA$ be a model for the type instance $\TI$.
An element $\ea\in\domA$ is an \emph{origin} for a \twotype/ $\tpt\in\TI$ if
some $\eb\in\domA\sub\set\ea$ has $\tpIab\StrA\ea\eb = \tpt$.
The set of origins of $\tpt$ in $\StrA$ is $\atx\StrA\tpt$.
Note that this set is nonempty.
\end{definition}
\begin{definition}
Let $\TI$ be a type instance over $\ClSig\SigS\sms$ and let $\nW\geq1$.
A structure $\StrA$ for $\TI$ is \emph{$\nW$-origin-differentiated}
if any invertible message type $\tpt\in\TIamm$ has either one or more than $\nW$
origins.
An invertible message type that is realized once in $\StrA$ is \emph{rare}.
The remaining invertible message types are \emph{common}.
An origin of a rare message type is a \emph{rare element} and the remaining
elements are \emph{common elements}.
\end{definition}

\begin{remark}[Origin-differentiated models]\label{rem:diff-msg-tp}
Let $\nWW6\geq1$.
The type instance $\TI$ has a (finite) model
iff some $\nWW6$-\twotype/-promotion $\TIprr6$
of some $\nWW5$-\twotype/-promotion $\TIprr5$
of some king-promotion $\TIprr4$
of some $\nZZ3$-promotion $\TIprr3$
of some $\nZZ2$-promotion $\TIprr2$
of some $\nZZ1$-promotion $\TIprr1$ 
of $\TI$ has a (finite) \emph{$\nWW6$-origin-differentiated} chromatic model
with silent types and with differentiated message types 
where each king sends an invertible message type,
where $\nZZ1 = \nM^2+1$, $\nZZ2 = 2\nM+2$, $\nZZ3 = 3$ and $\nWW5 = \nM$.
\end{remark}
\begin{proof}
Suppose that $\TI$ has a (finite) model, so by~\Cref{rem:prop-diff-msg-tp}
some $\nWW5$-\twotype/-promotion $\TIprr5$
of some king-promotion $\TIprr4$
of some $\nZZ3$-promotion $\TIprr3$
of some $\nZZ2$-promotion $\TIprr2$
of some $\nZZ1$-promotion $\TIprr1$ 
of $\TI$ has a (finite) chromatic model $\StrA$ with silent types and with
differentiated message types where each king sends an invertible message type.
By message differentiation, no element may send the same invertible message type
twice.
So any invertible message type is realized once per origin.
For any invertible message type $\tpt\in\tput{(\TIprr5)}\arm\arm$ that has at
most $\nWW6$ origins in $\StrA$, we can use the fresh $\nWW6$ binary symbols to
differentiate the realizations at the different origins.
The remaining properties are preserved since we didn't alter the kind of
\twotypes/ between elements and the \onetypes/ of elements.
\end{proof}

\begin{definition}[Special Models]
A model $\StrA$ for $\TI$ is \emph{special} if it is message-differentiated, has
silent types, has rare kings and is $\nW$-origin-differentiated for $\nW =
\card{\TI}+2$.
\end{definition}
