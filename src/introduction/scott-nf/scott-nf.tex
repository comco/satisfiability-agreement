% Scott normal form
In two-variable logics, a common technique of reducing formula quantifier rank
while preserving satisfiability is Skolemization~\cite{gradel1999logics}:
Let $\fphi$ be a $\vFo2$-sentence.
Consider a subformula $\fpsi = \psi(\gy) = Q\gx \falp(\gx,\gy)$ of $\fphi$ that
has the lowest possible nontrivial quantifier rank $1$, where
$\gy\in\set{\xx,\yy}$ may not necessarily occur freely in $\falp$, $\gx \in
\set{\xx,\yy}\sub\set\gy$ and $Q \in \set{\forall,\exists}$.
Introduce a new unary predicate symbol $\suu\psi$ with the intended
interpretation $\forall\gy(\suu\psi(\gy) \lequ Q\gx \falp(\gx,\gy))$ and
let $\fphip$ be the formula obtained from $\fphi$ by replacing the subformula
$\psi$ by $\suu\psi(\gy)$.
The original formula $\fphi$ is equisatisfiable with
$\fphi_1 = \forall\gy(\suu\psi(\gy) \lequ Q\gx \falp(\gx,\gy)) \land
\fphip$ in a strinct sense, that is any model for $\fphi$ can be
$\suu\fpsi$-enriched into a model for $\fphi_1$ and any model for $\fphi_1$ is a
model for $\fphi$.
By repeating this process, we can bring the formula to a form where the
quantifier rank is at most $2$~\cite{scott1962decision,gradel1999logics}:
\begin{theorem}[Scott]
There is a polynomial-time reduction $\sctr: \vFo2 \to \vFo2$ which reduces
every sentence $\fphi$ to a sentence $\sctr\fphi$ in \emph{Scott normal form}:
\[
  \forall\xx\forall\yy(\falpp0(\xx,\yy) \lor \xx = \yy) \land
  \bigwedge_{1 \leq \ii \leq \nm} \forall\xx\exists\yy
  (\falpp\ii(\xx,\yy) \land \xx \neq \yy),
\]
where $\nm\geq1$, the formulas $\falpp\ii$ are quantifier-free and use at most
linearly many new unary predicate symbols.
The sentences $\fphi$ and $\sctr\fphi$ are satisfiable over the same domains of
cardinality at least $2$.
Moreover the length $\sctr\fphi$ is linear in the length of $\fphi$.
\end{theorem}