% Scott normal form
In two-variable logics, a common technique of reducing formula quantifier rank
while preserving satisfiability is Skolemization~\cite{gradel1999logics}:
Let $\fphi$ be a $\vFo2$-sentence.
By replacing universally quantified subformulas $\forall\gx \fpsi$ by twofold
existential negations $\lnot\exists\gx \lnot\fpsi$, without loss of generality
assume that only existential quantifiers occur in $\fphi$.
Consider a subformula $\fpsi$ of $\fphi$ that has the lowest possible nontrivial
quantifier rank $1$.
Then $\psi = \psi(\gy) = \exists\gx \falp(\gx,\gy)$, where the formula
$\falp$ is quantifier-free, $\set{\gx,\gy} = \set{\xx,\yy}$ and $\gy$ may
or may not necessarly occur freely in $\falp$.
Introduce a new unary predicate symbol $\suu\psi$ with the intended
interpretation $\forall\gy \suu\psi(\gy) \lequ \exists\gx \falp(\gx,\gy)$ and
let $\fphip$ be the formula obtained from $\fphi$ by replacing the subformula
$\psi$ by $\suu\psi(\gy)$.
The original formula $\fphi$ is equisatisfiable with
$\fphi_1 = (\forall\gy \suu\psi(\gy) \lequ \exists\gx \falp(\gx,\gy)) \land
\fphip$ in a strinct sense, that is any model for $\fphi$ can be
$\suu\fpsi$-enriched into a model for $\fphi_1$ and any model for $\fphi_1$ is a
model for $\fphi$.
By repeating this process linearly many times, we can bring the formula to a
form where the quantifier rank is at most 
$2$~\cite{scott1962decision,gradel1999logics}:
\begin{theorem}[Scott]
There is a polynomial-time reduction $\sctr: \vFo2 \to \vFo2$ which reduces
every sentence $\fphi$ to a sentence $\sctr\fphi$ in \emph{Scott normal form}:
\[
  (\forall\xx\forall\yy \falpp0(\xx,\yy) \lor \xx = \yy) \land
  \bigwedge_{1 \leq \ii \leq \nm} \forall\xx\exists\yy
  \falpp\ii(\xx,\yy) \land \xx \neq \yy,
\]
where the formulas $\falpp\ii$ are quantifier-free and use at most linearly many
new unary predicate symbols. The sentences $\fphi$ and $\sctr\fphi$ are
satisfiable over the same domains.
Moreover the length $\sctr\fphi$ is linear in the length of $\fphi$.
\end{theorem}

A completely analogous normal form can be described for the two-variable
fragment with counting quantifiers~\cite{MALQ:MALQ201400102}:
\begin{theorem}[Pratt-Hartmann]
There is a polynomial-time reduction $\prtr: \vFoc2 \to \vFoc2$ with reduces
every sentence $\fphi$ to a sentence $\prtr\fphi$ in the form:
\[
  (\forall\xx\forall\yy \falpp0(\xx,\yy) \lor \xx=\yy) \land
  \bigwedge_{1 \leq \ii \leq \nm} \forall\xx\existseq{\nMM\ii}\yy
  \falpp\ii(\xx,\yy) \land \xx\neq\yy,
\]
where the formulas $\falpp\ii$ are quantifier-free and may use linearly many new
unary and binary predicate symbols. Let $\nM = \max\set{\nMM1,\nMM2,\dots,\nMM\nm}$.
Then $\fphi$ and $\prtr\fphi$ are satisfiable over the same domains of
cardinality greater than $\nM$.
Moreover the length $\prtr\fphi$ is linear in the length of $\fphi$.
\end{theorem}
