% Vectors

An $\nn$-vector $\vectv = (\vectvv1, \vectvv2, \dots, \vectvv\nn) \in
\nVects\nn$ is just a tuple of natural numbers.
The $\nn$-vector $\vectv$ is \emph{antilexicographically smaller}
than the $\nn$-vector $\vectw$ (written $\gls{vector-v-smaller-w}$) if there is
a position $\posp \in [1, \nn]$ such that $\vectvv\posp < \vectww\posp$ and
$\vectvv\posq = \vectww\posq$ for all $\posq \in [\posp+1, \nn]$.
For instance $(1,1,0)\lexlt(0,0,1)$.
