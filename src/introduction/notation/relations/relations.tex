% Relations

Let $\setA$ and $\setB$ be sets and let $\relR\subseteq\setA\cprod\setB$ be a
binary relation.
The \emph{domain} of $\relR$ is $\gls{domain} = A$ and its \emph{range} is
$\gls{range} = B$.
The \emph{inverse} $\inv\relR\subseteq\setB\cprod\setA$ of $\relR$ is
\[
  \gls{inverse-R} = 
  \setbd{(\eb, \ea)}{(\ea, \eb) \in \relR}.
\]
If $\setAp\subseteq\setA$, the \emph{restriction}
$(\relR\restriction\setAp) \subseteq \setAp\cprod\setB$ of $\relR$ to $\setAp$
is
\[
  \gls{restriction-R-S} =
  \setbd{(\ea, \eb) \in \relR}{\ea \in \setAp}.
\]
If $\ea \in \setA$, the \emph{$\relR$-successors} 
of $\ea$ are
\[
  \gls{R-successors-a} =
  \setbd{\eb \in \setB}{(\ea, \eb) \in \relR}.
\]
If $\relS \subseteq \setB \cprod \setC$ and $\relR \subseteq \setA \cprod \setB$
are two binary relations, their \emph{composition} is 
\[
  \gls{composition-S-R} =
  \setbd{(\ea, \ec) \in \setA \cprod \setC}{
  (\exists \eb \in \setB)\left((\ea, \eb)
  \in \relR \land (\eb, \ec) \in \relS\right)}.
\]