% Relations

If $\relR$ is a binary relation, its \emph{domain} is \gls{domain} and its
\emph{range} is \gls{range}.
The \emph{inverse} of $\relR \subseteq \setA \cprod \setB$ is 
\[
  \gls{inverse-R} = 
  \setbd{(\eb, \ea) \in \setB \cprod \setA}{(\ea, \eb) \in \relR}.
\]
If $\setS$ is a set and $\relR \subseteq \setA \cprod \setB$, the
\emph{restriction} of $\relR$ to $\setS$ is
\[
  \gls{restriction-R-S} =
  \setbd{(\ea, \eb) \in \relR}{\ea \in \setS}.
\]
If $\relR \subseteq \setA \cprod
\setB$ is a binary relation and $\ea \in \setA$, the \emph{$\relR$-successors} 
of $\ea$ are
\[
  \gls{R-successors-a} =
  \setbd{\eb \in \setB}{(\ea, \eb) \in \relR}.
\]
If $\relS \subseteq \setB \cprod \setC$ and $\relR \subseteq \setA \cprod \setB$
are two binary relations, their \emph{composition} is 
\[
  \gls{composition-S-R} =
  \setbd{(\ea, \ec) \in \setA \cprod \setC}{(\exists \eb \in \setB) (\ea, \eb)
  \in \relR \land (\eb, \ec) \in \relS}.
\]