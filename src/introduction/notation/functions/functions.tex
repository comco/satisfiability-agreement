% Functions

A \emph{function}
$\gls{total-function-f-A-B}$ is formally just a functional relation
$\funf\subseteq\setA\cprod\setB$.
An \emph{injective} function from $\setA$ into $\setB$ is denoted
$\gls{injective-function-f-A-B}$.
A \emph{surjective} function from $\setA$ onto $\setB$ is denoted
$\gls{surjective-function-f-A-B}$.
A \emph{bijective} function between $\setA$ and $\setB$ is denoted
$\gls{bijective-function-f-A-B}$.
The \emph{identity function} on $\setA$ is $\gls{identity-A}$.
A \emph{partial} function from $\setA$ to $\setB$ is denoted
$\gls{partial-function-f-A-B}$. If $\ff : \setA \pto \setB$ is a partial
function and $\ea \in \setA$, the notation $\gls{f-defined-a-b}$ means that
$\ff$ is defined at $\ea$ and its value is $\eb$;
the notation $\gls{f-undefined-a}$ means that $\ff$ is not defined at $\ea$,
where $\bot$ is specially chosen to never be an element of $\setB$.
If $\setS \subseteq \setA$, the \emph{characteristic function}
$\gls{characteristic-function-S-A} : \setA \to \set{0,1}$ of $\setS$ in $\setA$
is defined by:
\[
\chfun\setA\setS\ea = \begin{cases}
1 &\text{if } a \in S \\
0 &\text{otherwise.}
\end{cases}
\]