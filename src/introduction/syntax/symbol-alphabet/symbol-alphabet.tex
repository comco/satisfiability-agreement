% Symbol alphabet

The \emph{symbol alphabet} for the first-order logic with counting quantifiers
is
\[
  \gls{symbol-alphabet} = \set{\lnot, \allconns; \allQs;
  =; (,,,); {}^{\leq}, {}^{=}, {}^{\geq}, {}^0, {}^1}.
\]
The propositional connectives are listed in decreasing order of precedence.
The \emph{negation} $\lnot$ is unary;
the \emph{disjunction} $\lor$, \emph{conjunction} $\land$ and \emph{equivalence}
$\lequ$ are left-associative; the \emph{implication} $\limp$ is
right-associative.
Note that we consider logics with \emph{formal equality} $=$.

A \emph{counting quantifier} is a word over $\SymbAlph$ of the form
$\existsleq{\benc{\nm}}$ or $\existseq{\benc{\nm}}$ or $\existsgeq{\benc{\nm}}$,
where $\nm \in \Nats$ and $\benc{\nm} \in \Bitstrings$ is the binary encoding of
$\nm$. Note that this encoding of the counting quantifiers is \emph{succinct}.
As we note in~\Cref{rem:sttr-exp-size}, this succinct representation allows for
exponentially small counting formulas compared to their pure first-order
equivalents.
We denote the counting quantifiers by $\existsleq\nm$, $\existseq\nm$ and
$\existsgeq\nm$, that is, we omit the encoding notation for $\nm$.