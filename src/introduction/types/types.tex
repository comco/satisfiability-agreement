% Types
Let $\SigS = \seq{\pp1, \pp2, \dots, \pp\sizes}$ be a predicate signature.
A \emph{$1$-type} $\tpIp$ over $\SigS$ is a maximal consistent set of literals
featuring only the variable symbol $\xx$.
The set of $1$-types over $\SigS$ is $\gls{1-types-S}$.
Note that consistency here is relativised by the intended interpretations of the
predicate signature. For example if $\SigS$ contains the binary predicate symbol
$\se$ with intended interpretation as an equivalence, then every $1$-type over
$\SigS$ includes the literal $\se(\xx,\xx)$.
Also note that the cardinality of a $1$-type over $\SigS$ is exponentially
bounded by the length $\sizes$ of $\SigS$ and the cardinality of $\TpI\SigS$ is
doubly exponentially bounded by $\sizes$.

A \emph{$2$-type} $\tpTt$ over $\SigS$ is a maximal consistent set of literals
featuring only the variable symbols $\xx$ and $\yy$ and including the literal
$(\xx \neq \yy)$.
The set of $2$-types over $\SigS$ is $\gls{2-types-S}$.
Again, consistency is relativised by the intended interpretation of the
predicate signature.
For example, if $\SigS$ contains the binary predicate
symbol $\se$ with intended interpretation as an equivalence,
then if $\se(\xx, \yy) \in \tpTt$, then $\se(\yy, \xx) \in \tpTt$.
Again, the cardinality of a $2$-type over $\SigS$ is exponentially bounded by
$\sizes$ and the cardinality of $\TpT\SigS$ is doubly exponentially bounded by
$\sizes$.

If $\tpTt \in \TpT\SigS$, the \emph{inverse} $\gls{t-type-inverse}$ of $\tpTt$
is the $2$-type obtained from $\tpTt$ by
swapping the variables $\xx$ and $\yy$ in every literal.
The \emph{$\xx$-type} of $\tau$ is the $1$-type $\gls{x-type-t}$
consisting of all the literals of $\tpTt$ featuring only the variable symbol
$\xx$. Similarly, the \emph{$\yy$-type} of $\tau$ is the $1$-type
$\gls{y-type-t}$ consisting of all the literals of $\tpTt$ featuring only the
variable symbol $\yy$, that is replaced by $\xx$.
For example we have the identity $\xtp\inv\tpTt = \ytp\tpTt$.

If $\StrA$ is a $\SigS$-structure and $\ea \in \domA$, the \emph{$1$-type of
$\ea$ in $\StrA$} is
\[
  \gls{1-type-a-A} = \setbd{\litl(\xx) \in \LitF\SigS}{\StrA \vDash \litl(\ea)}.
\]
If $\tpIa\StrA\ea = \tpIp$, we say that the $1$-type $\tpIp$ is \emph{realized}
by $\ea$ in $\StrA$. The interpretation of the $1$-type $\tpIp$ in $\StrA$ is
the set of elements realizing $\tpIp$:
\[
  \gls{1-type-interpretation-A} = \setbd{\ea \in \domA}{\tpIa\StrA\ea = \tpIp}.
\]

If $\ea \neq \eb \in \domA$, the \emph{$2$-type of $(\ea, \eb)$ in $\StrA$} is
\[
  \gls{2-type-a-b-A} = \setbd{\litl(\xx,\yy) \in \LitF\SigS}{\StrA \vDash
  \litl(\ea, \eb)}.
\]
We do not define a $2$-type in case $\ea = \eb$.
If $\tpIab\StrA\ea\eb = \tpTt$, we say that the $2$-type $\tpTt$ is
\emph{realized} by $(\ea, \eb)$ in $\StrA$. The interpretation of the $2$-type
$\tpTt$ in $\StrA$ is the set of pairs realizing $\tpTt$:

\[
  \gls{2-type-interpretation-A} = \setbd{(\ea, \eb) \in
  \domA\cprod\domA}{\ea \neq \eb \land \tpIab\StrA\ea\eb = \tpTt}.
\]