% Logic games
Logic games capture structure equivalence.
Let $\SigS$ be a predicate signature and let $\StrA$ and $\StrB$ be
$\SigS$-structures.
A \emph{partial isomorphism} $\gls{partial-isomorphism-p} : \domA \pto \domB$
from $\StrA$ to $\StrB$ is a partial mapping that is an isomorphism between the 
induced substructures $(\StrA \restriction \dom \pisop)$ and 
$(\StrB \restriction \ran \pisop)$.

Let $\nr \in \pNats$.
The \emph{$\nr$-round Ehrenfeucht-Fra\"{i}ss\'{e} game}
$\gls{r-round-ef-game-A-B}$ is a two-player game, played with a pair of pebbles,
one for each structure. The two players are Spoiler and Duplicator.
Initially the pebbles are off the structures.
During each round, Spoiler picks a pebble
and places it on some element in its designated structure.
Duplicator responds by picking the other pebble and placing it on some
element in the other structure. Thus during round $\ii$, the players play a pair
of elements $\eaa\ii \mapsto \ebb\ii \in \domA \cprod \domB$. 
Collect the sequences of played elements $\many\ea = \seq{\eaa1, \eaa2, \dots,
\eaa\nr}$ and $\many\eb = \seq{\ebb1, \ebb2, \dots, \ebb\nr}$.
Duplicator wins the match if the
relation $\many\ea \mapsto \many\eb = \set{\eaa1 \mapsto \ebb1, \eaa2 \mapsto
\ebb2, \dots, \eaa\nr \mapsto \ebb\nr} \subseteq \domA \times \domB$, built from
the pairs of elements in each round, is a partial isomorphism from $\StrA$ to
$\StrB$.
Ehrenfeucht's theorem says that Duplicator has a winning strategy for
$\refG\nr\StrA\StrB$ iff $\StrA \requiv\nr \StrB$.
Fra\"{i}ss\'{e}'s theorem gives a back-and-forth characterization of the winning
strategy for Duplicator~\cite[ch. 2]{ebbinghaus1999finite}:
\begin{theorem}\label{thm:game-ef}
Suppose that $(\pisoII0, \pisoII1, \dots, \pisoII\nr)$ is a sequence of nonempty
sets of partial isomorphisms between $\StrA$ and $\StrB$ with the following
properties:
\begin{enumerate}
  \item For every $\jj < \nr$, $\pisop \in \pisoII{\jj+1}$ and $\ea \in \domA$,
  there is $\pisoq \in \pisoII\jj$ such that $\pisop \subseteq \pisoq$ and
  $\ea \in \dom\pisoq$.
  \item For every $\jj < \nr$, $\pisop \in \pisoII{\jj+1}$ and $\eb \in \domB$,
  there is $\pisoq \in \pisoII\jj$ such that $\pisop \subseteq \pisoq$ and 
  $\eb \in \ran\pisoq$.
\end{enumerate}
Then $\StrA \requiv\nr \StrB$.
\end{theorem}

Let $\nv \in \pNats$. By convention $\bot$ is an element not occurring in any
structure. If $\many\ea = \eaa1\eaa2\dots\eaa\nv \in (\domA \cup
\set{\bot})^\nv$ is a vector, the \emph{support} of $\many\ea$ is
$\gls{support-a} = \setbd{\ii \in [1,\nv]}{\eaa\ii \in \domA}$ and an
$\ii$-substitute of $\many\ea$ is $\gls{i-substitute-a} =
\eaa1\eaa2\dots\eaa{\ii-1}\ea\eaa{\ii+1}\dots\eaa\nv$.
If $\StrA$ and $\StrB$ are $\SigS$-structures, a \emph{$\nv$-partial
isomorphism} from $\StrA$ to $\StrB$ is a pair $\pisop = \many\ea \mapsto
\many\eb$, where $\many\ea \in (\domA \cup \set{\bot})^\nv$,
$\many\eb \in (\domB \cup \set{\bot})^\nv$, $\supp \many\ea = \supp \many\eb$
and $\many\eap \mapsto \many\ebp$ is a partial isomorphism, where $\many\eap$
and $\many\ebp$ are the subsequences of $\many\ea$ and $\many\eb$ with indexes
from the support. We also think of $\pisop$ as a partial function
$\pisop : \domA \pto \domB$, defined by
\[
  \pisop = \setbd{\many\ea_i \mapsto \many\eb_i}{i \in \supp \many\ea}.
\]
We now consider \emph{pebble games}, which characterise constrained variables
equivalence of structures.
Let $\nr \in \pNats$.
The \emph{$\nr$-round $\nv$-pebble game} $\gls{r-round-v-pebble-game-A-B}$ is a
two-player game, with $\nv$ pairs of pebbles.
Initially the pebbles are off the board.
During each round, Spoiler picks a pair of pebbles and puts one of them on an
element in one of the structures. Duplicator responds by putting the other
pebble from the pair on an element from the other structure.
Thus the state of the game after round $\ii$ is thus captured by two sequences
$\many\ea_\ii \in (\domA \cup \set{\bot})^\nv$ and 
$\many\eb_\ii \in (\domB \cup \set{\bot})^\nv$ having the same support, showing
which pairs of pebbles have been placed where.
Duplicator wins the match if for each $\ii \leq \nr$ we have that 
$\many\ea_\ii \mapsto \many\eb_\ii$ is a $\nv$-partial isomorphism.
Similarly to the Ehrenfeucht-Fra\"{i}ss\'{e} game, we have that Duplicator has a
winning strategy for $\rvpbG\nr\nv\StrA\StrB$ iff $\StrA \rvequiv\nr\nv \StrB$.
The following is a back-and-forth characterization of the winning strategy for
Duplicator~\cite[ch. 2]{ebbinghaus1999finite}:
\begin{theorem}\label{thm:game-2}
Suppose that $(\pisoII0, \pisoII1, \dots, \pisoII\nr)$ is a sequence of nonempty
sets of $\nv$-partial isomorphisms between $\StrA$ and $\StrB$ with the
following properties:
\begin{enumerate}
  \item For every $\jj \in [0,\nr-1]$, $\ii \in [1,\nv]$,
  $\many\ea \mapsto \many\eb \in \pisoII{\jj+1}$ and $\ea \in \domA$,
  there is $\eb \in \domB$ such that
  $\vectsub{\many\ea}\ii\ea \mapsto \vectsub{\many\eb}\ii\eb \in \pisoII\jj$.
  \item For every $\jj \in [0,\nr-1]$, $\ii \in [1,\nv]$,
  $\many\ea \mapsto \many\eb \in \pisoII{\jj+1}$ and $\eb \in \domB$,
  there is $\ea \in \domA$ such that
  $\vectsub{\many\ea}\ii\ea \mapsto \vectsub{\many\eb}\ii\eb \in \pisoII\jj$.
\end{enumerate}
Then $\StrA \rvequiv\nr\nv \StrB$.
\end{theorem}