% Builtin interpretations

A \emph{predicate signature with intended interpretations} $\SigS$ is formally a
predicate signature together with an \emph{intended interpretation condition}
$\mathcal{A}$, which is formally a class of $\SigS$-structures.
A $\SigS$-structure $\StrA$ is then just an element of $\mathcal{A}$. That is,
when we speak about a predicate signature with intended interpretations,
we are considering the logics strictly over the class of structures respecting
the intended interpretation condition. The semantic concepts are relativised
appropriately in this context. For example, if $\SigS = \seq{\se}$ is a
predicate signature consisting of the single binary predicate symbol $\se$,
having intended interpretation as an equivalence, then the $\SigS$-formula
$\forall\xx \se(\xx,\xx)$ is logically valid.
From now on, we will use the term \emph{predicate signature} as \emph{predicate
signature with possible intended interpretations}.

The predicate signature $\SigSp$ is an \emph{enrichment} of the predicate
signature $\SigS$ if $\SigSp$ contains all predicate symbols of $\SigS$ and
respects their intended interpretation in $\SigS$. A $\SigSp$-structure $\StrAp$
is an enrichment of the $\SigS$-structure $\StrA$ if they have the same domain
and the same interpretation of the predicate symbols of $\SigS$.
The basic semantic significance of enrichment is that if
$\fphi(\gxx1, \gxx2, \dots, \gxx\nn)$ is a $\SigS$-formula and $\eaa1, \eaa2,
\dots, \eaa\nn \in A$, then $\StrA \vDash \fphi(\eaa1, \eaa2, \dots, \eaa\nn)$
iff $\StrAp \vDash \fphi(\eaa1, \eaa2, \dots, \eaa\nn)$.

If $\fphi(\gxx1, \gxx2, \dots, \gxx\nn)$ is a focused formula, the
interpretation of $\fphi$ in $\StrA$ is
\[
  \gls{interpretation-phi-A} = \setbd{(\eaa1, \eaa2, \dots, \eaa\nn) \in
  \domA^\nn}{\StrA \vDash \fphi(\eaa1, \eaa2, \dots, \eaa\nn)}.
\]