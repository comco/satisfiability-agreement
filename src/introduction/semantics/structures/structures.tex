% Structures

If $\SigS$ is a predicate signature, a \emph{$\SigS$-structure}
$\gls{structure-A}$ consists of a nonempty set $\domA$ (the \emph{domain} of
$\StrA$), together with a relation $\at\StrA\gp \subseteq A^{\ar\gp}$ (the
\emph{interpretation} of $\gp$ at $\StrA$) for every predicate symbol $\gp
\in \SigS$. A structure is \emph{finite} if its domain is finite.
We omit the standard definition of semantic notions.
Seldom it will be useful to consider \emph{structures with possibly empty
domain}. We will be explicit when this is the case.
If $\StrA$ is a structure and $\domB \subseteq \domA$ there is a substructure
$\StrB \subseteq \StrA$ with possibly empty domain $\domB$. We call it the
substructure induced by $\domB$ and denote it $(\StrA \restriction \domB)$.

Note that the interpretation of the counting quantifiers is clear:
$\existsleq\nm\gx\fphi$ means that ``at most $\nm$ elements satisfy $\fphi$'';
$\existseq\nm\gx\fphi$ means that ``exactly $\nm$ elements satisfy $\fphi$'';
$\existsgeq\nm\gx\fphi$ means that ``at least $\nm$ elements satisfy $\fphi$''.