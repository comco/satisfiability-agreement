% Counters
A \emph{$\tbit$-bit counter setup} for $\tbit \in \pNats$ is a predicate
signature $\gls{counter-setup-C} = \seq{\suu1, \suu2, \dots, \suu\tbit}$
consisting of $\tbit$ distinct unary predicate symbols $\suu\ii$.

\begin{definition}
Let $\StrA$ be a $\CS$-structure. Define the function
$\gls{data-C-A} : \domA \to \tBitnums\tbit$ by:
\[
  \data\CS\StrA\ea = \sum_{1 \leq \ii \leq \tbit} 2^{\ii-1} 
  \data{\suu\ii}\StrA\ea.
\]
\end{definition}
\begin{definition}
Let $\datad \in \tBitnums\tbit$ be a $\tbit$-bit number. Define the
quantifier-free $\vFoF1\CS$-formula $\gls{feqA-C-d-x}$ by:
\[
  \feqA\CS\datad(\xx) = \bigwedge_{1 \leq \ii \leq \tbit}
  \feqA{\suu\ii}{\benc{\datad}_\ii}(\xx).
\]
\end{definition}
\begin{remark}
If $\StrA$ is a $\CS$-structure, $\ea \in \domA$ and $\datad \in
\tBitnums\tbit$, then $\StrA \vDash \feqA\CS\datad(a)$ iff $\data\CS\StrA a =
\datad$.

If $\domA$ is a nonempty set and $\dataOP : \domA \to \tBitnums\tbit$ is any
function, there is a $\CS$-structure $\StrA$ over $\domA$ such that
$\data\CS\StrA = \dataOP$.
\end{remark}

\begin{definition}
Define the quantifier-free $\vFoF2\CS$-formula $\gls{feq-C-x-y}$ by:
\[
  \feq\CS(\xx,\yy) = \bigwedge_{1 \leq \ii \leq \tbit} \feq{\suu\ii}(\xx,\yy).
\]
\end{definition}
\begin{remark}
If $\StrA$ is a $\CS$-structure and $\ea, \eb \in \domA$,
then $\StrA \vDash \feq\CS(\ea, \eb)$ iff $\data\CS\StrA\ea = \data\CS\StrA\eb$.
\end{remark}
The bitstring $\bstra \in \tBitstrings\tbit$ encodes a number less than the
number encoded by the bitstring $\bstrb \in \tBitstrings\tbit$, if they differ
and at least position where they are different $\jj \in [1,\tbit]$ the bitstring
$\bstra$ has value $0$ and the bitstring $\bstrb$ has value $1$, that is, iff
there is a position $\jj \in [1,\tbit]$ such that the following two conditions
hold:
\begin{align}
  \bstra_\jj = 0 \text{ and } \bstrb_\jj = 1 \tag{Less1}\label{Less1} \\
  \bstra_\kk = \bstrb_\kk \text{ for all } \kk \in [\jj+1,\tbit].
  \tag{Less2}\label{Less2}
\end{align}
\begin{definition}
Define the quantifier-free $\vFoF2\CS$-formula $\gls{fless-C-x-y}$ by:
\[
  \fless\CS(\xx,\yy) = \bigvee_{1 \leq \jj \leq \tbit} \feqOI{\suu\jj}(\xx,\yy)
  \land \bigwedge_{\jj < \kk \leq \tbit} \feq{\suu\kk}(\xx,\yy).
\]
\end{definition}
\begin{remark}
If $\StrA$ is a $\CS$-structure and $\ea, \eb \in \domA$,
then $\StrA \vDash \fless\CS(\ea, \eb)$ iff
$\data\CS\StrA\ea < \data\CS\StrA\eb$.
\end{remark}

The bitstring $\bstrb \in \tBitstrings\tbit$ encodes the successor of the number
encoded by the bitstring $\bstra$ if there is a position $\jj \in [1,\tbit]$
such that the following four conditions hold:
\begin{align}
  \bstra_\jj = 0 \text{ and } \bstrb_\jj = 1 \tag{Succ1}\label{Succ1} \\
  \bstra_\ii = 1 \text{ for all } i \in [1,\jj-1] \tag{Succ2}\label{Succ2} \\
  \bstrb_\ii = 0 \text{ for all } i \in [1,\jj-1] \tag{Succ3}\label{Succ3} \\
  \bstra_\kk = \bstrb_\kk \text{ for all } \kk \in [\jj+1,\tbit].
  \tag{Succ4}\label{Succ4}
\end{align}
\begin{definition}
Define the quantifier-free $\vFoF2\CS$-formula $\gls{fsucc-C-x-y}$ by:
\[
  \fsucc\CS(\xx,\yy) = \bigvee_{1 \leq \jj \leq \tbit}
  \feqOI{\suu\jj}(\xx,\yy) \land 
  \bigwedge_{1 \leq \ii < \jj} 
  \feqIO{\suu\ii}(\xx,\yy) \land
  \bigwedge_{\jj < \kk \leq \tbit} \feq{\suu\kk}(\xx,\yy).
\]
\end{definition}
\begin{remark}
If $\StrA$ is a $\CS$-structure and $\ea, \eb \in \domA$,
then:
\[
  \miff{
  \StrA \vDash \fsucc\CS(\ea, \eb)}{
  \data\CS\StrA\eb = 1 + \data\CS\StrA\ea}.
\]
\end{remark}
\begin{definition}
Let $\datad \in \tBitnums\tbit$.
Define the quantifier-free $\vFoF1\CS$-formula $\gls{fless-C-d-x}$ by:
\[
  \flessA\CS\datad(\xx) = \bigvee_{1 \leq \jj \leq \tbit} \lnot\suu\jj(\xx)
  \land \lnot\feqA{\suu\jj}{\benc\datad_\jj}(\xx)
    \land \bigwedge_{\jj < \kk \leq \tbit} \feqA{\suu\kk}{\benc\datad_\kk}(\xx).
\]
\end{definition}
\begin{remark}
If $\StrA$ is a $\CS$-structure, $\ea \in \domA$ and $\datad \in
\tBitnums\tbit$, then $\StrA \vDash \flessA\CS\datad(\ea)$ iff $\data\CS\StrA\ea
< \datad$.
\end{remark}
\begin{definition}
Let $\datad \leq \datae \in \tBitnums\tbit$.
Define the quantifier-free $\vFoF1\CS$-formula $\gls{fbetw-C-d-e-x}$ by:
\[
  \fbetw\CS\datad\datae(\xx) = \lnot\flessA\CS\datad(\xx) \land 
  (\flessA\CS\datae(\xx)\lor \feqA\CS\datae(\xx)).
\]
\end{definition}
\begin{remark}
If $\StrA$ is a $\CS$-structure, $\ea \in \domA$ and $\datad \leq \datae \in
\tBitnums\tbit$, then 
\[
  \miff{
  \StrA \vDash \fbetw\CS\datad\datae(\ea)}{
  \datad \leq \data\CS\StrA\ea \leq
  \datae}.
\]
\end{remark}

\begin{definition}
Let $\datad \leq \datae \in \tBitnums\tbit$.
Define the $\vFoF1\CS$-sentence $\gls{fallbetw-C-d-e}$ by:
\[
  \fallbetw\CS\datad\datae = \forall\xx\fbetw\CS\datad\datae(\xx).
\]
\end{definition}
\begin{remark}
If $\StrA$ is a $\CS$-structure and $\datad \leq \datae \in \tBitnums\tbit$,
then $\StrA \vDash \fbetw\CS\datad\datae$ iff
$\datad \leq \data\CS\StrA\ea \leq \datae$ for all $\ea \in \domA$.
\end{remark}