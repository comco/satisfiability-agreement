% Abstract

A sequence of equivalence relations $\relEE1,\relEE2,\dots,\relEE\nn$ on $\domA$
is in \emph{refinement} if $\relEE\ii\subseteq\relEE{\ii+1}$ for
$\ii\in[1,\nn-1]$, that is if
$\relEE1\subseteq\relEE2\subseteq\dots\subseteq\relEE\nn$.
The sequence is in \emph{global agreement} if there is some permutation
$\permnu$ of $[1,\nn]$ such that the sequence
$\relEE{\permnu(1)}, \relEE{\permnu(2)}, \dots, \relEE{\permnu(\nn)}$ is in
refinement.
The sequence is in \emph{local agreement} if for every $\ea\in\domA$ there is
some permutation $\permnu = \permnu(\ea)$ of $[1,\nn]$ such that
$\relEE{\permnu(1)}[\ea] \subseteq \relEE{\permnu(2)}[\ea] \subseteq \dots
\subseteq \relEE{\permnu(\nn)}[\ea]$.

The topic of this work is to investigate questions about the algorithmic
complexity of the satisfiability and finite satisfiability of logics featuring
equivalence symbols at different levels of agreement.
A summary of this work is as follows:
\begin{itemize}
  \item
  In~\Cref{ch:intro} we introduce the notations and the tools that we will
  need further.
  
  \item
  In~\Cref{ch:setups} we define various \emph{setups} --- suites of appropriate
  formulas --- that allow us to model bounded discrete objects such as $\tbit$
  numbers or permutations of $[1,\nn]$ into logical structures.

  \item
  In~\Cref{ch:equivalences} we define the three agreement properties: local,
  global agreement and refinement and develop the theory of equivalence
  relations in local agreement sufficiently for our purposes.
  In particular, we prove~\Cref{thm:local} that a sequence $\relE =
  \relEE1,\relEE2,\dots,\relEE\nn$ of equivalence relations on $\domA$ is in
  local agreement iff the union $\cup\setS$ of any nonempty subsequence $\setS$
  of $\relE$ is an equivalence relation on $\domA$.
  This allows us to define the \emph{level sequence} (\Cref{def:lvl-seq}) of a
  sequence of equivalence relations in local agreement and to characterize as a
  some kind of a ``skeleton'', which combined with a local permutation at every
  element $\ea\in\domA$ completely characterizes the sequence $\relE$
  (\Cref{lem:local-lvl-refine} and~\Cref{lem:local-lvl-perm}).
  
  \item
  In~\Cref{ch:reductions} we provide deterministic polynomial-time reductions
  for the (finite) satisfiability problem featuring equivalence symbols in
  global and local agreement into the corresponding problem for equivalence
  symbols in refinement (\Cref{prop:global-to-refine-n},
  \Cref{prop:global-to-refine}, \Cref{prop:local-to-refine-n},
  \Cref{prop:local-to-refine}).
  This allows us to concentrate on the case of refinement further.
  
  \item
  In~\Cref{ch:monadic} we determine the computational complexity of the
  (finite) satisfiability problem for the first-order logic featruing only unary
  predicate symbols together with $\sze$ equivalence symbols in agreement:
  $\Lvp\Fo\nonv1\Eea\sze\agrefine$,
  $\Lvp\Fo\nonv1\Eea\sze\agglobal$ and 
  $\Lvp\Fo\nonv1\Eea\sze\aglocal$.
  We prove that these logics have the finite model property and that the
  (finite) satisfiability problem for any of them is
  $\ceNExpTime{(\sze+1)}$-complete (\Cref{prop:mon-in}
  and~\Cref{prop:mon-hard}).
  
  \item
  In~\Cref{ch:twovar} we determine the computational complexity of the (finite)
  satisfiability problem for the two-variable first-order logic featuring unary
  and binary predicate symbols together with $\sze$ equivalence symbols in
  refinement,
  $\Lvp\Fo2\nopow\Eea\sze\agrefine$.
  We prove that this logic has the finite model property and that its (finite)
  satisfiability problem is in $\cNExpTime$ (\Cref{cor:twovar-log}).
\end{itemize}

As for future work in this area, we believe that the methods introduced
in~\Cref{ch:twovar} can be adapted to the two-variable first-order logic with
counting quantifiers.
Another direction for research is to check if the decidability of the
satisfiability in corresponding modal logics is computationally simpler than the
general two-variable case.
Alternatively, it may be interesting to consider more relaxed notions than
agreement, where two different equivalence classes may have common elements but
only to some limited extent.
