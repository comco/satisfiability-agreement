% Monadic first-order satisfiability

In this section we will employ the results on cells and organs to bound the size
of a small substructure of a general structure.

\begin{remark}\label{rem:monadic-granular}
Let $\szu, \sze \geq 2 \in \Nats$ and consider the predicate
signature $\SigS = \ueSigS\szu\sze =
\seq{\suu1,\suu2,\dots,\suu\szu,\see1,\see2,\dots,\see\sze}$.
Abbreviate $\sd = \see1$ and $\se = \see2$.
Let $\nr \in \pNats$.
There is $\StrB \subseteq \StrA$ such that
$\StrB \requiv\nr \StrA$ and $\seq{\at\StrB\sd, \at\StrB\se}$ is $\gr$-granular
for $\gr = \nr.((\nr+1)^{2^\szu}-1)$.
Furthermore, this $\StrB$ has the property that every $\StrB$-cell has
cardinality at most $\nr$.
\end{remark}
\begin{proof}
By \cref{lem:monadic-cell}, there is $\StrBp \subseteq \StrA$ such that
$\StrBp \requiv\nr \StrA$ and every $\StrBp$-cell has cardinality at most $\nr$.
By \cref{lem:monadic-organ}, there is $\StrB \subseteq \StrBp$ such that
$\StrB \requiv\nr \StrBp$ and the $\StrB$-organs have cardinality at most $\nr$.
Let $\relD = \at\StrB\sd$ and $\relE = \at\StrB\se$.
Since every $\relD$-class includes at most $2^\szu$ cells and is
nonempty and every cell has cardinality at most $\nr$,
there are at most $\left((\nr+1)^{2^\szu}-1\right)$ nonisomorphic
$\relD$-classes in $\StrB$.
Since every $\relE$-class includes at most $\nr$ isomorphic $\relD$-classes,
we get that $\seq{\relD, \relE}$ is $\gr$-granular.
\end{proof}

\begin{remark}\label{rem:monadic-1}
Let $\szu \in \Nats$ and consider $\SigS = \ueSigS\szu1 =
\seq{\suu1,\suu2,\dots,\suu\szu,\sd}$. Let $\nr \in \pNats$.
There is $\StrB \subseteq \StrA$ such that $\StrA \requiv\nr \StrB$ and
$\card\domB \leq \gr.r.2^\szu$ for $\gr = \nr.((\nr+1)^{2^\szu}-1)$.
\end{remark}
\begin{proof}
Let $\SigSp = \SigS + \seq{\se}$ be an enrichment of $\SigS$ with the builtin
equivalence symbols $\se$. Consider an enrichment $\StrAp$ of $\StrA$ to a
$\SigSp$-structure, where $\at\StrAp\se = \domAA$ is interpreted as the
full relation on $\domA$. Then $\seq{\at\StrAp\sd, \at\StrAp\se}$ is a sequence
of equivalence relations on $\domA$ in refinement.
By \cref{rem:monadic-granular}, there is $\StrBp \subseteq \StrAp$ such that
$\StrBp \requiv\nr \StrAp$ and $\seq{\at\StrBp\sd,\at\StrBp\se}$ is
$\gr$-granular.
Consider the impoverishment $\StrB$ of $\StrBp$ to a
$\SigS$-structure. Let $\relD = \at\StrB\sd$ and $\relE = \at\StrB\se$.
Since every $\StrB$-cell has
cardinality at most $\nr$ and every $\relD$-class includes at most $2^\szu$
cells, we have that every $\relD$-class has cardinality at most $r.2^\szu$.
Since $\se$ was interpreted in $\StrA$ as the full relation, it is also
interpreted in $\StrB$ as the full relation, so there is a single
$\relE$-class---the whole domain $\domB$.
Since the sequence $\seq{\relD, \relE}$ is $\gr$-granular, there are at most
$\gr$ $\relD$-classes, so $\card\domB \leq \gr.r.2^\szu$.
\end{proof}

\begin{corollary}
The logic $\Lvp\Fo\nonv1\Eea1\noag$ has the finite model property and its
(finite) satisfiability problem is in $\ceNExpTime2$.

% TODO: Add a reference to the tightness
In the next section we will show that this bound is tight.
\end{corollary}

By \cref{lem:granular-f-to-tr} and \cref{lem:granular-tr-to-f} about
granularity combined with \cref{rem:monadic-granular}, we get by induction on
$\sze$:
\begin{proposition}
For $\sze \in \pNats$, the logic $\Lvp\Fo\nonv1\Eea\sze\agrefine$ has the finite
model property and its (finite) satisfiability problem is in
$\ceNExpTime{(\sze+1)}$.

By \cref{prop:global-to-refine-n} and \cref{prop:local-to-refine-n}, the same
holds for the logics $\Lvp\Fo\nonv1\Eea\sze\agglobal$ and
$\Lvp\Fo\nonv1\Eea\sze\aglocal$.
\end{proposition}

Iterating this, we get:
\begin{proposition}
The logic $\Lvp\Fo\nonv1\Eea\nosze\agrefine$ has the finite
model property and its (finite) satisfiability problem is in
$\cElementary$.

By \cref{prop:global-to-refine} and \cref{prop:local-to-refine}, the same
holds for the logics $\Lvp\Fo\nonv1\Eea\nosze\agglobal$ and
$\Lvp\Fo\nonv1\Eea\nosze\aglocal$.
\end{proposition}
