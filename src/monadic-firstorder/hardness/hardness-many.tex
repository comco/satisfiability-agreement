% Hardness with many equivalences in refinement
The argument from the previous section can be iterated to yield the hardness of
the (finite) satisfiability of the monadic first-order logic with several
builtin equivalence symbols in refinement $\Lvp\Fo\nonv1\Eea\sze\agrefine$.

For $\sze \in \pNats$, consider the predicate signature $\ES(\sze) =
\seq{\see1,\see2,\dots,\see\sze}$ consisting of the builtin equivalence
symbols $\see\ii$ in refinement. Abbreviate the \emph{coarsest} equivalence 
symbol $\sd = \see\sze$.
\begin{definition}
Let $\sze \in \pNats$. An \emph{$(\sze+1)$-exponential setup} is a uniform
effective polynomial-time process for creating the following data structure.
For every $\nn \in \pNats$, there is a predicate signature $\DS(\sze,\nn)$
having length polynomial in $\nn$, consisting of unary predicate symbols and
containing $\ES(\sze)$. The following data is effectively defined:
\begin{enumerate}
  \item If $\StrA$ is a $\DS(\sze,\nn)$-structure and $\relD = \at\StrA\sd$,
  then there is a function 
  $\Data{\DS(\sze,\nn)}\StrA : \Ecl\relD \to [0, \tetr{\sze+1}2\nn-1]$ that
  assigns an $(\sze+1)$-exponential number to every $\relD$-class.
  \item There is a $\vFoF3{\DS(\sze,\nn)}$-formula
  $\fEq{\DS(\sze,\nn)}(\xx,\yy)$ whose length grows polynomially as $\nn$ grows,
  such that for all $\ea,\eb \in \domA$:
  \[
    \miff{
      \StrA \vDash \fEq{\DS(\sze,\nn)}(\ea,\eb)}{
      \Data{\DS(\sze,\nn)}\StrA\relD[\ea] = \Data{\DS(\sze,\nn)}\StrA\relD[\eb]}.
  \]
  \item There is a $\vFoF3{\DS(\sze,\nn)}$-formula $\fZero{\DS(\sze,\nn)}(\xx)$,
  whose length grows polynomially as $\nn$ grows, such that for all
  $\ea \in \domA$:
  \[
    \miff{
      \StrA \vDash \fZero{\DS(\sze,\nn)}(\ea)}{
      \Data{\DS(\sze,\nn)}\StrA\relD[\ea] = 0}.
  \]
  \item There is a $\vFoF3{\DS(\sze,\nn)}$-formula $\fMax{\DS(\sze,\nn)}(\xx)$,
  whose length grows polynomially as $\nn$ grows, such that for all
  $\ea \in \domA$:
  \[
    \miff{
      \StrA \vDash \fMax{\DS(\sze,\nn)}(\ea)}{
      \Data{\DS(\sze,\nn)}\StrA\relD[\ea] = \tetr{\sze+1}2\nn - 1}.
  \]
  \item There is a $\vFoF3{\DS(\sze,\nn)}$-formula
  $\fLess{\DS(\sze,\nn)}(\xx,\yy)$, whose length grows polynomially as $\nn$
  grows, such that for all $\ea, \eb \in \domA$:
  \[
    \miff{
      \StrA \vDash \fLess{\DS(\sze,\nn)}(\ea,\eb)}{
      \Data{\DS(\sze,\nn)}\StrA\relD[\ea] <
      \Data{\DS(\sze,\nn)}\StrA\relD[\eb]}.
  \]
  \item There is a $\vFoF3{\DS(\sze,\nn)}$-formula
  $\fSucc{\DS(\sze,\nn)}(\xx,\yy)$, whose length grows polynomially as $\nn$
  grows, such that for all $\ea, \eb \in \domA$:
  \[
    \miff{
      \StrA \vDash \fSucc{\DS(\sze,\nn)}(\ea,\eb)}{
      \Data{\DS(\sze,\nn)}\StrA\relD[\eb] = \Data{\DS(\sze,\nn)}\StrA\relD[\ea]
      + 1}.
  \]
  \item For every $\posp \in [0, \tetr{\sze+1}2\nn - 1]$, there is a
  $\vFoF3{\DS(\sze,\nn)}$-formula $\fDataA{\DS(\sze,\nn)}\posp(\xx)$, whose
  length grows polynomially as $\nn$ and $\posp$ grow, such that for all
  $\ea \in  \domA$:
  \[
    \miff{
      \StrA \vDash \fDataA{\DS(\sze,\nn)}\posp(\ea)}{
      \Data{\DS(\sze,\nn)}\StrA\relD[\ea] = \posp}.
  \]
\end{enumerate}
\end{definition}
It is easy to check that the previous section defines a $2$-exponential setup.
Suppose that we have an $(\sze+1)$-exponential setup having predicate signature
$\DS(\sze,\nn)$. Analogously to the previous section, we will describe an
$(\sze+2)$-exponential setup $\DS(\sze+1,\nn) = \DS(\sze,\nn) + \seq{\se}$ which
is based on $\DS(\sze,\nn)$, where $\se = \see{\sze+1}$ is the new coarsest
builtin equivalence symbol in $\DS(\sze+1,\nn)$. Define the following formulas:
\begin{equation*}
\begin{aligned}
  &\fposeq{\DS(\sze+1,\nn)}(\xx,\yy) = \fEq{\DS(\sze,\nn)}(\xx,\yy) \\
  &\fbitO{\DS(\sze+1,\nn)}(\xx) = \forall\yy(\se(\yy,\xx) \land 
    \fposeq{\DS(\sze+1,\nn)}(\yy,\xx) \limp \yy = \xx) \\
  &\fbitI{\DS(\sze+1,\nn)}(\xx) = \exists\yy(\se(\yy,\xx) \land
    \fposeq{\DS(\sze+1,\nn)}(\yy,\xx) \land \yy \neq \xx) \\
  &\fposzero{\DS(\sze+1,\nn)}(\xx) = \fZero{\DS(\sze,\nn)}(\xx) \\
  &\fposmax{\DS(\sze+1,\nn)}(\xx) = \fMax{\DS(\sze,\nn)}(\xx) \\
  &\fposless{\DS(\sze+1,\nn)}(\xx,\yy) = \se(\xx,\yy) \land
    \fLess{\DS(\sze,\nn)}(\xx,\yy) \\
  &\fpossucc{\DS(\sze+1,\nn)}(\xx,\yy) = \se(\xx,\yy) \land
    \fSucc{\DS(\sze,\nn)}(\xx,\yy) \\
  &\fposfull{\DS(\sze+1,\nn)}(\xx,\yy) = \forall\xx\exists\yy\Big(\se(\yy,\xx) \land \fposzero{\DS(\sze+1,\nn)}(\yy)\Big) \land \\
    &\quad\forall\xx\Big(\lnot\fposmax{\DS(\sze+1,\nn)}(\xx) \limp
    \exists\yy\fpossucc{\DS(\sze+1,\nn)}(\xx,\yy)\Big) \\
\end{aligned}
\end{equation*}
\begin{comment}
We can employ an $\sze$-exponential $\nn$-bit counter setup to construct an
$(\sze+1)$-exponential $\nn$-bit counter setup by introducing a new coarsest
builtin equivalence symbol $\se = \see{\sze+1}$ and a new control unary
predicate symbol $\sco$, similarly to what we did in~\Cref{sec:hardness-one}:
the $\relD$-classes within a single $\relE$-class correspond to the
$\sze$-exponential bit positions of a $(\sze+1)$-exponential number---the
$(\sze+1)$-exponential data at the $\relE$-class. If there is a single
$\relD$-class at position $\posp$ in a $\relE$-class, the bit at that position
is $0$; if there are more $\relD$-classes at position $\posp$, the bit at that
position is $1$, so we may define the formulas:
\[
  \fbitO{\DS_\nn^{\sze+1}}(\xx) = \sco(\xx) \land \forall\yy \sco(\yy) \land
  \se(\yy,\xx) \land \feeqData\sze\DS(\yy,\xx) \limp \yy = \xx, \text{etc.}
\]
Repeating the argument from section~\Cref{sec:hardness-one}, by reducing the
$\ceNExpTime\sze$-complete $\sze$-exponential tiling problem to (finite)
satisfiability of $\Lvp\Fo\nonv1\Eea\sze\agrefine$, we obtain:
\begin{proposition}
The (finite) satisfiability problem for the monadic first-order logic with
$\sze \in \pNats$ equivalence symbols in refinement
$\Lvp\Fo\nonv1\Eea\sze\agrefine$ is $\ceNExpTime{(\sze+1)}$-hard.
By \Cref{prop:global-to-refine-n} and \Cref{prop:local-to-refine-n}, the same
holds for the logics $\Lvp\Fo\nonv1\Eea\sze\agglobal$ and
$\Lvp\Fo\nonv1\Eea\sze\aglocal$.
\end{proposition}

\begin{proposition}
The (finite) satisfiability problem for the monadic first-order logic with many
equivalence symbols in refinement $\Lvp\Fo\nonv1\Eea\nosze\agrefine$ is
$\cElementary$-hard.

By \Cref{prop:global-to-refine} and \Cref{prop:local-to-refine}, the same claim
holds for the logics $\Lvp\Fo\nonv1\Eea\nosze\agglobal$ and
$\Lvp\Fo\nonv1\Eea\nosze\aglocal$.
\end{proposition}
\end{comment}