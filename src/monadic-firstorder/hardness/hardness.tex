% Hardness of the monadic logic with equivalence
In this section we show that the (finite) satisfiability of monadic first-order
logic with a single equivalence symbol $\Lvp\Fo\nonv1\Eea1\noag$ is
$\ceNExpTime2$-hard by reducing the doubly exponential tiling problem to such
satisfiability. Our strategy is to employ a counter setup of $\szu$ unary
predicate symbols to encode the exponentially many positions of a binary
encoding of a doubly exponentially bounded quantity, encoding the coordinaes of
a cell of the doubly exponential tiling square.

Consider the counter setup $\CS(\szu) = \seq{\suu1,\suu2,\dots,\suu\szu}$ for
$\szu \in \pNats$. Recall that the intention of a counter setup is to encode an
arbitrary exponentially bounded value at every element of a structure.
Let $\DS(\szu, \sco) = \CS(\szu) + \seq{\sco, \sd}$ be a predicate signature
enriching $\CS(\szu)$ with the \emph{control predicate symbol} $\sco$ and the
builtin equivalence symbol $\sd$.
We will define a system where every $\sd$-equivalence class includes
exponentially many cells satisfying the control predicate symbol. These cells
will correspond to the exponentially many positions of the binary encoding of a
doubly exponential value for the $\sd$-class. The bit values at each cell
position will be encoded by the cardinality of that cell: bit value $0$ if the
cardinality of the cell is $1$ and bit value $1$ if the cardinality is greater
than $1$.
Call the data $\data\CS\StrA\ea$, encoded by the counter setup at $\ea$ the
\emph{position} of $\ea$.

\begin{definition}
Let $\StrA$ be a $\DS = \DS(\szu,\sco)$-structure. Define the quantifier-free
$\vFoF2\DS$-formula $\feqpos\DS(\xx,\yy)$ by:
\[
  \feqpos\DS(\xx,\yy) = \sco(\xx) \land \sco(\yy) \land \feq\CS(\xx,\yy).
\]
\end{definition}
Then $\StrA \vDash \feqpos\DS(\ea,\eb)$ iff $\ea$ and $\eb$ satisfy the control
predicate and are at the same positions:
$\data\CS\StrA\ea = \data\CS\StrA\eb$.

\begin{definition}
Define the quantifier-rank-$1$ $\vFoF2\DS$-formula $\fbitO\DS(\xx)$ by:
\[
  \fbitO\DS(\xx) = \sco(\xx) \land \forall\yy 
  \sco(\yy) \land \sd(\yy,\xx) \land \feq\CS(\yy,\xx)
  \limp \yy = \xx.
\]
\end{definition}
Then $\StrA \vDash \fbitO\DS(\ea)$ iff the cell of $\ea$ has cardinality $1$.

\begin{definition}
Define the quantifier-rank-$1$ $\vFoF2\DS$-formula $\fbitI\DS(\xx)$ by:
\[
  \fbitI\DS(\xx) = \exists\yy \sco(\yy) \land \sd(\xx,\yy) \land
  \feq\CS(\yy,\xx) \land \yy \neq \xx.
\]
\end{definition}
Then $\StrA \vDash \fbitI\DS(\ea)$ iff the cell of $\ea$ has cardinality greater
than $1$.

\begin{definition}
Define the quantifier-free $\vFoF2\DS$-formula $\fposzero\DS(\xx)$ by:
\[
  \fposzero\DS(\xx) = \sco(\xx) \land
  \bigwedge_{1 \leq \ii \leq \szu} \lnot\suu\ii(\xx).
\]
\end{definition}
Then $\StrA \vDash \fposzero\DS(\ea)$ iff the position of $\ea$ is $0$.

\begin{definition}
Define the quantifier-free $\vFoF2\DS$-formula $\fposmax\DS(\xx)$ by:
\[
  \fposmax\DS(\xx) = \sco(\xx) \land
  \bigwedge_{1 \leq \ii \leq \szu} \suu\ii(\xx).
\]
\end{definition}
Then $\StrA \vDash \fposmax\DS(\ea)$ iff the position of $\ea$ is
$\largtbit\szu$.

\begin{definition}
Define the quantifier-free $\vFoF2\DS$-formula $\fpossucc\DS(\xx,\yy)$ by:
\[
  \fpossucc\DS(\xx,\yy) = \sco(\xx) \land \sco(\yy) \land \sd(\xx,\yy) \land
  \fsucc\CS(\xx,\yy).
\]
\end{definition}
Then $\StrA \vDash \fpossucc\DS(\ea,\eb)$ iff $\ea$ and $\eb$ are in the same
$\relD$-class and the position of $\eb$ is the successor of the position of
$\ea$.

\begin{definition}
Define the quantifier-free $\vFoF2\DS$-formula $\fposless\DS(\xx,\yy)$ by:
\[
  \fposless\DS(\xx,\yy) = \sco(\xx) \land \sco(\yy) \land \sd(\xx,\yy) \land
  \fless\CS(\xx,\yy).
\]
\end{definition}
Then $\StrA \vDash \fposless\DS(\ea,\eb)$ iff $\ea$ and $\eb$ are in the same
$\relD$-class and the position of $\ea$ is less than the position of $\eb$.

\begin{definition}
For $\posp \in \tBitnums\szu$, define the $\vFoF2\DS$-formula
$\fposp\DS\posp(\xx)$ recursively by:
\[
  \fposp\DS0(\xx) = \fposzero\DS(\xx)
\]
and for $\posp < \largtbit\szu$:
\[
  \fposp\DS{(\posp+1)}(\xx) = \exists\yy \fposp\DS\posp(\yy) \land
  \fpossucc\DS(\xx,\yy).
\]
In this case, for the formula to be a two-variable formula,
the formula $\fposp\DS\posp(\yy)$ is obtained from $\fposp\DS\posp(\xx)$ by
swapping all occurences (not only the unbounded ones) of the variables $\xx$ and
$\yy$.
Note that the length of the formula $\fposp\DS\posp(\xx)$ is linear in $\posp$.
\end{definition}
Then $\StrA \vDash \fposp\DS\posp(\ea)$ iff $\posp$ is the position of $\ea$.

\begin{definition}
Define the closed $\vFoF2\DS$-sentence $\ffullpos\DS$ by:
\begin{align*}
  \ffullpos\DS =
  \left(\forall\yy\exists\xx \sco(\xx) \land \sd(\xx,\yy) \land
    \fposzero\DS(\xx)\right) \land \\
  \left(
  \forall\xx \sco(\xx) \land \lnot\fposmax\DS(\xx) \limp
  \exists\yy \fpossucc\DS(\xx,\yy)\right).
\end{align*}
\end{definition}
The first part of this formula asserts that every $\sd$-class has an element at
position $0$. The second part asserts that if $\ea$ is an element at position
$\posp$, that is not the largest possible, there exists an element $\eb$ in the
same $\sd$-class at position $\posp + 1$.
Therefore in any model of $\ffullpos\DS$, every $\sd$-class has $2^\szu$ cells
satisfying $\sco$. For example, in particular, every $\sd$-class has cardinality
at least $2^\szu$. For the rest of the section, suppose that $\StrA \vDash
\ffullpos\DS$.

\begin{definition}
Let $\StrA$ be a $\DS$-structure. Let $\relD = \at\StrA\sd$.
Define the function
$\gls{Data-D-A} : \Ecl\relD \to \tBitstrings{2^\szu}$ by:
\[
  \Data\DS\StrA_\posp \setS = \begin{cases}
  1 &\text{if } \data\CS\StrA(\ea) = \posp-1 \text{ implies } 
  \StrA \vDash \fbitI\DS(\ea)
  \text{ for all } \ea \in \setS
  \\
  0 &\text{otherwise.}
  \end{cases}
\]
\end{definition}
\begin{definition}
Define the quantifier-rank-$1$ $\vFoF2\DS$-formula $\gls{fZero-D-x}$ by:
\[
  \fZero\DS(\xx) = \forall\yy \sco(\yy) \land \sd(\yy,\xx) \limp
  \fbitO\DS(\yy).
\] 
\end{definition}

Then $\StrA \vDash \fZero\DS(\ea)$ iff $\Data\DS\StrA(\ea) = \benc{0}$.

\begin{definition}
Let $\nN \in \tBitnums{2^\szu}$ be a $\tbit$-bit number, where
$\tbit = \bsz\nN \leq 2^\szu$.
Define the $\vFoF2\DS$-formula $\fDataN\DS\nN(\xx)$ by:
\begin{align*}
  \fDataN\DS\nN(\xx) = \forall\yy \sco(\yy) \land \sd(\yy,\xx) \limp
  \forall\xx \sco(\xx) \land \sd(\xx,\yy) \limp \\
  \left(\bigwedge_{0 \leq \posp < \tbit} \fposp\DS\posp(\yy) \limp
  \fbit\DS{(\benc\nN_{\posp+1})}(\yy)\right) \land \\
  \left(\fposp\DS{(\tbit-1)}(\yy) \land \fposless\DS(\yy,\xx) \limp
  \fbitO\DS(\xx) \right).
\end{align*}
\end{definition}
The first part of this formula asserts that the bits at the first $\tbit$
positions of the $\sd$-class of $\xx$ encode the number $\nN$.
The second part asserts that all the remaining bits at larger positions are
zeroes.
Note that the length of this formula is polynomially bounded by $\tbit$, the
bitsize of $\nN$.
We have $\StrA \vDash \fDataN\DS\nN(\ea)$ iff $\bdec{\data\DS\StrA(\ea)} = \nN$.

\begin{definition}
Define the $\vFoF4\DS$-formula $\fSucc\DS(\xx,\yy)$ by:
\begin{align}
  \fSucc\DS(\xx,\yy) = \exists\xxp\exists\yyp
  \sd(\xxp,\xx) \land \sd(\yyp,\yy) \land \nonumber \\
  \left(
    \feqpos\DS(\xxp,\yyp) \land \fbitO\DS(\xxp) \land \fbitI\DS(\yyp)
  \right) \land \tag{\ref{Succ1}} \\
  \left(
    \forall\xxpp \fposless\DS(\xxpp,\xxp) \limp \fbitI\DS(\xxpp)
  \right) \land \tag{\ref{Succ2}} \\
  \left(
    \forall\yypp \fposless\DS(\yypp,\yyp) \limp \fbitO\DS(\yypp)
  \right) \land \tag{\ref{Succ3}} \\
  \Big(
    \forall\xxpp \fposless\DS(\xxp,\xxpp) \limp \exists\yypp \sd(\yypp,\yyp)
    \land \tag{\ref{Succ4}} \\
    \feqpos\DS(\yypp,\xxpp) \land (\fbitO\yypp \lequ \fbitO\xxpp)
  \Big). \nonumber
\end{align}
By rearrangement and reusing variables, this can be also written using just
three variables (but not with just two variables).
\end{definition}
Then $\StrA \vDash \fSucc\DS(\ea,\eb)$ iff 
$\bdec{\data\DS\StrA(\eb)} = 1 + \bdec{\data\DS\StrA(\ea)}$.

\begin{definition}
Define the $\vFoF3\DS$-sentence $\ffullData\DS$ by:
\[
  \ffullData\DS = \exists\xx \fZero\DS(\xx) \land
  \forall\xx \lnot\fMax\DS(\xx) \limp \exists\yy \fSucc\DS(\xx,\yy).
\]
\end{definition}
If $\StrA$ satisfies $\ffullData\DS$ then $\StrA$ contains a $\relD$-class
having data $\benc\nn$ for every $\nn \in [0, \largtbit{2^\szu}]$.

\begin{definition}
Define the $\vFoF4\DS$-sentence $\falldiffData\DS$ by:
\begin{align*}
  &\falldiffData\DS = \forall\xx \forall\yy \lnot\sd(\xx,\yy) \limp \\
  &\exists\xxp \exists\yyp \sd(\xxp,\xx) \land \sd(\yyp,\yy) \land
  \feqpos\DS(\xxp,\yyp) \land \lnot (\fbitO\DS(\xxp) \lequ \fbitO\DS(\yyp)).
\end{align*}
By rearrangement and reusing variables, this can be also written using just
three variables (but not with just two variables).
\end{definition}
If $\StrA$ satisfies $\falldiffData\DS$ then all $\relD$-classes in $\StrA$ have
different data.
% TODO: Add proper glossaries to the formulas of this section

Recall from \cref{sec:complexity} that an instance of the \emph{doubly
exponential tiling problem} is a tuple $(\tileT, \isg, \nn)$,
where $\tileT = \set{\tilett1,\tilett2,\dots,\tilett\cardtf}$ is a tile family.







