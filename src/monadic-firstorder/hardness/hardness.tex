% Hardness of the monadic logic with equivalence
In this section we show that the (finite) satisfiability of monadic first-order
logic with a single equivalence symbol $\Lvp\Fo\nonv1\Eea1\noag$ is
$\ceNExpTime2$-hard by reducing the doubly exponential tiling problem to such
satisfiability. Our strategy is to employ a counter setup of $\szu$ unary
predicate symbols to encode the exponentially many positions of a binary
encoding of a doubly exponentially bounded quantity, encoding the coordinaes of
a cell of the doubly exponential tiling square.

Consider the counter setup $\CS(\szu) = \seq{\suu1,\suu2,\dots,\suu\szu}$ for
$\szu \in \pNats$. Recall that the intention of a counter setup is to encode an
arbitrary exponentially bounded value at every element of a structure.
Let $\DS(\szu) = \CS(\szu) + \seq{\sd}$ be a predicate signature enriching
$\CS(\szu)$ with the builtin equivalence symbol $\sd$.
We will define a system where every $\sd$-equivalence class includes
exponentially many cells. These cells
will correspond to the exponentially many positions of the binary encoding of a
doubly exponential value for the $\sd$-class. The bit values at each cell
position will be encoded by the cardinality of that cell: bit value $0$ if the
cardinality of the cell is $1$ and bit value $1$ if the cardinality is greater
than $1$. This will allow us to encode a doubly exponential value at each
$\sd$-class. Call the data $\data\CS\StrA\ea$, encoded by the counter setup
at $\ea$ the \emph{position} of $\ea$.

Let $\StrA$ be a $\DS = \DS(\szu)$-structure.
\begin{definition}
Define the quantifier-free $\vFoF2\DS$-formula $\fposeq\DS(\xx,\yy)$ by:
\[
  \fposeq\DS(\xx,\yy) = \feq\CS(\xx,\yy).
\]
\end{definition}
Then $\StrA \vDash \fposeq\DS(\ea,\eb)$ iff $\ea$ and $\eb$ are at the same
positions (in possibly distinct $\sd$-classes):
$\data\CS\StrA\ea = \data\CS\StrA\eb$.

\begin{definition}
Define the quantifier-rank-$1$ $\vFoF2\DS$-formula $\fbitO\DS(\xx)$ by:
\[
  \fbitO\DS(\xx) = \forall\yy \left(
  \sd(\yy,\xx) \land \fposeq\DS(\yy,\xx) \limp \yy = \xx\right).
\]
\end{definition}
Then $\StrA \vDash \fbitO\DS(\ea)$ iff the cell of $\ea$ has cardinality $1$.

\begin{definition}
Define the quantifier-rank-$1$ $\vFoF2\DS$-formula $\fbitI\DS(\xx)$ by:
\[
  \fbitI\DS(\xx) = \exists\yy \left(
  \sd(\yy,\xx) \land \fposeq\DS(\yy,\xx) \land \yy \neq \xx\right).
\]
\end{definition}
Then $\StrA \vDash \fbitI\DS(\ea)$ iff the cell of $\ea$ has cardinality greater
than $1$.

\begin{definition}
Define the quantifier-free $\vFoF2\DS$-formula $\fposzero\DS(\xx)$ by:
\[
  \fposzero\DS(\xx) = \bigwedge_{1 \leq \ii \leq \szu} \lnot\suu\ii(\xx).
\]
\end{definition}
Then $\StrA \vDash \fposzero\DS(\ea)$ iff the position of $\ea$ is $0$.

\begin{definition}
Define the quantifier-free $\vFoF2\DS$-formula $\fposmax\DS(\xx)$ by:
\[
  \fposmax\DS(\xx) = \bigwedge_{1 \leq \ii \leq \szu} \suu\ii(\xx).
\]
\end{definition}
Then $\StrA \vDash \fposmax\DS(\ea)$ iff the position of $\ea$ is the largest
$\szu$-bit number $\largtbit\szu$.

\begin{definition}
Define the quantifier-free $\vFoF2\DS$-formula $\fposless\DS(\xx,\yy)$ by:
\[
  \fposless\DS(\xx,\yy) = \sd(\xx,\yy) \land \fless\CS(\xx,\yy).
\]
\end{definition}
Then $\StrA \vDash \fposless\DS(\ea,\eb)$ iff $\ea$ and $\eb$ are in the same
$\sd$-class and the position of $\ea$ is less than the position of $\eb$.

\begin{definition}
Define the quantifier-free $\vFoF2\DS$-formula $\fpossucc\DS(\xx,\yy)$ by:
\[
  \fpossucc\DS(\xx,\yy) = \sd(\xx,\yy) \land \fsucc\CS(\xx,\yy).
\]
\end{definition}
Then $\StrA \vDash \fpossucc\DS(\ea,\eb)$ iff $\ea$ and $\eb$ are in the same
$\sd$-class and the position of $\eb$ is the successor of the position of
$\ea$.

\begin{definition}
Define the closed $\vFoF2\DS$-sentence $\fposfull\DS$ by:
\begin{align*}
  \fposfull\DS =
  \forall\xx\exists\yy \Big(\sd(\yy,\xx) \land \fposzero\DS(\yy)\Big) \land \\
  \forall\xx \Big(
  \lnot\fposmax\DS(\xx) \limp \exists\yy \fpossucc\DS(\xx,\yy)\Big).
\end{align*}
\end{definition}
The first part of this formula asserts that every $\sd$-class has an element at
position $0$. The second part asserts that if $\ea$ is an element at position
$\posp$, that is not the largest possible, there exists an element $\eb$ in the
same $\sd$-class at position $\posp + 1$.
Therefore in any model of $\fposfull\DS$, every $\sd$-class has $2^\szu$ cells.
For example, in particular, every $\sd$-class has cardinality at least $2^\szu$.
For the rest of the section, suppose that $\StrA \vDash \fposfull\DS$.

\begin{definition}
For every $\szu$-bit number $\posp \in \tBitnums\szu$, define the
$\vFoF2\DS$-formula $\fposA\DS\posp(\xx)$ recursively by:
\[
  \fposA\DS0(\xx) = \fposzero\DS(\xx)
\]
and for $\posp \in [0, \largtbit\szu-1]$:
\[
  \fposA\DS{(\posp+1)}(\xx) = \exists\yy \Big(\fposA\DS\posp(\yy) \land
  \fpossucc\DS(\yy,\xx)\Big).
\]
In this case, for the formula to be a two-variable formula,
the formula $\fposA\DS\posp(\yy)$ is obtained from $\fposA\DS\posp(\xx)$ by
swapping all occurences (not only the unbounded ones) of the variables $\xx$ and
$\yy$\footnote{this is reminiscent to the process of defining a standard
translation of monadic logic to the two-variable first-order fragment}.
Note that the length of the formula $\fposA\DS\posp(\xx)$ grows linearly as
$\posp$ grows.
\end{definition}
Then $\StrA \vDash \fposA\DS\posp(\ea)$ iff $\posp$ is the position of $\ea$.

\begin{definition}
Let $\StrA$ be a $\DS$-structure. Let $\relD = \at\StrA\sd$.
Define the function
$\gls{Data-D-A} : \Ecl\relD \to \tBitstrings{2^\szu}$, assiging a
$2^\szu$-bit bitstring to any $\relD$-class $\eclX$ by:
\[
  \Data\DS\StrA_\posp \eclX = \begin{cases}
  1 &\text{if } \data\CS\StrA(\ea) = (\posp-1) \text{ implies } 
  \StrA \vDash \fbitI\DS(\ea)
  \text{ for all } \ea \in \eclX
  \\
  0 &\text{otherwise}
  \end{cases}
\]
for $\posp \in [1,2^\szu]$.
\end{definition}
\begin{definition}
Define the quantifier-rank-$1$ $\vFoF2\DS$-formula $\gls{fZero-D-x}$ by:
\[
  \fZero\DS(\xx) = \forall\yy\Big(\sd(\yy,\xx) \limp \fbitO\DS(\yy)\Big).
\] 
\end{definition}
Then $\StrA \vDash \fZero\DS(\ea)$ iff the data at the $\relD$-class of $\ea$
encodes $0$: $\bdec{\Data\DS\StrA\relD[\ea]} = 0$.

\begin{definition}
Define the quantifier-rank-$1$ $\vFoF2\DS$-formula $\gls{fMax-D-x}$ by:
\[
  \fLargest\DS(\xx) = \forall\yy \Big(\sd(\yy,\xx) \limp \fbit\DS1(\yy)\Big).
\]
\end{definition}
Then $\StrA \vDash \fZero\DS(\ea)$ iff the data at the $\relD$-class of $\ea$
encodes the largest $2^\szu$-bit number:
$\bdec{\Data\DS\StrA\relD[\ea]} = \largtbit{2^\szu}$.

\begin{definition}
Let $\nM \in \tBitnums{2^\szu}$ be a $\tbit$-bit number (where $\tbit \leq
2^\szu$).
Define the $\vFoF2\DS$-formula $\fDataA\DS\nM(\xx)$ by:
\begin{align*}
  \fDataA\DS\nM(\xx) =
  \forall\yy \biggl(\sd(\yy,\xx) \limp 
  \bigwedge_{0 \leq \posp < \tbit} \left(\fposA\DS\posp(\yy) \limp
  \fbit\DS{(\benc\nM_{\posp+1})}(\yy)\right) \land \\
  \biggl. \forall\xx \Big(
  \fposA\DS{(\tbit-1)}(\yy) \land \fposless\DS(\yy,\xx) \limp
  \fbitO\DS(\xx) \Big)\biggr).
\end{align*}
\end{definition}
The first part of this formula asserts that the bits at the first $\tbit$
positions of the $\sd$-class of $\xx$ encode the number $\nM$.
The second part asserts that all the remaining bits at larger positions are
zeroes.
Note that the length of this formula is polynomially bounded by $\tbit$, the
bitsize of $\nM$.
We have $\StrA \vDash \fDataA\DS\nM(\ea)$ iff the data at the $\relD$-class of
$\ea$ encodes $\nM$: $\bdec{\Data\DS\StrA\relD[\ea]} = \nM$.

\begin{definition}
Define the $\vFoF6\DS$-formula $\fLess\DS(\xx,\yy)$ by:
\begin{align}
  \fLess\DS(\xx,\yy) = \exists\xxp\exists\yyp\Bigg(
  \sd(\xxp,\xx) \land \sd(\yyp,\yy) \land \nonumber \\
  \Big(
    \fposeq\DS(\xxp,\yyp) \land \fbitO\DS(\xxp) \land \fbitI\DS(\yyp)
  \Big) \land \tag{\ref{Less1}} \\
  \forall\xxpp \Big( \fposless\DS(\xxp,\xxpp) \limp \exists\yypp
  \Big(\sd(\yypp,\yyp) \land \nonumber \\
  \fposeq\DS(\yypp,\xxpp) \land (\fbitO\DS(\yypp) \lequ \fbitO\DS(\xxpp))
  \Big)\Big)\Bigg). \tag{\ref{Less2}}
\end{align}
\end{definition}
Then $\StrA \vDash \fLess\DS(\ea,\eb)$ iff
$\bdec{\Data\DS\StrA\relD[\ea]} < \bdec{\Data\DS\StrA\relD[\eb]}$.
By rearrangement and reusing variables, this can be also written using just
three variables (but not using just two variables). Indeed, $\fLess\DS(\xx,\yy)$
is logically equivalent to:
\begin{align}
  \exists\zz \Bigg(\sd(\zz,\xx) \land \exists\xx \Bigg(\xx=\zz \land
  \exists\zz \Bigg(\sd(\zz,\yy) \land \exists\yy \Bigg(\yy=\zz \land \nonumber\\
  \Big(\fposeq\DS(\xx,\yy) \land \fbitO\DS(\xx) \land \fbitI\DS(\yy)\Big)
  \land \tag{\ref{Less1}} \\
  \forall\zz\Big(
    \fposless\DS(\xx,\zz) \limp \exists\xx\Big(\xx=\zz \land
    \exists\zz\Big(\sd(\zz,\yy) \land \exists\yy\Big(\yy=\zz \land \nonumber \\
    \fposeq\DS(\yy,\xx) \land (\fbitO\DS(\yy) \lequ
    \fbitO\DS(\xx))\Big)\Big)\Big)\Big)\Bigg)\Bigg)\Bigg)\Bigg).
    \tag{\ref{Less2}}
\end{align}

\begin{definition}
Define the $\vFoF6\DS$-formula $\fSucc\DS(\xx,\yy)$ by:
\begin{align}
  \fSucc\DS(\xx,\yy) = \exists\xxp\exists\yyp\Bigg(
  \sd(\xxp,\xx) \land \sd(\yyp,\yy) \land \nonumber \\
  \Big(
    \fposeq\DS(\xxp,\yyp) \land \fbitO\DS(\xxp) \land \fbitI\DS(\yyp)
  \Big) \land \tag{\ref{Succ1}} \\
  \forall\xxpp \Big(\fposless\DS(\xxpp,\xxp) \limp \fbitI\DS(\xxpp)
  \Big) \land \tag{\ref{Succ2}} \\
  \forall\yypp \Big( \fposless\DS(\yypp,\yyp) \limp \fbitO\DS(\yypp)
  \Big) \land \tag{\ref{Succ3}} \\
  \forall\xxpp\Big( \fposless\DS(\xxp,\xxpp) \limp \exists\yypp\Big(
    \sd(\yypp,\yyp) \land \nonumber \\
    \fposeq\DS(\yypp,\xxpp) \land (\fbitO\DS(\yypp) \lequ \fbitO\DS(\xxpp))
  \Big)\Big)\Bigg). \tag{\ref{Succ4}}
\end{align}
By rearrangement and reusing variables, this can be also written using just
three variables (but not using just two variables).
\end{definition}
Then $\StrA \vDash \fSucc\DS(\ea,\eb)$ iff 
$\bdec{\Data\DS\StrA\relD[\eb]} = 1 + \bdec{\Data\DS\StrA\relD[\ea]}$.

\begin{definition}
Define the $\vFoF3\DS$-sentence $\fFull\DS$ by:
\[
  \fFull\DS = \exists\xx \fZero\DS(\xx) \land
  \forall\xx \Big(\lnot\fLargest\DS(\xx) \limp \exists\yy
  \fSucc\DS(\xx,\yy)\Big).
\]
\end{definition}
If $\StrA$ satisfies $\fFull\DS$ then $\StrA$ contains a $\sd$-class of
encoding any possible data: for every $\nM \in [0,\largtbit{2^\szu}]$,
there is a $\sd$-class $\eclX$ such that $\bdec{\Data\DS\StrA\eclX} = \nM$.
\begin{definition}
Define the $\vFoF4\DS$-formula $\fEq\DS(\xx,\yy)$ by:
\begin{align*}
  \fEq\DS(\xx,\yy) = \forall\xxp\forall\yyp \Big(
  \sd(\xxp,\xx) \land \sd(\yyp,\yy) \land \\ \fposeq\DS(\xxp,\yyp) \limp 
  (\fbitO\DS(\xxp) \lequ \fbitO\DS(\yyp))\Big).
\end{align*}
By rearrangement and reusing variables, this can be also written using just
three variables (but not using just two variables).
\end{definition}
Then $\StrA \vDash \fEq\DS(\xx,\yy)$ iff
$\Data\DS\StrA\relD[\ea] = \Data\DS\StrA\relD[\eb]$.
\begin{definition}
Define the $\vFoF4\DS$-sentence $\fAlldiff\DS$ by:
\begin{align*}
  \fAlldiff\DS = \forall\xx \forall\yy\Big( \lnot\sd(\xx,\yy) \limp 
  \exists\xxp \exists\yyp\Big( \sd(\xxp,\xx) \land \sd(\yyp,\yy)  \land \\
  \fposeq\DS(\xxp,\yyp) \land \lnot (\fbitO\DS(\xxp) \lequ
  \fbitO\DS(\yyp))\Big)\Big).
\end{align*}
By rearrangement and reusing variables, this can be also written using just
three variables (but not using just two variables).
\end{definition}
If $\StrA$ satisfies $\fAlldiff\DS$ then all $\relD$-classes in $\StrA$
encode different data.
% TODO: Add proper glossaries to the formulas of this section

Recall from \Cref{sec:complexity} that an instance of the \emph{doubly
exponential tiling problem} is an initial condition
$\icond = \seq{\icondt1,\icondt2,\dots,\icondt\nn} \subseteq \tiles = [1,\kk]$
of tiles from the ``Turing-complete'' domino system
$\tcdomsys = (\tiles,\horm,\verm)$, where $\horm,\verm \subseteq
\tiles\cprod\tiles$ are the horizontal and vertical matching relations.
We need to define a predicate signature capable enough to express a doubly
exponential grid of tiles. Consider the predicate signature
\[
  \DS = \seq{\suuu\horm1,\suuu\horm2,\dots,\suuu\horm\nn;
  \suuu\verm1,\suuu\verm2,\dots,\suuu\verm\nn; 
  \suuu\tiles1,\suuu\tiles2,\dots,\suuu\tiles\kk;\sd}.
\]
It has the following relevant subsignatures:
\begin{itemize}
  \item $\DS^\horm = 
  \seq{\suuu\horm1,\suuu\horm2,\dots,\suuu\horm\nn,\sd}$ encodes the
  horizontal index of a tile
  \item $\DS^\verm =
  \seq{\suuu\verm1,\suuu\verm2,\dots,\suuu\verm\nn,\sd}$ encodes the
  vertical index of a tile
  \item $\DS^{\horm\verm} =
  \seq{\suuu\horm1,\suuu\horm2,\dots,\suuu\horm\nn,
  \suuu\verm1,\suuu\verm2,\dots,\suuu\verm\nn,\sd}$ encodes the combined
  horizontal and vertical index of a tile; we need this to define the full grid
  \item $\DS^\tiles = \seq{\suuu\tiles1,\suuu\tiles2,\dots,\suuu\tiles\kk}$
  encodes the type of a tile.
\end{itemize}
Let $\StrA$ be a $\DS$-structure satisfying $\fposfull{\DS^{\horm\verm}}$ and
let $\relD = \at\StrA\sd$.
The sentence
\begin{equation}\label{fla:tiling-first}
  \fFull{\DS^{\horm\verm}} \land \fAlldiff{\DS^{\horm\verm}}
\end{equation}
asserts that the $\relD$-classes form a doubly exponential grid.
The sentence
\begin{equation}
  \forall\xx \Big(\bigwedge_{1 \leq \ii \leq \kk} \suuu\tiles\ii(\xx) \limp
  \bigwedge_{\ii < \jj \leq \kk} \lnot\suuu\tiles\jj(\xx)\Big)
\end{equation}
asserts that every element has a unique type.
The sentence
\begin{equation}
  \forall\xx\forall\yy \Big(\sd(\xx,\yy) \limp
  \bigwedge_{1 \leq \ii \leq \kk} 
  (\suuu\tiles\ii(\xx) \lequ \suuu\tiles\ii(\xx))\Big)
\end{equation}
asserts that all elements in a $\relD$-class have the same type---the type of
the tile corresponding to that $\relD$-class.
For $\jj \in [1,\nn]$, the sentence
\begin{equation}
  \forall\xx \Big(\fDataA{\DS^\horm}{(\jj-1)}(\xx) \land \fZero{\DS^\verm}(\xx)
  \limp \suuu\tiles{\icondt\jj}(\xx)\Big)
\end{equation}
encodes the initial segment in the first row of the square.
The sentence
\begin{equation}
  \forall\xx\forall\yy \Big(\fSucc{\DS^\horm}(\xx,\yy) \land
  \fEq{\DS^\verm}(\xx,\yy) \limp
  \bigvee_{(\ii,\jj) \in \horm} \suuu\tiles{\ii}(\xx) \land
  \suuu\tiles{\jj}(\yy)\Big)
\end{equation}
encodes the horizontal matching condition.
The sentence
\begin{equation}\label{fla:tiling-last}
  \forall\xx\forall\yy \Big(\fSucc{\DS^\verm}(\xx,\yy) \land
  \fEq{\DS^\horm}(\xx,\yy) \limp
  \bigvee_{(\ii,\jj) \in \verm} \suuu\tiles{\ii}(\xx) \land
  \suuu\tiles{\jj}(\yy)\Big)
\end{equation}
encodes the vertical matching condition.

Combining $\fposfull{\DS^{\horm\verm}}$ with the formulas
\ref{fla:tiling-first}--\ref{fla:tiling-last}, we may encode the instance an
instance of the doubly exponential tiling problem as a (finite) satisfiability
of a formula, so we have:
\begin{proposition}
The (finite) satisfiability problem for the monadic first-order logic with a
single equivalence symbol $\Lvp\Fo\nonv1\Eea1\noag$ is $\ceNExpTime2$-hard.
More precisely, even the three-variable fragment $\Lvp\Fo31\Eea1\noag$ has this
property.
\end{proposition}